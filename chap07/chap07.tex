\chapter{神经系统的细胞} \label{chap:chap7}
神经系统的细胞——神经元和胶质细胞——与一般细胞有许多共同特征。 
然而,神经元具有特殊的天赋,能够与体内远处的其他细胞进行精确、快速的通信。 
两个特征赋予了神经元这种能力。


首先,它们具有高度的形态和功能不对称性:神经元的一端有接受树突,另一端有传递轴突。 
这种排列是单向神经元信号传导的结构基础。


其次,神经元既可电兴奋又可化学兴奋。 
神经元的细胞膜包含特殊的蛋白质——离子通道和受体——它们促进特定无机离子的流动,从而重新分配电荷并产生改变跨膜电压的电流。 
这些电荷变化可以沿轴突产生动作电位形式的去极化波,这是信号在神经元内传播的通常方式。 
胶质细胞不太容易兴奋,但它们的膜含有促进离子摄取的转运蛋白,以及从细胞外空间去除神经递质分子的蛋白质,从而调节神经元功能。


根据树突形态、轴突投射模式和电生理特性,有数百种不同类型的神经元。 
这种结构和功能多样性主要由每种神经元细胞类型表达的基因决定。 
尽管神经元都继承了相同的基因组,但每个神经元都表达一组受限的基因,因此只产生某些分子——酶、结构蛋白、膜成分和分泌产物——而不是其他分子。 
在很大程度上,这种表达取决于细胞的发育历史。 
从本质上讲,每个细胞都是它所表达的分子集合。


神经胶质细胞的种类也很多,可以根据其独特的形态、生理和生化特征进行鉴定。 
神经胶质细胞的不同形态表明神经胶质细胞可能与神经元一样异质。 
尽管如此,脊椎动物神经系统中的胶质细胞可分为两大类:大胶质细胞和小胶质细胞。 
大胶质细胞主要分为三种类型:少突胶质细胞、雪旺细胞和星形胶质细胞。 
在人脑中,大约 90\% 的胶质细胞是大胶质细胞。 
其中,大约一半是髓鞘生成细胞(少突胶质细胞和雪旺细胞),一半是星形胶质细胞。 
少突胶质细胞为中枢神经系统 (CNS) 中某些神经元的轴突提供绝缘髓鞘(图 7-1A)。 
雪旺细胞使周围神经系统中神经元的轴突形成髓鞘(图 7-1B); 非髓鞘化雪旺细胞具有其他功能,包括促进神经肌肉突触的发育、维持和修复。 
星形胶质细胞因其不规则的大致星形细胞体和大量突起而得名; 它们支持神经元并以多种方式调节神经元信号(图 7-1C)。 
小胶质细胞是大脑的常驻免疫细胞和吞噬细胞,但在健康大脑中也具有稳态功能。


\section{神经元和胶质细胞具有许多结构和分子特征}
神经元和胶质细胞从胚胎神经系统的共同神经上皮祖细胞发育而来,并具有许多结构特征(图 7-2)。 
这些细胞的边界由细胞膜或质膜界定,其具有所有生物膜的不对称双层结构,并提供对大多数水溶性物质不可渗透的疏水屏障。 
细胞质有两个主要成分:细胞质和膜状细胞器。


胞质溶胶是细胞质的水相。 
在此阶段,实际上只有少数蛋白质游离在溶液中。 
除了一些催化代谢反应的酶外,大多数蛋白质都被组织成功能复合物。 
最近一个叫做蛋白质组学的分支学科已经确定这些复合物可以由许多不同的蛋白质组成,其中没有一个与另一个共价连接。 
例如,N-甲基-d-天冬氨酸 (NMDA) 型谷氨酸受体(一种介导中枢神经系统兴奋性突触传递的膜相关蛋白)的细胞质尾部锚定在由 100 多种支架蛋白组成的大型复合体中, 蛋白质修饰酶。 (许多参与第二信使信号转导的胞质蛋白,将在后面的章节中讨论,它们嵌入质膜正下方的细胞骨架基质中。)核糖体是翻译信使 RNA (mRNA) 分子的细胞器,由几个蛋白质亚基组成 . 蛋白酶体是一种大型多酶细胞器,可降解泛素化蛋白质(本章稍后描述的过程),它也存在于神经元和胶质细胞的胞质溶胶中。


膜细胞器是细胞质的第二个主要成分,包括线粒体和过氧化物酶体,以及由小管、囊泡和称为液泡器的池组成的复杂系统。 
线粒体和过氧化物酶体处理分子氧。 
线粒体产生三磷酸腺苷 (ATP),这是细胞能量转移或消耗的主要分子,而过氧化物酶体可防止强氧化剂过氧化氢的积累。 
线粒体源自在进化早期侵入真核细胞的共生古细菌,在功能上与液泡器不连续。 
线粒体还在 Ca2+ 稳态和脂质生物合成中发挥其他重要作用。


液泡器包括光滑内质网、粗面内质网、高尔基复合体、分泌小泡、核内体、溶酶体,以及连接这些不同隔室的多种运输小泡(图 7-3)。 它们的管腔在拓扑上对应于细胞的外部; 因此,它们脂质双层的内层小叶对应于质膜的外层小叶。

该系统的主要子隔室在解剖学上是不连续的,但在功能上是相连的,因为膜状和腔内物质通过运输囊泡从一个隔室移动到另一个隔室。 例如,在粗面内质网(布满核糖体的网状部分)和光滑内质网中合成的蛋白质和磷脂被运送到高尔基复合体,然后运送到分泌囊泡,当囊泡膜与 质膜(称为胞吐作用的过程)。 这种分泌途径将膜成分添加到质膜,并将这些分泌囊泡的内容物释放到细胞外空间。

相反,细胞膜的成分通过内吞囊泡进入细胞(内吞作用)。 这些被纳入早期内体,分选集中在细胞外围的隔室。 内吞膜通常含有特定的蛋白质,如跨膜受体,可以通过成熟为循环核内体而直接回到质膜,也可以成熟为晚期核内体,后者通过与溶酶体融合而被靶向降解。 (胞吐作用和胞吞作用将在本章后面详细讨论。)平滑内质网还充当整个神经元细胞质中受调节的内部 Ca2+ 储存(参见第 14 章中关于 Ca2+ 释放的讨论)。

粗面内质网的一个特殊部分形成了核包膜,这是一个球形扁平池,围绕着染色体 DNA 及其相关蛋白(组蛋白、转录因子、聚合酶和异构酶)并定义了细胞核(图 7-3)。 因为核包膜与内质网的其他部分和液泡器的其他膜是连续的,所以推测它已经进化为质膜的内陷以包裹真核染色体。 核膜被核孔打断,核膜的内外膜融合导致亲水通道的形成,蛋白质和 RNA 通过该通道在细胞质本身和核细胞质之间交换。

尽管核质和细胞质是细胞质的连续结构域,但只有分子量小于 5,000 的分子才能通过扩散自由地穿过核孔。 较大的分子需要帮助。 一些蛋白质具有特殊的核定位信号,这些区域由一系列基本氨基酸(精氨酸和赖氨酸)组成,可被称为核输入受体 (importins) 的可溶性蛋白质识别。 在核孔中,这种复合物被另一组称为核孔蛋白的蛋白质引导进入细胞核。

神经细胞体的细胞质延伸到树突状树中,没有功能分化。 通常,细胞体细胞质中的所有细胞器也存在于树突中,尽管粗面内质网、高尔基复合体和溶酶体的密度随着与细胞体的距离而迅速减小。 在树突中,光滑的内质网在称为棘的细突底部突出(图 7-4 和 7-5),这是兴奋性突触的接受部分。 树突棘中多聚核糖体的浓度介导局部蛋白质合成(见下文)。

与细胞体和树突的连续性相反,轴突出现的轴突小丘处的细胞体之间存在明显的功能边界。 构成神经元中蛋白质主要生物合成机制的细胞器——核糖体、粗面内质网和高尔基复合体——通常被排除在轴突之外(图 7-4),溶酶体和某些蛋白质也是如此。 然而,轴突富含光滑内质网、单个突触小泡及其前体膜。


\section{细胞骨架决定细胞形状}
细胞骨架决定细胞的形状,并负责细胞质内细胞器的不对称分布。 它包括三种丝状结构:微管、神经丝和微丝。 这些细丝和相关蛋白约占细胞总蛋白的四分之一。

微管形成长支架,从神经元的一端延伸到另一端,在发育和维持细胞形状方面发挥关键作用。 单个微管可以长达 0.1 毫米。 微管由原丝组成,每个原丝由多对沿微管纵向排列的 a- 和 b- 微管蛋白亚基组成(图 7-6A)。 微管蛋白亚基沿原丝与相邻的亚基结合,并在相邻原丝之间横向结合。 微管用正端(或生长端)和负端(微管可以解聚的地方)极化。 有趣的是,轴突和树突之间的微管方向不同。 在轴突中,微管显示单一方向,正端远离细胞体。 在近端树突中,微管可以双向定向,正端朝向或远离细胞体。

微管通过在其正端添加三磷酸鸟苷 (GTP) 结合的微管蛋白二聚体而生长。 聚合后不久,GTP 被水解为二磷酸鸟苷 (GDP)。 当微管停止生长时,其正端被 GDP 结合的微管蛋白单体所覆盖。 GDP 结合的微管蛋白对聚合物的低亲和力会导致灾难性的解聚,这并不是因为微管通过与其他蛋白质的相互作用而稳定。

事实上,虽然微管在分裂细胞中经历聚合和解聚的快速循环,这种现象被称为动态不稳定,但在成熟的树突和轴突中,它们更稳定。 这种稳定性被认为是由促进微管蛋白聚合物定向聚合和组装的微管相关蛋白 (MAP) 引起的。 轴突中的 MAP 不同于树突中的 MAP。 例如,MAP2 存在于树突中,但不存在于轴突中,轴突中存在 tau 蛋白(见方框 7-1)和 MAP1b。 此外,微管稳定性也受到许多不同类型的可逆微管蛋白翻译后修饰的严格调节,例如乙酰化、去酪氨酸化和聚谷氨酰化。 在阿尔茨海默病和其他一些退行性疾病中,tau 蛋白被修饰并异常聚合,形成一种称为神经原纤维缠结的特征性损伤(方框 7-1)。

微管蛋白由多基因家族编码。 至少有六个基因编码 α 和 β 亚基。 由于不同基因的表达或转录后修饰,大脑中存在 20 多种微管蛋白亚型。

直径为 10 nm 的神经丝是细胞骨架的骨骼(图 7-6B)。 神经丝与其他细胞类型的中间丝有关,包括上皮细胞(头发和指甲)的细胞角蛋白、星形胶质细胞中的神经胶质原纤维酸性蛋白和肌肉中的结蛋白。 与微管不同,神经丝是稳定的并且几乎完全聚合在细胞中。

微丝的直径为 3 至 7 nm,是构成细胞骨架的三种主要纤维类型中最细的(图 7-6C)。 与肌肉的细丝一样,微丝由两股聚合的球状肌动蛋白单体组成,每条单体都带有 ATP 或二磷酸腺苷 (ADP),缠绕成双链螺旋。 肌动蛋白是所有细胞的主要成分,可能是自然界中最丰富的动物蛋白。 有几种密切相关的分子形式:骨骼肌的 α 肌动蛋白和至少两种其他分子形式,β 和 γ。 每个都由不同的基因编码。 高等脊椎动物的神经肌动蛋白是 β 和 γ 种类的混合物,它与肌肉肌动蛋白的区别在于几个氨基酸残基。 大多数肌动蛋白分子是高度保守的,不仅在一个物种的不同细胞类型中,而且在人类和原生动物等远亲生物中也是如此。

与微管和神经丝不同,肌动蛋白丝很短。 它们集中在质膜下方的皮质细胞质中的细胞外围,在那里它们与许多肌动蛋白结合蛋白(例如,spectrinfodrin、ankyrin、talin 和 actinin)形成致密网络。 该基质在细胞外围的动态功能中起着关键作用,例如发育过程中生长锥(轴突的生长尖端)的运动、细胞表面特化微区的产生以及突触前和突触后形态的形成 专长。

与微管一样,微丝经历聚合和解聚循环。 在任何时候,细胞中总肌动蛋白的大约一半可以作为未聚合的单体存在。 肌动蛋白的状态由结合蛋白控制,结合蛋白通过覆盖快速生长的细丝末端或切断它来促进组装并限制聚合物长度。 其他结合蛋白交联或束缚肌动蛋白丝。 微管和微丝的动态状态允许成熟的神经元收缩旧的轴突和树突并延伸新的。 这种结构可塑性被认为是突触连接和功效变化的主要因素,因此也是长期记忆和学习的细胞机制。


\section{蛋白质颗粒和细胞器沿轴突和树突主动运输}
在神经元中,大多数蛋白质是在细胞体中由 mRNA 制成的。 重要的例子是神经递质生物合成酶、突触小泡膜成分和神经分泌肽。 由于轴突和末端通常距离细胞体很远,因此维持这些偏远区域的功能是一项挑战。 被动扩散速度太慢,无法在这么远的距离内输送囊泡、颗粒,甚至单个大分子。

轴突末端是神经递质的分泌部位,离细胞体特别远。 在支配人类腿部肌肉的运动神经元中,神经末梢与细胞体的距离可以超过细胞体直径的 10,000 倍。 因此,细胞体内形成的膜和分泌产物必须主动运输到轴突末端(图 7-9)。

1948 年,Paul Weiss 在绑扎坐骨神经时首次展示了轴突运输,并观察到神经中的轴浆在结扎线的近端随时间积累。 他得出结论,在他称之为轴浆流的过程中,轴浆以缓慢、恒定的速度从细胞体向末端移动。 今天我们知道,Weiss 观察到的流动由两种不同的机制组成,一种快,另一种慢。

膜状细胞器通过快速轴突运输向轴突末端(顺行方向)移动并返回细胞体(逆行方向),这种运输形式在温血动物中每天高达 400 毫米。 相比之下,细胞溶质和细胞骨架蛋白仅通过更慢的运输形式(轴突运输)沿顺行方向移动。 神经元中的这些传输机制是对促进所有分泌细胞中细胞器细胞内运动的过程的适应。 因为所有这些机制都沿着轴突运作,神经解剖学家已经使用它们来追踪单个轴突的过程以及神经元之间的相互联系(方框 7-2)。

\subsection{快速轴突运输携带膜细胞器}

大型膜状细胞器通过快速运输进出轴突末端(图 7-11)。 这些细胞器包括突触小泡前体、大的致密核心小泡、线粒体、光滑内质网的元素和携带 RNA 的蛋白质颗粒。 直接显微分析表明,快速运输沿着与轴突主轴对齐的微管线性轨道以停止和启动(跳跃式)的方式发生。 运动的跳跃性质是由于细胞器从轨道上周期性解离或与其他粒子的碰撞造成的。

背根神经节细胞的早期实验表明,顺行快速转运严重依赖于 ATP,不受蛋白质合成抑制剂的影响(一旦注射的标记氨基酸被掺入),并且不依赖于细胞体,因为它发生在轴突中 从他们的细胞体中分离出来。 事实上,主动运输可以发生在重组的无细胞轴浆中。

微管提供了一个基本固定的轨道,分子马达可以在该轨道上移动特定的细胞器。 微管参与快速转运的想法源于以下发现:某些破坏微管和阻断依赖于微管的有丝分裂的生物碱也会干扰快速转运。

分子马达首先在电子显微照片中显示为微管和运动粒子之间的交叉桥(图 7-8)。 更先进的荧光延时显微镜技术能够可视化特定货物(如线粒体和突触小泡)的轴突运输动力学。 用于顺行运输的运动分子是称为驱动蛋白和多种驱动蛋白相关蛋白的加端定向运动。 驱动蛋白代表一大类腺苷三磷酸酶 (ATPase),每一种都运输不同的货物。 驱动蛋白是由两条重链和两条轻链组成的异四聚体。 每条重链具有三个结构域:(1) 一个球状头部(ATPase 结构域),当连接到微管时充当马达,(2) 一个卷曲螺旋茎,负责与另一条重链二聚化,以及 (3) 与轻链相互作用的扇形羧基末端。 复合物的这一端间接连接到细胞器,细胞器通过称为货物适配器的特定蛋白质家族移动。

快速逆行运输主要移动由神经末梢、线粒体和内质网元件的内吞活动产生的核内体。 许多这些成分通过与溶酶体融合而被降解。 快速逆行运输还传递调节神经元细胞核中基因表达的信号。 例如,神经末梢处激活的生长因子受体被吸收到囊泡中,并沿着轴突运回细胞核。 转录因子的转运通知细胞核中的基因转录装置周围的条件。 这些分子的逆行运输在神经再生和轴突再生过程中尤为重要(第 47 章)。 某些毒素(破伤风毒素)以及病原体(单纯疱疹病毒、狂犬病病毒和脊髓灰质炎病毒)也沿着轴突向细胞体输送。

逆行快速运输的速度大约是顺行快速运输的二分之一到三分之二。 与顺行运输一样,颗粒在逆行流动期间沿着微管移动。 逆行轴突运输的马达分子是负端定向马达,称为动力蛋白,类似于在非神经元细胞的纤毛和鞭毛中发现的马达。 它们由一个多聚体 ATPase 蛋白复合物组成,两个球状头部位于连接到基础结构的两个茎上。 球形头部附着在微管上并充当马达,向聚合物的负端移动。 与驱动蛋白一样,复合体的另一端通过专门的货物适配器连接到运输的细胞器上。

微管还介导由 RNA 结合蛋白形成的颗粒携带的 mRNA 和核糖体 RNA 的顺行和逆行运输。 这些蛋白质已在脊椎动物和无脊椎动物神经系统中得到表征,包括细胞质聚腺苷酸化元件结合蛋白 (CPEB)、脆性 X 蛋白、Hu 蛋白、NOVA 和 Staufen。 这些蛋白质的活性至关重要。 例如,CPEB 在从细胞体到神经末梢的运输过程中使选定的 mRNA 保持休眠状态; 一旦到达那里(受到刺激),结合蛋白可以通过介导聚腺苷酸化和信使激活来促进 RNA 的局部翻译。 CPEB 和 Staufen 都是在果蝇中发现的,它们在未受精卵中保持母体 mRNA 休眠,并在受精后将 mRNA 分布和定位到分裂胚胎的各个区域。 脆性 X (FMR1) 基因的功能缺失突变会导致严重的精神发育迟滞。

蛋白质、核糖体和 mRNA 集中在一些树突棘的底部(图 7-12)。 只有一组选定的 mRNA 从胞体转运到树突中。 这些包括编码肌动蛋白和细胞骨架相关蛋白、MAP2 和 Ca2+/钙调蛋白依赖性蛋白激酶的 α 亚基的 mRNA。 它们在树突中翻译以响应突触前神经元中的活动。 这种局部蛋白质合成被认为对于维持作为长期记忆和学习基础的突触分子变化很重要。 同样,髓鞘碱性蛋白的 mRNA 被运送到少突胶质细胞的远端,在那里它随着髓鞘的生长而被翻译,这将在本章后面讨论。

\subsection{缓慢的轴突运输携带细胞溶质蛋白和细胞骨架的元素}

细胞溶质蛋白和细胞骨架蛋白通过缓慢的轴突运输从细胞体中移出。 慢转运只发生在顺行方向,由至少两个以不同速率携带不同蛋白质的动力学成分组成。

较慢的成分每天移动 0.2 至 2.5 毫米,并携带构成细胞骨架纤维状元素的蛋白质:神经丝的亚基和微管的 α- 和 β- 微管蛋白亚基。 这些纤维蛋白约占较慢成分中移动的总蛋白的 75\%。 微管通过涉及微管滑动的机制以聚合形式运输,其中相对较短的预组装微管沿着现有微管移动。 神经丝单体或短聚合物与微管一起被动移动,因为它们通过蛋白质桥交联。

缓慢轴突运输的另一个组成部分的速度大约是较慢部分的两倍。 它携带网格蛋白、肌动蛋白和肌动蛋白结合蛋白以及多种酶和其他蛋白质。


\section{与其他分泌细胞一样,蛋白质也是在神经元中制造的}
\subsection{分泌蛋白和膜蛋白在内质网中合成和修饰}
分泌蛋白和膜蛋白的 mRNA 通过粗面内质网膜翻译,其多肽产物在内质网腔内广泛加工。 大多数注定要成为蛋白质的多肽在合成过程中会跨过粗面内质网膜,这一过程称为共翻译转移。

转移是可能的,因为核糖体,蛋白质合成的位点,附着在网状细胞的胞质表面(图 7-13)。 多肽链完全转移到网状细胞腔中会产生一种分泌蛋白(回想一下,网状细胞的内部与细胞的外部有关)。 重要的例子是神经活性肽。 如果转移不完全,则会产生完整的膜蛋白。 由于多肽链在合成过程中可以多次穿过膜,因此根据蛋白质的一级氨基酸序列,可能有几种跨膜构型。 重要的例子是神经递质受体和离子通道(第 8 章)。

一些运输到内质网中的蛋白质保留在那里。 其他的则移动到液泡器的其他隔室或质膜,或分泌到细胞外空间。 在内质网中加工的蛋白质被广泛修饰。 一个重要的修饰是由成对的游离巯基侧链氧化引起的分子内二硫键 (Cys-S-S-Cys) 的形成,这一过程不能发生在胞质溶胶的还原环境中。 二硫键对于这些蛋白质的三级结构至关重要。

蛋白质可以在合成过程中(共翻译修饰)或之后(翻译后修饰)被胞质酶修饰。 一个例子是 N-酰化,将酰基转移到生长中的多肽链的 N-末端。 14-碳脂肪酸肉豆蔻酰基的酰化允许蛋白质通过脂质链锚定在膜中。

其他脂肪酸可以与半胱氨酸的巯基结合,产生硫酰化作用:

异戊二烯化是另一种翻译后修饰,对于将蛋白质锚定到细胞膜的胞质侧很重要。 它在蛋白质合成完成后不久发生,涉及一系列酶促步骤,导致巯基的两个长链疏水聚异戊二烯基(法尼基,具有 15 个碳原子,或香叶基-香叶基,具有 20 个)之一发生硫代酰化 蛋白质 C 末端的半胱氨酸。

一些翻译后修饰很容易可逆,因此可用于瞬时调节蛋白质的功能。 这些修饰中最常见的是蛋白激酶对丝氨酸、苏氨酸或酪氨酸残基中羟基的磷酸化。 去磷酸化由蛋白磷酸酶催化。 (这些反应在第 14 章中讨论。)与所有翻译后修饰一样,要磷酸化的位点由要修饰的残基周围的特定氨基酸序列决定。 磷酸化可以以可逆的方式改变生理过程。 例如,蛋白质磷酸化-去磷酸化反应调节离子通道的动力学、转录因子的活性和细胞骨架的组装。

另一个重要的翻译后修饰是将泛素(一种具有 76 个氨基酸的高度保守的蛋白质)添加到蛋白质分子中特定赖氨酸残基的ε-氨基上。 调节蛋白质降解的泛素化由三种酶介导。 E1是一种利用ATP能量的活化酶。 激活的泛素接下来被转移到结合酶 E2,然后将激活的部分转移到连接酶 E3。 E3 单独或与 E2 一起将泛素基转移到蛋白质的赖氨酸残基上。 特异性的产生是因为给定的蛋白质分子只能被特定的 E3 或 E3 和 E2 的组合泛素化。 一些 E3 还需要特殊的辅助因子——泛素化仅在 E3 和辅助因子蛋白存在的情况下发生。

单泛素化标记一种蛋白质在内体-溶酶体系统中降解。 这在表面受体的内吞作用和再循环中尤为重要。 泛素基单体依次连接到先前添加的泛素部分中赖氨酸残基的ε-氨基。 在多泛素链上添加超过 5 个泛素,标记蛋白质被蛋白酶体降解,蛋白酶体是一种大型复合物,包含可将蛋白质切割成短肽的多功能蛋白酶亚基。

ATP-泛素-蛋白酶体途径是一种选择性和调节蛋白水解的机制,在神经元所有区域(树突、细胞体、轴突和末端)的胞质溶胶中起作用。 直到最近,这一过程还被认为主要针对折叠不良、变性或老化和受损的蛋白质。 我们现在知道泛素介导的蛋白水解可以受神经元活动的调节,并在许多神经元过程中发挥特定作用,包括突触发生和长期记忆存储。

另一个重要的蛋白质修饰是糖基化,它发生在天冬酰胺残基的氨基上(N-连接糖基化)并导致复杂多糖链的整体添加。 然后,通过伴侣分子控制的一系列反应,包括热休克蛋白、钙联接蛋白和钙网蛋白,这些链在内质网内被修剪。 由于寡糖部分具有很强的化学特异性,这些修饰对细胞功能具有重要意义。 例如,发育过程中发生的细胞间相互作用依赖于两个相互作用细胞表面糖蛋白之间的分子识别。 此外,由于给定的蛋白质可能具有略微不同的寡糖链,糖基化可以使蛋白质的功能多样化。 它可以增加蛋白质的亲水性(对分泌蛋白有用),微调其结合大分子伙伴的能力,并延缓其降解。

一种有趣的 mRNA 翻译后修饰是 RNA 干扰 (RNAi),即双链 RNA 的靶向破坏。 这种机制被认为是为了保护细胞免受病毒和其他无赖核酸片段的侵害而产生的,它会关闭任何目标蛋白质的合成。 双链 RNA 被一种酶复合物吸收,该酶复合物将分子切割成寡聚体。 RNA 序列被复合物保留。 结果,任何同源杂交 RNA 链,无论是双链还是单链,都将被破坏。 这个过程是再生的:复合物保留一个杂交片段,然后继续破坏另一个 RNA 分子,直到细胞中没有分子为止。 尽管 RNAi 在正常细胞中的生理作用尚不清楚,但将 RNAi 转染或注射到细胞中具有重要的研究和临床意义(第 2 章)。s


\subsection{分泌蛋白在高尔基复合体中被修饰}

来自内质网的蛋白质在运输囊泡中被携带到高尔基复合体,在那里它们被修饰,然后移动到突触末端和质膜的其他部分。 高尔基复合体表现为一组以长带状排列的膜质袋。

从简单的单细胞原核生物(酵母)到多细胞生物体的神经元和胶质细胞,囊泡在分泌和内吞途径站之间运输的机制一直非常保守。 运输囊泡从膜发展而来,首先是在膜的胞质表面的选定斑块处组装形成外壳的蛋白质或外壳蛋白。 外套有两个功能。 它形成刚性笼状结构,使膜外翻成芽状,并选择要掺入囊泡的蛋白质货物。

有几种类型的外套。 网格蛋白涂层有助于在内吞过程中外翻高尔基复合体膜和质膜。 另外两个外壳,COPI 和 COPII,覆盖在内质网和高尔基复合体之间穿梭的运输囊泡。 一旦游离囊泡形成,包衣通常会迅速溶解。 囊泡与靶膜的融合是由一系列分子相互作用介导的,其中最重要的是两个相互作用膜的胞质表面上小蛋白的相互识别:囊泡可溶性 N-乙基马来酰亚胺敏感因子附着蛋白受体 (v-SNAREs) 和 t-SNAREs (targetmembrane SNAREs)。 第 15 章讨论了 SNARE 蛋白通过突触小泡与质膜融合释放神经递质的作用。

来自内质网的囊泡到达高尔基复合体的顺侧(面向细胞核的一侧)并与其膜融合以将其内容物输送到高尔基复合体中。 这些蛋白质从一个高尔基体隔室(水池)移动到下一个,从顺式到反式,经历一系列酶促反应。 每个高尔基池或一组池专门用于特定类型的反应。 几种类型的蛋白质修饰,其中一些开始于内质网,发生在高尔基复合体本身或与其反侧相邻的运输站内,反式高尔基网络(复合体的一面通常背对细胞核朝向 轴突丘)。 这些修饰包括添加 N-连接寡糖、O-连接(在丝氨酸和苏氨酸的羟基上)糖基化、磷酸化和硫酸化。

穿过高尔基体复合体的可溶性和膜结合蛋白均从跨高尔基体网络出现在各种具有不同分子组成和目的地的囊泡中。 从跨高尔基体网络转运的蛋白质包括分泌产物以及质膜、核内体和其他膜细胞器的新合成成分(见图 7-2)。 一类囊泡携带新合成的质膜蛋白和持续分泌的蛋白质(组成型分泌)。 这些囊泡以不受管制的方式与质膜融合。 这些囊泡的一种重要类型将溶酶体酶递送至晚期核内体。

还有其他类型的囊泡携带由细胞外刺激(调节分泌)释放的分泌蛋白。 一种类型以高浓度储存分泌产物,主要是神经活性肽。 由于它们在电子显微镜下的电子致密(亲渗)外观而被称为大的致密核心囊泡,这些囊泡在功能和生物发生方面与内分泌细胞的含肽颗粒相似。 大的致密核心囊泡主要针对轴突,但在神经元的所有区域都可以看到。 它们积聚在质膜正下方的细胞质中,并高度集中在轴突末端,在那里它们的内容物通过 Ca2+ 调节的胞吐作用释放。

最近的研究表明,小的突触小泡——负责在轴突末端快速释放神经递质的电子透明小泡——作为单独的货物被积极地运送到突触末端。 据认为,小突触小泡的蛋白质成分源自跨高尔基体网络的大前体小泡。 这些突触小泡已经包含了大部分能够在突触前活动区融合的蛋白质。 存储在这些突触小泡中的神经递质分子通过胞吐作用释放,胞吐作用受 Ca2+ 通过靠近释放位点的通道流入的调节。 然后,囊泡会经历第 15 章所述的循环/胞吐作用循环。重要的是,这些囊泡通过称为囊泡转运蛋白的专门转运蛋白重新填充,这些转运蛋白对每种神经递质(例如,谷氨酸、γ-氨基丁酸 [GABA]、乙酰胆碱)具有特异性。


\section{表面膜和细胞外物质在细胞内循环}
从质膜到内部细胞器的内吞流量不断平衡流向细胞表面的囊泡流量。 这种流量对于维持膜面积处于稳定状态是必不可少的。 它可以改变细胞表面许多重要调节分子的活性(例如,通过去除受体和粘附分子)。 它还将营养物质和分子(例如可消耗的受体配体和受损的膜蛋白)去除到细胞的降解区室中。 最后,它用于在神经末梢回收突触小泡(第 15 章)。

很大一部分内吞交通是在网格蛋白包被的囊泡中进行的。 网格蛋白涂层通过跨膜受体选择性地与将被吸收到细胞中的细胞外分子相互作用。 因此,网格蛋白介导的摄取通常被称为受体介导的内吞作用。 囊泡最终脱落其网格蛋白外壳并与早期核内体融合,在核内体中,将被回收到细胞表面的蛋白质与那些用于其他细胞内细胞器的蛋白质分开。 质膜的斑块也可以通过更大的、未包被的液泡循环,这些液泡也与早期内体融合(大量内吞作用)。



\section{胶质细胞在神经功能中发挥多种作用}
Ramón y Cajal 认识到神经胶质细胞与大脑中的神经元和突触的密切联系(图 7-14)。 尽管当时它们的功能还是个谜,但他预测神经胶质细胞的功能肯定不仅仅是将神经元聚集在一起。 事实上,现在很清楚神经胶质细胞在大脑发育、功能和疾病中起着关键作用。

\subsection{胶质细胞形成轴突的绝缘鞘}
少突胶质细胞和雪旺细胞的主要功能是提供绝缘材料,使电信号能够沿轴突快速传导。 这些细胞产生薄薄的髓磷脂片,多次同心地包裹在轴突周围。 由少突胶质细胞产生的 CNS 髓磷脂与由雪旺细胞产生的周围神经系统髓磷脂相似,但不完全相同。

两种类型的神经胶质细胞都只为部分轴突产生髓磷脂。 这是因为轴突并没有连续包裹在髓鞘中,髓鞘是一种促进动作电位传播的特征(第 9 章)。 一个雪旺细胞为一个轴突的一个节段产生一个髓鞘,而一个少突胶质细胞为多达 30 个轴突的节段产生髓鞘(图 7-1 和 7-15)。

轴突上髓鞘的层数与轴突的直径成正比——较大的轴突具有较厚的鞘。 直径非常小的轴突没有髓鞘; 无髓鞘轴突传导动作电位的速度比有髓鞘轴突慢得多,因为它们的直径较小且缺乏髓鞘绝缘(第 9 章)。

鞘的规则层状结构和生化成分是神经胶质质膜形成髓磷脂的结果。 在周围神经系统的发育过程中,在髓鞘形成之前,轴突位于雪旺细胞形成的槽内。 雪旺细胞以规则的间隔沿着轴突排列,成为轴突的有髓鞘部分。 每个雪旺细胞的外膜围绕着轴突形成一个双膜结构,称为中轴突,它在同心层中围绕轴突伸长和螺旋(图 7-15C)。 当轴突被包裹时,雪旺细胞的细胞质被挤出,形成紧凑的层状结构。

髓鞘的规则间隔部分被无髓鞘间隙隔开,称为朗飞结,其中轴突的质膜暴露于细胞外空间约 1 μm(图 7-16)。 这种安排极大地提高了神经冲动传导的速度(人类高达 100 m/s),因为信号从一个节点跳到下一个节点,这种机制称为跳跃式传导(第 9 章)。 节点很容易兴奋,因为产生动作电位的 Na+ 通道的密度在节点处的轴突膜中比在膜的髓鞘区域中高大约 50 倍。 节点周围的细胞粘附分子使髓鞘边界保持稳定。

在人类股神经中,初级感觉轴突长约0.5米,节间距离为1~1.5毫米; 因此,大约 300 到 500 个朗飞节点沿着大腿肌肉和背根神经节细胞体之间的初级传入纤维出现。 因为每个节间节段由单个雪旺细胞形成,所以多达 500 个雪旺细胞参与每个外周感觉轴突的髓鞘形成。

髓磷脂具有散布在蛋白质层之间的双分子脂质层。 其成分与质膜相似,由 70\% 的脂质和 30\% 的蛋白质以及高浓度的胆固醇和磷脂组成。 在中枢神经系统中,髓磷脂有两种主要蛋白质:髓磷脂碱性蛋白,一种位于致密髓磷脂细胞质表面的小的带正电荷的蛋白质,以及蛋白脂质蛋白,一种疏水性整合膜蛋白。 据推测,两者都为护套提供了结构稳定性。

两者也被认为是重要的自身抗原,免疫系统可以针对这些自身抗原产生反应,从而产生脱髓鞘疾病多发性硬化症。 在周围神经系统中,髓磷脂含有一种主要的蛋白质 P0,以及疏水性蛋白质 PMP22。 对这些蛋白质的自身免疫反应会产生脱髓鞘周围神经病,即吉兰-巴利综合征。 髓鞘蛋白基因的突变也会导致外周和中央轴突发生多种脱髓鞘疾病(方框 7-3)。 脱髓鞘作用减慢甚至停止受影响轴突中动作电位的传导,因为它允许电流从轴突膜泄漏。 因此,脱髓鞘疾病对中枢和外周神经系统的神经回路具有破坏性影响(第 57 章)。

\subsection{星形胶质细胞支持突触信号}
星形胶质细胞存在于大脑的所有区域; 实际上,它们几乎占脑细胞数量的一半。 它们在滋养神经元和调节细胞外空间中离子和神经递质的浓度方面发挥着重要作用。 但是星形胶质细胞和神经元也可以相互交流以调节突触信号,其方式仍知之甚少。 星形胶质细胞通常分为两大类,以形态、位置和功能来区分。 原生质星形胶质细胞存在于灰质中,它们的过程与突触和血管密切相关。 白质中的纤维状(或纤维状)星形胶质细胞接触轴突和 Ranvier 结。 此外,专门的星形胶质细胞包括小脑中的 Bergmann 胶质细胞和视网膜中的 Müller 胶质细胞。

星形胶质细胞具有大量细突,包裹着大脑的所有血管并包裹突触或突触群。 通过与突触的密切物理联系(通常小于 1 μm),星形胶质细胞可以调节细胞外离子、神经递质和其他分子的浓度(图 7-19)。 事实上,星形胶质细胞表达许多与神经元相同的电压门控离子通道和神经递质受体,因此能够很好地接收和传输可能影响神经元兴奋性和突触功能的信号。

星形胶质细胞如何调节轴突传导和突触活动? 第一个公认的生理作用是 K+ 缓冲作用。 当神经元激发动作电位时,它们会将 K+ 离子释放到细胞外空间。 由于星形胶质细胞的膜中具有高浓度的 K+ 通道,因此它们可以充当空间缓冲器:它们在神经元活动部位(主要是突触)吸收 K+,并在与血管的远距离接触时释放。 星形胶质细胞还可以在其细胞质过程中局部积累 K+ 以及 Cl- 离子和水。 不幸的是,星形胶质细胞中离子和水的积累会导致头部受伤后严重的脑肿胀。

星形胶质细胞还调节大脑中的神经递质浓度。 例如,位于星形胶质细胞质膜中的高亲和力转运蛋白可快速清除突触间隙中的神经递质谷氨酸(图 7-19C)。 一旦进入神经胶质细胞,谷氨酸就会被谷氨酰胺合成酶转化为谷氨酰胺。 然后谷氨酰胺被转移到神经元,在那里它作为谷氨酸的直接前体(第 16 章)。 干扰这些摄取机制会导致细胞外谷氨酸浓度升高,从而导致神经元死亡,这一过程称为兴奋性毒性。 星形胶质细胞还降解多巴胺、去甲肾上腺素、肾上腺素和血清素。

星形胶质细胞会感知神经元何时活跃,因为它们会被神经元释放的 K+ 去极化,并且具有与神经元相似的神经递质受体。 例如,小脑中的 Bergmann 胶质细胞表达谷氨酸受体。 因此,小脑突触释放的谷氨酸不仅影响突触后神经元,还影响突触附近的星形胶质细胞。 这些配体与神经胶质受体的结合增加了细胞内游离 Ca2+ 浓度,这具有几个重要的后果。 一个星形胶质细胞的过程通过称为间隙连接的细胞间水通道(第 11 章)连接到邻近星形胶质细胞的过程,允许许多细胞之间的离子和小分子转移。 一个星形胶质细胞内游离 Ca2+ 的增加会增加相邻星形胶质细胞中 Ca2+ 的浓度。 这种 Ca2+ 通过星形胶质细胞网络的扩散发生在数百微米范围内。 这种 Ca2+ 波很可能通过触发营养物质的释放和调节血流来调节附近的神经元活动。 星形胶质细胞中 Ca2+ 的增加会导致信号的分泌,从而增强突触功能甚至行为。 因此,星形胶质细胞-神经元信号传导有助于正常的神经回路功能。

星形胶质细胞对于突触的发育也很重要。 它们在产后大脑突触处的出现与突触发生和突触成熟的时期一致。 星形胶质细胞为突触形成准备神经元表面并稳定新形成的突触。 例如,星形胶质细胞分泌多种突触因子,包括血小板反应蛋白、hevin 和 glycipans,它们促进新突触的形成。 星形胶质细胞还可以通过吞噬作用帮助重塑和消除发育过程中多余的突触(第 48 章)。 在成人 CNS 中,星形胶质细胞继续吞噬突触,并且由于这种吞噬作用依赖于神经元活动,因此突触的这种重塑可能有助于学习和记忆。 在病理状态下,例如轴突损伤产生的染色质分解,星形胶质细胞和突触前末端会暂时从受损的突触后细胞体中缩回。 星形胶质细胞释放神经营养因子和胶质营养因子,促进神经元和少突胶质细胞的发育和存活。 它们还保护其他细胞免受氧化应激的影响。 例如,星形胶质细胞中的谷胱甘肽过氧化物酶可以解毒缺氧、炎症和神经元变性期间释放的有毒氧自由基。

最后,星形胶质细胞包裹整个大脑的小动脉和毛细血管,在星形胶质细胞突起的末端和内皮细胞周围的基底层之间形成接触。 CNS 与全身循环隔绝,因此血液中的大分子不会被动进入大脑和脊髓(血脑屏障)。 屏障主要是内皮细胞和大脑毛细血管之间紧密连接的结果,而身体其他部位的毛细血管则不具备这一特征。 然而,内皮细胞具有许多运输特性,允许一些分子通过它们进入神经系统。 由于星形胶质细胞和血管的密切接触,运输的分子,如葡萄糖,可以被星形胶质细胞的末端吸收。

在脑损伤和疾病之后,星形胶质细胞会经历一种称为反应性星形胶质细胞增多症的戏剧性转变,这涉及基因表达、形态和信号传导的变化。 反应性星形胶质细胞的功能很复杂且知之甚少,因为它们既阻碍又支持中枢神经系统的恢复。 最近的研究发现了至少两种反应性星形胶质细胞的证据; 一种有助于促进修复和恢复,而另一种是有害的,积极促进急性中枢神经系统损伤后神经元的死亡; 然而,可能还有其他亚型。 这些神经毒性反应性星形胶质细胞在阿尔茨海默病和其他神经退行性疾病患者中很突出,因此是新疗法的一个有吸引力的目标。 一个有趣的问题是为什么大脑会产生神经毒性反应性星形胶质细胞。 很可能,移除受伤或患病的神经元可以使突触重组以帮助保持神经回路功能。 此外,去除被病毒感染的神经元可能有助于限制病毒感染的传播。

\subsection{小胶质细胞在健康和疾病中具有多种功能}

小胶质细胞约占中枢神经系统神经胶质细胞的 10\%,并以多种形态存在于健康和受损的大脑中。 尽管 100 多年前 Rio Hortega 已经描述过,但与其他细胞类型相比,小胶质细胞的功能仍知之甚少。 与神经元、星形胶质细胞和少突胶质细胞不同,小胶质细胞不属于神经外胚层谱系。 长期以来被认为源自骨髓,最近的命运图谱研究表明,小胶质细胞实际上源自卵黄囊中的骨髓祖细胞。

小胶质细胞在胚胎发育的早期就在大脑中定殖,并在整个生命过程中驻留在大脑的所有区域(图 7-20)。 在发育过程中,小胶质细胞通过吞噬突触前和突触后结构来帮助塑造发育中的神经回路(图 7-21),并且新出现的证据表明小胶质细胞可能调节大脑发育和大脑稳态的其他方面。 最近的体内成像研究揭示了小胶质细胞和神经元之间的动态相互作用。 在健康的成人大脑皮层中,小胶质细胞过程不断地调查它们周围的细胞外环境并接触神经元和突触,但这种活动的功能意义仍然未知。

在受伤和疾病之后,小胶质细胞的过程运动性急剧增加,形态和基因表达发生变化,并且可以迅速募集到损伤部位,在那里它们可以发挥有益的作用。 例如,它们用于将淋巴细胞、中性粒细胞和单核细胞带入中枢神经系统并扩大淋巴细胞数量,在感染、中风和免疫脱髓鞘疾病中发挥重要的免疫活性。 它们还通过吞噬碎片以及不需要的和垂死的细胞和有毒蛋白质来保护大脑,这些作用对于防止进一步损伤和维持大脑稳态至关重要。 尽管对感染或创伤的免疫反应至关重要,但小胶质细胞还通过释放细胞因子和神经毒性蛋白以及诱导神经毒性反应性星形胶质细胞来促进病理性神经炎症。 它们还导致阿尔茨海默病和神经退行性疾病模型中的突触丢失和功能障碍。


\section{脉络丛和室管膜细胞产生脑脊液}

神经元和胶质细胞的功能受中枢神经系统细胞外环境的严格调控。 间质液 (ISF) 填充实质中神经元和胶质细胞之间的空间。 脑脊液 (CSF) 浸润脑室、脑和脊髓的蛛网膜下腔以及中枢神经系统的主要池。 ISF 和 CSF 将营养物质输送到 CNS 中的细胞,维持离子稳态,并作为代谢废物的清除系统。 CSF 与围绕大脑和脊髓的脑膜层一起提供保护 CNS 组织免受机械损伤的缓冲层。 CNS 的液体环境由血脑屏障的内皮细胞和血脑脊液屏障的脉络丛上皮细胞维持。 这些屏障不仅用于调节大脑和脊髓的细胞外环境,而且还在中枢神经系统和外周之间传递关键信息。

脉络丛和室管膜层的细胞有助于 CSF 的产生、组成和动力学。 脉络丛在神经管闭合后不久表现为上皮内陷,最终将形成侧脑室、第三脑室和第四脑室。 通过胚胎发育,脉络丛成熟,每个形成纤毛立方上皮层,包裹基质和免疫细胞网络以及广泛的毛细血管床。 室管膜是单层纤毛立方细胞,一种排列在脑室的神经胶质细胞。 在侧脑室和第四脑室的几个地方,专门的室管膜细胞形成了围绕脉络丛的上皮层(图 7-22B)。

脉络丛产生大部分沐浴大脑的脑脊液。 室管膜细胞之间的松散连接为脑脊液提供了进入大脑间质空间的通道。 室管膜细胞中的纤毛运动有助于移动脑脊液通过心室系统(图 7-22A),促进分子远距离输送到中枢神经系统中的其他细胞,并将废物从中枢神经系统输送到外周。

脉络丛将液体和溶质从血清输送到中枢神经系统以产生脑脊液。 穿过脉络丛的开窗毛细血管允许水和小分子从血液中自由通过进入脉络丛的基质空间。 然而,脉络丛上皮细胞形成紧密连接,防止这些分子进一步不受管制地移动到大脑中。 相反,组成 CSF 的水、离子、代谢物和蛋白质介质的输入受到脉络丛上皮中转运蛋白和通道的严格调节。 上皮细胞中的主动转运机制是双向的,另外还介导分子从 CSF 返回外周循环的流量。

脉络丛上皮细胞也合成许多蛋白质并将其分泌到脑脊液中。 在健康的胚胎和出生后的大脑中,这些蛋白质调节神经干细胞的发育,并可能调节皮质可塑性等过程。 脉络丛上皮细胞分泌蛋白组也可以被来自外周或大脑内部的炎症信号改变,对感染和衰老过程中的神经元功能产生影响。 其他脉络丛衍生因子在健康和患病大脑中的功能作用——包括 microRNA、长链非编码 RNA 和细胞外囊泡——开始出现,进一步强调了这种结构对大脑发育和体内平衡的重要贡献。

\section{亮点}

1. 神经元的形态非常适合在大脑中接收、传导和传递信息。 树突为接收信号提供了高度分支的细长表面。 轴突长距离快速传导电脉冲到它们的突触末端,突触末端将神经递质释放到目标细胞上。 

2. 尽管所有神经元都符合相同的基本细胞结构,但不同亚型的神经元在其特定的形态特征、功能特性和分子特性方面差异很大。 

3. 不同位置的神经元在其树突树的复杂性、轴突分支的范围以及它们形成和接收的突触末端的数量方面存在差异。 这些形态差异的功能意义显而易见。 例如,运动神经元必须具有比感觉神经元更复杂的树突树,因为即使是简单的反射活动也需要整合许多兴奋性和抑制性输入。 不同类型的神经元使用不同的神经递质、离子通道和神经递质受体。 这些生物化学、形态学和电生理学的差异共同导致了大脑中信息处理的巨大复杂性。 

4. 神经元是我们体内极化程度最高的细胞之一。 它们的树突和轴突区室的相当大和复杂性代表了这些细胞面临的重大细胞生物学挑战,包括长距离运输各种细胞器、蛋白质和 mRNA(某些轴突可达一米)。 大多数神经元蛋白在细胞体内合成,但一些合成发生在树突和轴突中。 新合成的蛋白质在分子伴侣的帮助下折叠,其最终结构通常通过永久或可逆的翻译后修饰进行修饰。 神经元中蛋白质的最终目的地取决于其氨基酸序列中编码的信号。 

5. 蛋白质和 mRNA 的转运具有很强的特异性,并导致选定膜成分的矢量转运。 细胞骨架除了控制轴突和树突形态外,还为将细胞器运输到不同的细胞内位置提供了重要的框架。 

6. 所有这些基本的细胞生物学过程都可以通过神经元活动进行深刻的修改,从而使神经回路适应经验(学习)的细胞结构和功能发生巨大变化。 

7. 神经系统还包含几种类型的神经胶质细胞。 少突胶质细胞和雪旺细胞产生髓鞘绝缘层,使轴突能够快速传导电信号。 星形胶质细胞和非髓鞘化雪旺细胞包裹着神经元的其他部分,尤其是突触。 星形胶质细胞控制细胞外离子和神经递质浓度,并积极参与突触的形成和功能。 小胶质细胞常驻免疫细胞和吞噬细胞与神经元和神经胶质细胞动态相互作用,并在健康和疾病中发挥多种作用。 

8. 脉络丛和室管膜层的细胞有助于 CSF 的产生、组成和动力学。 

9. 基因组学和单细胞 RNA 测序的新进展开始定义细胞类型的巨大多样性,不仅在神经元之间而且在神经胶质细胞之间。 

10. 遗传学、细胞生物学和体内显微术(双光子显微术、光片显微术)的最新进展为神经元在整个生命周期中建立和维持其极性的独特机制提供了新的见解。 

11. 这些新见解为细胞生物学步骤提供了重要线索,例如轴突运输缺陷,这些缺陷会引发亨廷顿病、帕金森病和阿尔茨海默病等神经退行性疾病。

\section{选读}

\section{参考文献}
