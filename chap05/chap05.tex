\chapter{调节行为的神经回路的计算基础}
% PDF所在目录: /data2/whd/win10/learn/neuro/neuro_神经科学原理_28_中枢神经系统的听觉处理.pdf

\section{声音向有听觉的动物传达多种类型的信息}

\section{哺乳动物的上橄榄复合体包包含于检测双耳时间差和双耳强度差的独立回路}

\section{传入听觉通路在下丘汇聚}


\section{下丘脑传输信息给大脑皮层}

\subsection{沿着上行通路刺激选择性逐渐增加}

\subsection{听觉皮层映射众多的声音方位}

\subsection{从下丘而来的第二声音定位通路涉及凝视控制的大脑皮层}


\subsection{大脑皮层中的听觉回路被分离成分开的处理流}
带状区的 嘴侧和腹侧-- 颞叶的嘴侧和腹侧
带状区的 尾侧--颞叶的尾侧和背侧

\subsection{大脑皮层在皮下听区调制感觉加工}


\section{大脑皮层形成复杂的声音表示}

\subsection{听觉皮层使用时间和速率编码来表征时变声音}

\subsection{灵长类有专门的皮层神经元编码音高和泛音}

\subsection{食虫蝙蝠有皮层区域专门负责行为相关的声音特征}

\subsection{听觉皮层涉及处理说话时的声音反馈}