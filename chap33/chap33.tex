\chapter{运动}
% PDF所在目录: /data2/whd/win10/learn/neuro/neuro_神经科学原理_28_中枢神经系统的听觉处理.pdf

\section{运动需要产生精确协调的肌肉激活模式}

\section{踏步的运动模式是在脊髓水平组织的}
\subsection{负责运动的脊髓回路可以根据经验进行修改}
\subsection{脊髓运动网络被组织成节奏和模式生成电路}

\section{来自移动肢体的体感输入调节运动}
\subsection{本体感觉调节步进的时间和幅度}
\subsection{皮肤中的机械感受器允许行走以适应意外障碍}

\section{脊柱上结构负责步进的启动和自适应控制}
\subsection{中脑核启动并维持运动和控制速度}
\subsection{启动脑干神经元运动项目的中脑核}
\subsection{脑干核团在运动过程中调节姿势}

\section{视觉引导运动涉及运动皮层}

\section{运动规划涉及后顶叶皮层}

\section{小脑调节下行信号的时间和强度}

\section{基底神经节改变皮质和脑干回路}

\section{计算神经科学提供了对运动回路的见解}

\section{人类运动的神经元控制与四足动物相似}

\section{要点}
\subsection{荐读}
\subsection{参考文献}