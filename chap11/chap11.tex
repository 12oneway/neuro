\chapter{突触传递概述}
% PDF所在目录: /data2/whd/win10/learn/neuro/neuro_神经科学原理_28_中枢神经系统的听觉处理.pdf

\section{突触主要是电的或化学的}

\section{电突触提供快速信号传输}
\subsection{电突触处的细胞通过间隙连接通道连接}
\subsection{电传输允许互连细胞的快速同步点火}
\subsection{间隙连接在胶质细胞功能和疾病中发挥作用}
\subsection{神经递质的作用取决于突触后受体的特性}
\subsection{突触后受体的激活直接或间接地控制离子通道}

\section{电突触和化学突触可以共存并相互作用}

\section{亮点}

\section{选读}

\section{参考文献}