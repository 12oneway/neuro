\chapter{中枢神经系统的突触整合} \label{chap:chap13}
与神经肌肉接头处的突触传递一样,中枢神经系统神经元之间最快速的信号传递涉及突触后膜中的离子型受体。 因此,许多适用于神经肌肉接头处运动神经元和骨骼肌纤维之间突触连接的原理也适用于中枢神经系统。 然而,由于多种原因,中枢神经元之间的突触传递更为复杂。

首先,尽管大多数肌肉纤维通常仅受一个运动神经元支配,但一个中枢神经细胞(例如新皮质中的锥体神经元)接收来自数千个神经元的连接。 其次,肌肉纤维只接收兴奋性输入,而中枢神经元同时接收兴奋性和抑制性输入。 第三,肌肉纤维上的所有突触作用均由一种神经递质乙酰胆碱 (ACh) 介导,它仅激活一种类型的受体(离子型烟碱 ACh 受体)。 然而,单个中枢神经元可以对许多不同类型的输入作出反应,每种输入都由激活特定类型受体的不同发射器介导。 这些受体包括离子型受体,其中递质的结合直接打开离子通道,以及代谢型受体,其中递质结合通过激活第二信使间接调节通道。 因此,与肌肉纤维不同,中枢神经元必须将不同的输入整合到一个单一的协调动作中。

最后,神经-肌肉突触是效率模型——运动神经元中的每个动作电位都会在肌纤维中产生一个动作电位。 相比之下,突触前神经元与中枢神经元之间的连接效果有限——在许多情况下,至少 50 到 100 个兴奋性神经元必须同时激发才能产生足够大的突触电位,以触发突触后神经元的动作电位。

对中枢神经系统突触传递的第一个见解来自 John Eccles 和他的同事在 1950 年代的实验,该实验研究了突触输入到控制牵张反射的脊髓运动神经元,这是我们在第 3 章中考虑的简单行为。脊髓运动神经元 对于检查中枢突触机制特别有用,因为它们具有大的、可接近的细胞体,最重要的是,它们同时接收兴奋性和抑制性连接,因此使我们能够在细胞水平上研究神经系统的综合作用。

\section{中枢神经元接收兴奋性和抑制性输入}
为了分析介导牵张反射的突触,Eccles 激活了大量的感觉细胞轴突,这些轴突支配股四头肌(伸肌)中的牵张受体器官(图 13-1A、B)。 如今,同样的实验可以通过刺激单个感觉神经元来完成。

通过微电极将足够的电流传递到支配伸肌的牵张感受器感觉神经元的细胞体中会产生动作电位。 这反过来会在运动神经元中产生一个小的兴奋性突触后电位 (EPSP),该电位支配由感觉神经元监测的同一块肌肉(在本例中为股四头肌)(图 13–1B,上图)。 由一个感觉细胞(单一 EPSP)产生的 EPSP 使伸肌运动神经元去极化小于 1 mV,通常仅为 0.2 至 0.4 mV,远低于产生动作电位的阈值。 通常,需要 10 mV 或更多的去极化才能达到阈值。

因此,运动神经元中动作电位的产生需要大量感觉神经元近乎同步的放电。 这可以在实验中观察到,在该实验中,通过细胞外电极传递电流来刺激感觉神经元群。 随着细胞外刺激强度的增加,更多的感觉传入纤维被兴奋,EPSP产生的去极化变大。 去极化最终变得足够大,使运动神经元轴突初始段(具有最低阈值的区域)的膜电位达到动作电位的阈值。

除了在伸肌运动神经元中产生的 EPSP 外,刺激伸肌牵张受体神经元还会在支配屈肌的运动神经元中产生小的抑制性突触后电位 (IPSP),这与伸肌具有拮抗作用(图 13– 1B,下面板)。 这种超极化作用是由抑制性中间神经元产生的,它从伸肌的感觉神经元接收兴奋性输入,进而与支配屈肌的运动神经元形成突触。 在实验室中,可以在细胞内刺激单个中间神经元,以直接在运动神经元中引发一个小的单一 IPSP。 整个中间神经元群的细胞外激活会引发更大的 IPSP。 如果足够强,IPSP 可以抵消 EPSP 并防止膜电位达到阈值。


\section{兴奋性和抑制性突触具有独特的超微结构并针对不同的神经元区域}
正如我们在第 11 章中了解到的,突触电位的作用——无论是兴奋性的还是抑制性的——不是由突触前神经元释放的递质类型决定的,而是由递质激活的突触后细胞中的离子通道类型决定的。 尽管一些递质可以产生 EPSP 和 IPSP,但通过作用于不同突触处的不同类别的离子型受体,大多数递质产生一种主要类型的突触反应; 也就是说,递质通常是抑制性的或兴奋性的。 例如,在脊椎动物的中枢神经系统中,释放谷氨酸的神经元通常作用于产生兴奋的受体; 释放 γ-氨基丁酸 (GABA) 或甘氨酸的神经元作用于产生抑制的受体。

兴奋性和抑制性神经元的突触末端可以通过它们的超微结构来区分。 两种形态类型的突触在大脑中很常见:灰色类型 I 和 II(以 E. G. Gray 命名,他使用电子显微镜描述了它们)。 大多数 I 型突触是谷氨酸能和兴奋性的,而大多数 II 型突触是 GABA 能和抑制性的。 I 型突触有圆形突触小泡,突触前膜上有一个电子致密区(活性区),突触后膜上有一个更大的电子致密区,与活性区相对(称为突触后密度),这使得 I 型突触 突触外观不对称。 II 型突触具有椭圆形或扁平的突触小泡和不太明显的突触前膜特化和突触后密度,导致更对称的外观(图 13-2)。 (虽然 I 型突触主要是兴奋性的,II 型突触是抑制性的,但这两种形态类型已被证明只是对递质生物化学的初步近似。免疫细胞化学提供了递质类型之间更可靠的区别,如第 16 章所述)。

虽然树突通常位于突触后,而轴突末端位于突触前,但神经细胞的所有四个区域——轴突、突触前末端、细胞体和树突——都可以是化学突触的突触前或突触后部位。 如图 13-2 所示,最常见的接触类型是轴突、轴体和轴突-轴突(按照惯例,首先确定突触前元件)。 兴奋性突触通常是轴突状的,主要发生在树突棘上。 抑制性突触通常形成于树突轴、细胞体和轴突起始段。 还发现了树突状突触和体细胞突触,但它们很少见。

作为一般规则,突触与轴突初始段的接近程度被认为决定了其有效性。 在靠近细胞体的部位产生的给定突触后电流将在轴突起始段的触发区产生更大的膜电位变化,因此对动作电位输出的影响比在更远的部位产生的相同电流更大 树突。 这是因为当突触电位传播到细胞体时,一些在远处进入突触后膜的电荷会从树突膜中泄漏出来(第 9 章)。 一些神经元通过在远端突触放置比近端突触更多的谷氨酸受体来补偿这种影响,确保树突树上不同位置的输入在初始段具有更等效的影响。 与轴突和轴体输入相反,大多数轴突突触对突触后细胞的触发区没有直接影响。 相反,它们通过控制从突触前末梢释放的递质数量来影响神经活动(第 15 章)。



\section{兴奋性突触传递由可渗透阳离子的离子型谷氨酸受体通道介导}
从牵张受体感觉神经元的突触前末梢释放的兴奋性递质是氨基酸 l-谷氨酸,它是大脑和脊髓中的主要兴奋性递质。 Eccles 和他的同事发现,脊髓运动细胞中的 EPSP 是离子型谷氨酸受体通道打开的结果,该通道可渗透 Na+ 和 K+。 这种离子机制类似于第 12 章中描述的 ACh 在神经肌肉接头处产生的机制。与 ACh 受体通道一样,谷氨酸受体通道以几乎相等的渗透性传导 Na+ 和 K+。 因此,流过这些通道的电流的反转电位为 0 mV(见图 12–7)。

谷氨酸受体可分为两大类:离子型受体和代谢型受体(图 13-3)。 存在三种主要类型的离子型谷氨酸受体:AMPA、红藻氨酸和 NMDA,根据激活它们的药理学激动剂的类型命名(α-amino-3-hydroxy-5-methylisoxazole-4-propionic acid、kainate 和 N -甲基-d-天冬氨酸,分别)。 这些受体对拮抗剂也有不同的敏感性。 NMDA 受体被药物 APV(2-氨基-5-膦戊酸)选择性阻断。 AMPA 和红藻氨酸受体不受 APV 的影响,但两者都被药物 CNQX(6-cyano-7-nitroquinoxaline-2,3-dione)阻断。 由于这种共同的药理学敏感性,这两种类型有时被称为非 NMDA 受体。 NMDA 和非 NMDA 受体之间的另一个重要区别是 NMDA 受体通道对 Ca2+ 具有高度渗透性,而大多数非 NMDA 受体则不然。 有几种类型的代谢型谷氨酸受体,其中大部分可以被反式-(1S,3R)-1-氨基-1,3-环戊二甲酸 (ACPD) 激活。

所有离子型谷氨酸受体的作用都是兴奋性或去极化的,因为它们的离子电流的反转电位接近于零,导致通道开放在负膜电位下产生去极化内向电流。 相反,代谢型受体可以产生兴奋或抑制,这取决于它们调节的离子电流的逆转电位以及它们是促进通道开放还是通道关闭。

\subsection{离子型谷氨酸受体由一个大基因家族编码}
在过去的 30 年里,已经确定了编码所有主要神经递质受体亚基的多种基因。 此外,这些亚基基因中有许多是可变剪接的,从而产生了更多的多样性。 这种分子分析证明了受体结构之间的进化联系,使我们能够将它们分为三个不同的家族(图 13-4)。

离子型谷氨酸受体家族包括 AMPA、红藻氨酸和 NMDA 受体。 与编码 NMDA 受体的基因相比,编码 AMPA 和红藻氨酸受体的基因彼此之间的关系更为密切。 令人惊讶的是,谷氨酸受体家族与另外两个编码离子型受体的基因家族几乎没有相似之处(其中一个编码烟碱 ACh、GABA 和甘氨酸受体,另一个编码 ATP 受体,如后所述)。

AMPA、红藻氨酸和 NMDA 受体是由两种或多种类型的相关亚基组成的四聚体,所有四个亚基都排列在一个中心孔周围。 AMPA 受体亚基由四个独立的基因 (GluA1–GluA4) 编码,而红藻氨酸受体亚基由五个不同的基因 (GluK1–GluK5) 编码。 AMPA 受体 GluA3 亚基的自身抗体被认为在某些形式的癫痫中发挥重要作用。 这些抗体实际上通过激活含有 GluA3 的受体来模拟谷氨酸,从而导致过度兴奋和癫痫发作。 另一方面,NMDA 受体由一个家族编码,该家族由五个基因组成,分为两组:GluN1 基因编码一种亚基,而四个不同的 GluN2 基因 (A–D) 编码第二种亚基。 每个 NMDA 受体包含两个 GluN1 亚基和两个 GluN2 亚基。



\subsection{谷氨酸受体由一组结构模块构成}
所有离子型谷氨酸受体亚基都具有具有相似基序的共同结构。 Eric Gouaux 及其同事对离子型谷氨酸受体的结构提供了重要见解,最初是通过由四个 GluA2 亚基组成的 AMPA 受体的 X 射线晶体学模型。 这些亚基有一个大的细胞外氨基末端结构域,在一级氨基酸序列中紧随其后的是一个细胞外配体结合结构域和一个跨膜结构域(图 13-4B 和 13-5)。 跨膜结构域包含三个跨膜 α-螺旋(M1、M3 和 M4)和位于 M1 和 M3 螺旋之间的环 (M2),该环可进出细胞膜的细胞质侧。 这个 M2 环类似于 K+ 通道的孔隙衬里 P 环,有助于形成通道的选择性过滤器(参见图 8-12)。

两个细胞外结构域都与细菌氨基酸结合蛋白结构域同源。 配体结合域是双叶蛤壳状结构(图 13-5A),而氨基末端域与代谢型谷氨酸受体的谷氨酸结合域同源,但不结合谷氨酸。 相反,在离子型谷氨酸受体中,该结构域参与亚基组装、谷氨酸以外的配体对受体功能的调节和/或与其他突触蛋白的相互作用以调节突触发育。

配体结合域由蛋白质线性序列中的两个不同区域形成。 一个区域包括氨基末端结构域的末端直至 M1 跨膜螺旋; 第二个区域由连接 M3 和 M4 螺旋的大细胞外环形成(图 13-5A)。 在离子型受体中,蛤壳内谷氨酸分子的结合会触发蛤壳叶的闭合; 竞争性拮抗剂也与翻盖结合,但无法触发翻盖关闭。 这表明与蛤壳闭合相关的构象变化对于打开离子通道很重要。

除了形成受体通道的核心亚基外,AMPA 受体还包含其他(或辅助)亚基,可调节受体向膜的运输和功能。 一类重要的辅助亚基包括跨膜 AMPA 受体调节蛋白 (TARP)。 TARP 亚基有四个跨膜结构域,它与成孔 AMPA 受体亚基的结合增强了表面膜运输、突触定位和 AMPA 受体的门控。 第一个被确认的 TARP 家族成员是观星蛋白,它是通过观星突变小鼠的基因筛选分离出来的,之所以这样命名是因为这些动物倾向于将头向后仰并向上凝视。 stargazin 的缺失会导致小脑颗粒细胞的 AMPA 受体完全缺失,从而导致小脑性共济失调和频繁癫痫发作。 TARP 家族的其他成员同样需要 AMPA 受体运输到其他类型神经元的表面膜。

高分辨率冷冻电子显微镜揭示了与 AMPA 受体亚基相关的 TARP 亚基结构(图 13-5D,E)。 这些研究表明,TARP 亚基与 AMPA 受体的配体结合域蛤壳之间的相互作用可以使受体稳定在谷氨酸结合的开放状态,从而增强通道开放时间、单通道电导和对谷氨酸的亲和力。

鉴于谷氨酸受体的各种亚型之间的同源性,红藻氨酸和 NMDA 受体的整体结构与同源 GluA2 受体的结构相似也就不足为奇了。 然而,有一些重要的差异导致不同受体的不同生理功能。 NMDA 受体通道对 Ca2+ 的高渗透性已定位于成孔 M2 环中的单个氨基酸残基。 所有 NMDA 受体亚基在孔中的这个位置都含有中性残基天冬酰胺。 在大多数类型的 AMPA 受体亚基中,该位置的残基是不带电荷的氨基酸谷氨酰胺; 然而,在 GluA2 亚基中,相应的 M2 残基是精氨酸,一种带正电荷的碱性氨基酸。 即使包含一个 GluA2 亚基也会阻止 AMPA 受体通道传导 Ca2+(图 13-6B),这很可能是精氨酸强静电排斥的结果。 缺少 GluA2 亚基的细胞中 AMPA 受体通道的打开会产生显着的 Ca2+ 流入,因为这些受体的孔缺乏带正电的精氨酸残基。

有趣的是,GluA2 基因的 DNA 在 M2 环的这个位置不编码精氨酸残基,而是编码谷氨酰胺残基。 转录后,GluA2 mRNA 中的谷氨酰胺密码子通过称为 RNA 编辑的酶促过程被替换为精氨酸密码子(图 13-6A)。 使用转基因小鼠研究了这种 RNA 编辑的重要性,该小鼠的 GluA2 基因经过工程改造,因此谷氨酰胺密码子中的相关核苷酸不能再变为精氨酸。 这些小鼠在出生后几周内出现癫痫发作并死亡,这可能是因为所有 AMPA 受体的高 Ca2+ 渗透性导致细胞内 Ca2+ 过量。

\subsection{NMDA 和 AMPA 受体由突触后密度的蛋白质网络组织}
不同的谷氨酸受体是如何在兴奋性突触中定位和排列的? 与大多数离子型受体一样,谷氨酸受体通常聚集在膜的突触后位点,与谷氨酸能突触前末端正好相反。 成熟神经系统中的绝大多数兴奋性突触同时包含 NMDA 和 AMPA 受体,而在早期发育中,仅包含 NMDA 受体的突触很常见。 受体在单个突触处的定位和表达模式取决于构成突触后密度并帮助组织突触后细胞膜三维结构的大量调节蛋白。

突触后密度 (PSD) 是一种非常稳定的结构,允许其生化分离、纯化和表征。 完整和分离的 PSD 的电子显微镜研究提供了其结构的惊人详细视图(图 13-7A)。 通过使用金标抗体,可以识别突触后膜的特定蛋白质成分,包括谷氨酸受体的位置和数量。 典型的 PSD 直径约为 350 nm,包含大约 20 个 NMDA 受体,它们往往位于 PSD 的中心附近,以及 10 到 50 个 AMPA 受体,它们不太集中。 代谢型谷氨酸受体位于外围,PSD 主要区域之外。 所有三种受体类型都与多种细胞质和膜蛋白相互作用,以确保它们的正确定位(图 13-7C)。

PSD 中对谷氨酸受体聚集很重要的最突出的蛋白质之一是 PSD-95(95 kD 分子量的 PSD 蛋白质)。 PSD-95 是一种膜相关蛋白,包含三个重复区域——所谓的 PDZ 结构域——对蛋白质-蛋白质相互作用很重要。 (PDZ 结构域以首次发现它们的三种蛋白质命名:PSD-95、果蝇中的 DLG 肿瘤抑制蛋白和一种称为 ZO-1 的蛋白质。) 一些蛋白质。 在 PSD-95 中,PDZ 结构域结合 NMDA 受体和 Shaker 型电压门控 K+ 通道,从而将这些通道定位并集中在突触后位点。 PSD-95 还与突触后膜蛋白 neuroligin 相互作用,后者与突触间隙中的突触前膜蛋白 neurexin 相互作用,这是一种对突触发育很重要的相互作用。 neuroligin 的突变被认为是导致某些自闭症病例的原因。

尽管 PSD-95 不直接与 AMPA 受体结合,但它确实与 TARP 亚基相互作用。 AMPA 受体在突触后膜中的正确定位取决于 TARP 亚基的羧基末端与 PSD-95 之间的相互作用。 AMPA 受体还与称为 GRIP 的独特 PDZ 结构域蛋白结合,而代谢型谷氨酸受体与另一种称为 Homer 的 PDZ 结构域蛋白相互作用。 除了与受体相互作用外,具有 PDZ 结构域的蛋白质还与许多其他细胞蛋白质相互作用,包括与肌动蛋白细胞骨架结合的蛋白质,从而提供支架,围绕其构建突触后蛋白质复合物。 事实上,对 PSD 的生化分析已经确定了数十种参与 NMDA 或 AMPA 受体复合物的蛋白质。

\subsection{NMDA 受体具有独特的生物物理和药理学特性}

NMDA 受体有几个有趣的特性,可以将其与 AMPA 受体区分开来。 如前所述,NMDA 受体对 Ca2+ 具有特别高的渗透性。 此外,NMDA 受体在配体门控通道中是独一无二的,因为它的开放取决于膜电压和递质结合。

电压依赖性是由一种与产生动作电位的电压门控通道所采用的机制完全不同的机制引起的。 在后者中,膜电位的变化通过固有电压传感器转化为通道中的构象变化。 然而,在 NMDA 受体中,去极化从通道中移除了外源性栓塞。 在静息膜电位 (−65 mV) 下,细胞外 Mg2+ 与通道孔隙中的一个位点紧密结合,从而阻断离子电流。 但是,当膜去极化时(例如,通过打开 AMPA 受体通道),Mg2+ 通过静电排斥从通道中排出,从而允许 Na+、K+ 和 Ca2+ 流动(图 13-8)。 NMDA 受体还有一个更有趣的特性,即被致幻剂苯环利定(PCP,也称为天使尘)和实验化合物 MK801 抑制。 两种药物都结合到通道孔隙中与 Mg2+ 结合位点不同的位点(图 13-3A)。

在大多数谷氨酸能中央突触中,突触后膜包含 NMDA 和 AMPA 受体。 通过 NMDA 和 AMPA 受体的电流对总兴奋性突触后电流 (EPSC) 的相对贡献可以在电压钳实验中使用药理学拮抗剂进行量化(图 13-9)。 由于 NMDA 受体在大多数神经元的正常静息电位下主要被 Mg2+ 抑制,因此 EPSC 主要由流经 AMPA 受体的电荷决定。 该电流具有非常快速的上升和衰减阶段。 然而,当神经元去极化并且 Mg2+ 被驱出 NMDA 受体的口时,更多的电荷流过它们。 因此,当满足两个条件时,NMDA 受体通道最大程度地传导电流:存在谷氨酸,并且细胞去极化。 也就是说,NMDA 受体充当分子“巧合检测器”,在突触前和突触后细胞同时激活时打开。 此外,由于其配体门控的内在动力学,通过 NMDA 受体通道的电流上升和衰减的时间进程比通过 AMPA 受体通道的电流慢得多。 因此,NMDA 受体有助于 EPSC 和 EPSP 的晚期、缓慢期。

由于大多数谷氨酸能突触都含有能够自行触发动作电位的 AMPA 受体,那么 NMDA 受体的功能是什么? 乍一看,这些受体的功能更加令人费解,因为它们的内在通道通常在静息电位时被 Mg2+ 阻断。 然而,NMDA 受体通道对 Ca2+ 的高渗透性赋予它们产生细胞内 [Ca2+] 显着升高的特殊能力,从而激活各种钙依赖性信号级联反应,包括几种不同的蛋白激酶(第 15 和 53 章)。 因此,NMDA 受体激活可以将电信号转化为生化信号。 其中一些生化反应通过称为长期突触可塑性的一系列过程导致突触强度的长期变化,这对于在早期发育过程中完善突触连接和调节成人大脑中的神经回路(包括长期关键回路)非常重要 - 长期记忆。

\subsection{NMDA 受体的特性是长期突触可塑性的基础}
1973 年,Tim Bliss 和 Terje Lomo 发现短暂的高强度和高频突触刺激(称为破伤风)会导致海马体中兴奋性突触传递的长期增强 (LTP),海马体是大脑的一个区域 许多形式的长期记忆需要哺乳动物的大脑(图 13-10;见第 53 和 54 章)。 随后的研究表明,LTP 需要通过 NMDA 受体通道流入 Ca2+,而 NMDA 受体通道会在强直刺激期间响应谷氨酸释放和强烈的突触后去极化的联合作用而打开。 如果在阻断 NMDA 受体的 APV 存在的情况下递送破伤风,或者如果突触后神经元被注射螯合细胞内 Ca2+ 的化合物,则 LTP 被阻断。

突触后细胞中 Ca2+ 的升高被认为通过激活突触后生化级联反应来增强突触传递,从而触发额外的 AMPA 受体插入突触后膜。 在某些情况下,突触后 Ca2+ 会触发逆行信使的产生,这是一种化学信号,可增强突触前末梢的递质释放(第 14 章)。 正如我们稍后将讨论的那样,Ca2+ 积累和生化激活在很大程度上局限于被破伤风刺激激活的单个脊柱。 因此,LTP 是输入特定的; 只有那些在强直刺激期间被激活的突触才会被增强。

诱导 LTP 所需的长时间高频突触前放电不太可能在生理条件下实现。 然而,如果单个突触前刺激以低频与触发的一个或多个突触后动作电位配对,则可以诱导一种更具生理相关性的可塑性形式,称为尖峰时间依赖性可塑性 (STDP),提供足够的去极化以减轻 Mg2+ 阻断 NMDA 受体孔。 突触前活动必须先于突触后放电,遵循心理学家唐纳德·赫布 (Donald Hebb) 于 1949 年提出的关于在联想记忆存储过程中单个神经元如何组合成功能组件的规则。 许多证据表明,LTP、STDP 或相关过程为记忆存储(第 53 和 54 章)和发育过程中的突触连接微调(第 49 章)提供了重要的细胞机制。

\subsection{NMDA 受体导致神经精神疾病}
不幸的是,通过 NMDA 受体募集 Ca2+ 也有不利之处。 过高浓度的谷氨酸被认为会导致突触后神经元中的 Ca2+ 超载,这种情况可能对神经元有毒。 在组织培养中,即使是短暂接触高浓度谷氨酸也会杀死许多神经元,这种作用称为谷氨酸兴奋性毒性。 高浓度的细胞内 Ca2+ 被认为会激活钙依赖性蛋白酶和磷脂酶,并导致产生对细胞有毒的自由基。

谷氨酸毒性可能导致中风后的细胞损伤、癫痫持续状态患者经历的快速反复癫痫发作时发生的细胞死亡,以及亨廷顿病等退行性疾病。 选择性阻断 NMDA 受体的药物可以防止谷氨酸的毒性作用,并且已经过临床测试。 迄今为止,伴随 NMDA 受体阻断的幻觉限制了此类化合物的用途。 通过阻断 NMDA 受体功能来控制兴奋性毒性的尝试的另一个并发症是 NMDA 受体激活的生理水平实际上可以保护神经元免受损伤和细胞死亡。

并非所有由 NMDA 受体介导的生理学和病理生理学效应都可能由 Ca2+ 流入引起。 越来越多的证据表明,谷氨酸与 NMDA 受体的结合可能会导致受体发生构象变化,从而独立于离子通量激活下游细胞内信号通路。 NMDA 受体的这种促代谢功能可能导致长期抑制,这是一种突触可塑性形式,其中低频突触活动导致谷氨酸能突触传递的长期减少,这与 LTP 相反。 NMDA 受体的促代谢作用也可能有助于 β-淀粉样蛋白(与阿尔茨海默病有关的肽片段)抑制突触功能的作用。

许多证据表明精神分裂症中的 NMDA 受体功能障碍。 使用苯环利定或全身麻醉剂氯胺酮(PCP 的衍生物)等药物对 NMDA 受体进行药理学阻断会产生类似于精神分裂症相关幻觉的症状; 相反,某些抗精神病药物会增强通过 NMDA 受体通道的电流。 在抗 NMDA 受体脑炎中可以看到与精神分裂症的一个特别显着的联系,这是一种自身免疫性疾病,其中针对 NMDA 受体的抗体的产生降低了膜中受体的水平。 患有这种疾病的人经常会出现严重的癫痫发作,这很可能是由于 GABA 能中间神经元中 NMDA 受体兴奋减少导致抑制性张力丧失,以及精神病,包括幻觉和其他类似精神分裂症的症状。 降低抗体水平的治疗通常会导致这些症状完全缓解。 最近的全基因组连锁分析进一步支持了 NMDA 受体功能下降可能导致精神分裂症症状的观点,表明 NR2A 基因与精神分裂症之间存在关联。 NMDA 受体与神经精神疾病之间的另一个联系是由以下发现提供的:低剂量的氯胺酮发挥快速而强大的抗抑郁作用。


\section{快速抑制性突触作用由离子型 GABA 和甘氨酸受体-可渗透氯离子的通道介导}
虽然谷氨酸能兴奋性突触占大脑中绝大多数突触,但抑制性突触在神经系统中起着至关重要的作用,既可以防止过度兴奋,也可以调节神经元网络的放电模式。 脊髓运动神经元和大多数中枢神经元中的 IPSP 是由氨基酸神经递质 GABA 和甘氨酸产生的。

GABA 作用于离子型和代谢型受体。 GABAA 受体是一种离子型受体,可直接打开 Cl- 通道。 GABAB 受体是一种代谢型受体,可激活第二信使级联,通常会间接激活 K+ 通道(第 15 章)。 甘氨酸是大脑中一种不太常见的抑制性递质,它还能激活直接打开 Cl- 通道的离子型受体。 甘氨酸是抑制拮抗运动神经元的中间神经元在脊髓中释放的主要递质。

\subsection{离子型谷氨酸、GABA 和甘氨酸受体是由两个不同的基因家族编码的跨膜蛋白}
形成 GABAA 和甘氨酸受体的各个亚基由两组不同但密切相关的基因编码。 更令人惊讶的是,这些受体亚基在结构上与烟碱 ACh 受体亚基相关,尽管后者选择阳离子并因此具有兴奋性。 因此,正如我们在上面看到的(图 13-4),三种类型的受体亚基是一个大基因家族的成员。

与烟碱 ACh 受体通道一样,GABAA 和甘氨酸受体通道也是五聚体。 GABAA 受体通常由两个 α-、两个 β- 和一个 γ- 或 δ- 亚基组成,并通过在两个 α- 和 β- 亚基之间形成的裂隙中结合两个 GABA 分子而被激活。 甘氨酸受体由三个 α- 和两个 β- 亚基组成,需要结合最多三个配体分子才能打开。 每个 GABAA 和甘氨酸受体亚基的跨膜拓扑结构类似于烟碱 ACh 受体亚基,由一个大的细胞外配体结合结构域和四个疏水性跨膜 α-螺旋(标记为 M1、M2、M3 和 M4)组成, M2 螺旋形成通道孔的衬里(图 13-4A)。 然而,M2 结构域侧翼的氨基酸与烟碱 ACh 受体的氨基酸明显不同。 如第 12 章所述,ACh 受体的孔包含带负电荷的酸性残基环,有助于通道选择阳离子而不是阴离子。 相反,GABA 和甘氨酸受体通道在同源位置包含中性或带正电荷的碱性残基,这有助于这些通道对阴离子的选择性。

大多数主要类别的受体亚基由多个相关基因编码。 因此,有六种类型的 GABAA α-亚基 (α1-α6)、三种 β-亚基 (β1-β3)、三种 γ-亚基 (γ1-γ3) 和一种 δ-亚基。 这些不同亚型的基因通常在不同类型的神经元中差异表达,赋予它们的抑制性突触不同的特性。 这些亚基在完全组装的五聚体受体中的可能组合排列提供了受体的巨大潜在多样性。

GABAA 和甘氨酸受体在疾病和药物作用中起着重要作用。 GABAA 受体是多种药物的靶标,这些药物在临床上很重要,但在社会上却被滥用,包括全身麻醉药、苯二氮卓类药物、巴比妥类药物和酒精。 全身麻醉剂,无论是气体还是可注射的化合物,都会导致意识丧失,因此在手术过程中被广泛使用。 苯二氮卓类药物是抗焦虑剂和肌肉松弛剂,包括地西泮 (Valium)、劳拉西泮 (Ativan) 和氯硝西泮 (Klonopin)。 唑吡坦 (Ambien) 是一种促进睡眠的苯二氮卓类化合物。 巴比妥类药物包括一组不同的催眠药,包括苯巴比妥和司可巴比妥。

不同类别的化合物——GABA、全身麻醉药、苯二氮卓类药物、巴比妥类药物和酒精——结合受体上的不同位点,但作用相似以增加 GABA 受体通道的开放。 例如,GABA 与 α- 和 β- 亚基之间的裂缝结合,而苯二氮卓类药物与 α- 和 γ- 亚基之间的裂缝结合。 此外,这些类别药物中任何一种的结合都会影响其他药物的结合。 例如,当 GABA 也被结合时,苯二氮卓类药物(或巴比妥类药物)与受体通道的结合更牢固,这种紧密结合有助于将通道稳定在开放状态。 以这种方式,各种化合物都增强了抑制性突触传递。

这些不同的化合物都作用于 GABAA 受体以促进通道开放,如何产生如此多样的行为和心理效应,例如,减少焦虑与促进睡眠? 事实证明,许多这些化合物选择性地结合特定的亚基类型,这些亚基类型可以在大脑不同区域的不同类型的神经元中表达。 例如,唑吡坦选择性地结合含有 α1-亚基的 GABAA 受体。 相反,苯二氮卓类药物的抗焦虑作用需要与 α2- 和 γ- 亚基结合。

除了作为重要的药理靶标外,GABAA 和甘氨酸受体还是疾病和毒物的靶标。 甘氨酸受体 α 亚基的错义突变是一种遗传性神经系统疾病的基础,称为家族性惊吓病(或惊跳过度),其特征是肌肉张力异常高和对噪音反应过度。 这些突变减少了甘氨酸受体的开放,从而降低了脊髓中抑制性传递的正常水平。 有毒的马钱子碱是一种植物生物碱化合物,它通过阻断甘氨酸受体和降低抑制作用而引起惊厥。 导致 GABAA 受体 α 和 γ 亚基截断的无义突变与先天性癫痫有关。

\subsection{通过 GABAA 和甘氨酸受体通道的氯离子电流通常会抑制突触后细胞}
GABA 受体的功能与其生物物理特性密切相关。 Eccles 和他的同事通过在刺激突触前抑制性中间神经元的同时系统地改变运动神经元中静息膜电位的水平,确定了 IPSP 在脊髓运动神经元中的离子机制(图 13-11)。

当运动神经元膜保持在正常静息电位 (-65 mV) 时,当突触前中间神经元受到刺激时会产生一个小的超极化电位。 当运动神经元膜保持在-70 mV 时,刺激中间神经元时不会记录到电位变化。 但在低于-70 mV 的电位下,运动神经元在抑制性中间神经元受到刺激后会产生去极化反应。 这种 -70 mV 的反转电位对应于脊髓运动神经元中的 Cl- 平衡电位(Cl- 的细胞外浓度远大于细胞内浓度)。 因此,在 -70 mV 时,Cl- 沿着其化学浓度梯度扩散到细胞中的趋势被反对 Cl- 流入的电力(负膜电位)所平衡。 根据能斯特方程的预测,用非渗透性阴离子替代细胞外 Cl- 会减小 IPSP 的大小,并将反转电位转变为更正的值。 因此,IPSP 是由 Cl- 电导的增加引起的。

已经使用膜片钳技术测量了通过单个 GABA 和甘氨酸受体通道的电流,单一电流。 两种递质都激活以全氮-无方式打开的 Cl- 通道,类似于 ACh 和谷氨酸门控通道的打开。 GABA 和甘氨酸对神经元放电的抑制作用取决于两个相关机制。 首先,在典型的神经元中,-65 mV 的静息电位比 ECl (-70 mV) 略微更正。 在此静息电位下,驱动 Cl− 进入电池的化学力略大于反对 Cl− 流入的电力——也就是说,Cl− (Vm − ECl) 上的电化学驱动力为正。 因此,根据 ICl = gCl (Vm − ECl) 的关系,Cl− 通道的打开会导致正电流。 因为电荷载体是带负电的 Cl- 离子,正电流对应于 Cl- 流入神经元,沿其电化学梯度下降。 这导致膜内部负电荷的净增加——膜变得超极化。

然而,一些中枢神经元的静息电位大约等于 ECl。 在此类电池中,Cl- 电导的增加不会改变膜电位 - 电池不会变得超极化 - 因为 Cl- 上的电化学驱动力几乎为零。 然而,此类细胞中 Cl- 通道的打开仍然会抑制细胞响应几乎同时发生的 EPSP 而激发动作电位。 这是因为兴奋性输入产生的去极化取决于所有类型开放通道的电池的加权平均值 - 即兴奋性和抑制性突触电导的电池以及静息电导 - 加权因子等于总 特定类型通道的电导(参见第 12 章,后记)。 因为 Cl- 通道的电池位于静息电位附近,打开这些通道有助于通过增加 Cl- 电池的加权因子在 EPSP 期间将膜保持在其静息电位附近。

Cl− 通道的开放对 EPSP 大小的影响也可以用欧姆定律来描述。 因此,EPSP 期间的去极化幅度 ΔVEPSP 由下式给出:

\begin{equation}\label{depolarization_amplitude}
	\delta V_{EPSP} = I_{EPSP} / g_1,
\end{equation}

其中 IEPSP 是兴奋性突触电流,gl 是膜中所有其他开放通道的电导,包括静息通道和传输门控 Cl- 通道。 因为 Cl- 通道的开放增加了静息电导,即使神经元更加渗漏,EPSP 期间的去极化减少。 突触抑制的这种后果称为短路或分流效应。

通过抵消突触兴奋,突触抑制可以对由于内在起搏器通道的存在而自发活跃的神经元中的动作电位放电施加强有力的控制。 这种称为抑制的雕刻作用的功能塑造了此类细胞的放电模式(图 13-12)。 事实上,抑制的这种塑造作用可能发生在所有神经元中,导致神经元脉冲的时间模式和神经回路同步的控制。

突触电导的不同生物物理特性可以理解为突触后神经元执行的不同数学运算。 因此,使细胞超极化的抑制性输入对兴奋性输入进行减法,而电导增加的分流效应进行除法。 添加兴奋性输入(或移除非分流抑制性输入)会导致求和。 最后,兴奋性输入与去除抑制性分流的组合产生乘法。 然而,这些算术效应通常是混合的,并且随着神经元膜电位不断变化而随时间变化,导致通过 GABAA 受体通道对 Cl− 的驱动力发生变化。

在一些细胞中,例如那些具有代谢型 GABAB 受体的细胞,抑制是由 K+ 通道的开放引起的。 因为神经元的 K+ 平衡电位 (EK = −80 mV) 总是对静息电位负,所以打开 K+ 通道比打开 Cl- 通道(假设具有相似大小的突触电导)更能抑制细胞,产生更多“ 减法”抑制。 与 GABAA 响应相比,GABAB 响应开启得更慢并且持续时间更长。

矛盾的是,在某些情况下,脑细胞中 GABAA 受体的激活会引起兴奋。 这是因为在强烈刺激后 Cl- 的流入可能非常大,以至于细胞内 Cl- 浓度显着增加。 它甚至可能翻倍。 结果,Cl- 平衡电位可能变得比静息电位更正。 在这些条件下,Cl- 通道的开放导致 Cl- 流出和神经元去极化。 这种去极化 Cl- 反应通常发生在新生动物的许多神经元中,其中细胞内 Cl- 浓度即使在静止时也往往很高。 这是因为负责维持低细胞内 Cl- 的 K+-Cl- 协同转运蛋白在早期发育期间以低水平表达(第 9 章)。 去极化 Cl- 反应也可能发生在更成熟神经元的远端树突中,也可能发生在它们的轴突初始段。 成人中的这种兴奋性 GABAA 受体作用可能有助于癫痫放电,其中观察到大量、同步和去极化的 GABA 反应。


\section{中枢神经系统中的一些突触动作依赖于其他类型的离子型受体}
大脑中少数快速兴奋性突触动作是由作用于 5-HT3 类离子型受体通道的神经递质血清素 (5-HT) 介导的。 这些五聚体受体由具有四个跨膜区段的亚基组成,在结构上类似于烟碱 ACh 受体。 与 ACh 受体通道一样,5-HT3 受体通道可渗透单价阳离子,并具有接近 0 mV 的逆转电位。

三磷酸腺苷 (ATP) 的离子型受体在其他选定的突触中发挥兴奋作用,并构成第三个传递离子通道家族。 这些所谓的嘌呤能受体(以腺苷中的嘌呤环命名)出现在由自主神经节的交感神经元以及某些中枢和外周神经元支配的平滑肌细胞上。 在这些突触处,ATP 激活一个离子通道,该通道可渗透单价阳离子和 Ca2+,反转电位接近 0 mV。 已经鉴定了编码该离子型 ATP 受体(称为 P2X 受体)家族的几个基因。 这些 ATP 受体的氨基酸序列和亚基结构不同于其他两个配体门控通道家族。 P2X 受体的 X 射线晶体结构显示它具有极其简单的组织,其中三个亚基(每个仅包含两个跨膜片段)围绕着一个中央孔(图 13-4C)。


\section{神经元将兴奋性和抑制性突触动作整合为单一输出}
中枢神经系统中的每个神经元不断受到来自许多其他神经元的一系列突触输入的轰击。 例如,一个运动神经元可能是多达 10,000 个不同突触前末梢的目标。 有些是兴奋性的,有些是抑制性的; 有些强,有些弱。 一些输入在其顶端树突的尖端接触运动细胞,其他的在近端树突上,一些在树突轴上,其他的在体细胞上。 不同的输入可以相互加强或抵消。 给定的神经元如何将这些信号整合成一致的输出?

正如我们之前看到的,单个突触前神经元产生的突触电位通常不足以将突触后细胞去极化到动作电位的阈值。 大多数拉伸敏感传入神经元在运动神经元中产生的 EPSP 振幅仅为 0.2 至 0.4 mV。 如果单个运动神经元中产生的 EPSP 线性求和,则至少 25 个传入神经元必须一起发射并释放发射器以使触发区去极化达到阈值所需的 10 mV。 但在突触后细胞接收兴奋性输入的同时,它也可能接收抑制性输入,通过减法或分流效应阻止动作电位的发射。

因此,任何单个兴奋性或抑制性突触的输入的净效应将取决于几个因素:突触的位置、大小和形状; 其他协同或拮抗突触的接近度和相对强度; 和细胞的静息电位。 此外,所有这一切都非常依赖于兴奋性和抑制性输入的时间。 输入在突触后神经元中通过称为神经元整合的过程进行协调。 这个细胞过程反映了整个神经系统所面临的任务。 在任何给定时刻,细胞都有两种选择:激发或不激发动作电位。 查尔斯·谢林顿 (Charles Sherrington) 将大脑在相互竞争的选项之间进行选择的能力描述为神经系统的综合作用。 他认为这种决策是大脑最基本的运作(见第 56 章)。

\subsection{突触输入整合在轴突初始段}
在大多数神经元中,启动动作电位输出的决定是在一个部位做出的:轴突起始段。 此处,细胞膜的动作电位生成阈值低于细胞体或树突,因为它具有更高密度的电压依赖性 Na+ 通道(图 13-13)。 随着膜去极化的每次增加,更多的 Na+ 通道打开,在轴突初始段提供比细胞其他地方更高的内向电流密度(每单位膜面积)。

在初始段,达到动作电位阈值 (-55 mV) 所需的去极化增量仅比静息水平 -65 mV 高 10 mV。 相反,细胞体的膜必须在达到其阈值 (-35 mV) 之前去极化 30 mV。 因此,突触激发首先在初始段释放膜区域,也称为触发区。 然后在该位点产生的动作电位使细胞体膜去极化至阈值,同时沿轴突传播。

因为神经元整合涉及传播到触发区的突触电位的总和,所以它受到神经元的两个被动膜特性的严重影响(第 9 章)。 首先,膜时间常数有助于确定突触电位响应 EPSC 的时间进程,从而控制时间总和,即突触后细胞中连续突触电位相加的过程。 具有大膜时间常数的神经元比具有较短时间常数的神经元具有更大的时间总和能力(图 13-14A)。 结果,时间常数越长,两个连续输入相加使细胞膜达到动作电位阈值的可能性就越大。

其次,细胞的长度常数决定了 EPSP 减少的程度,因为它从突触沿着树突的长度被动扩散到细胞体和轴突初始段(触发区)。 在具有较长长度常数的单元格中,信号以最小的衰减传播到触发区域; 在具有短长度常数的细胞中,信号随距离迅速衰减。 由于一个突触产生的去极化几乎永远不足以在触发区触发动作电位,因此必须将作用于突触后神经元不同部位的许多突触前神经元的输入加在一起。 这个过程称为空间求和。 与长度常数较短的神经元相比,长度常数较大的神经元更有可能被来自不同部位的输入带到阈值(图 13-14B)。

\subsection{GABA 能神经元的亚类靶向其突触后靶神经元的不同区域以产生具有不同功能的抑制作用}
与相对较少类型的谷氨酸能锥体神经元相比,哺乳动物中枢神经系统具有种类繁多的 GABA 能抑制性中间神经元,它们在发育起源、分子组成、形态和连通性方面各不相同(图 13-15)。 仅在海马体的一个分区中,就已鉴定出多达 20 种不同的 GABA 能神经元亚型。 不同类型的 GABAergic 中间神经元与其邻近的兴奋性和抑制性神经元形成广泛的突触连接。 因此,即使所有神经元中只有 20\% 是抑制性的,但在大多数大脑区域中,抑制和兴奋的总体水平往往接近平衡。 这导致调整神经回路以仅响应最显着的兴奋信息。 虽然中间神经元的多样性很难理解,但很明显,不同类型的中间神经元选择性地针对其突触后神经元的不同区域。

这种选择性靶向很重要,因为与兴奋性突触相关的抑制性输入的位置对于确定抑制的有效性至关重要(图 13-16)。 当在细胞体或轴突触发区附近启动抑制时,响应兴奋性输入的动作电位输出抑制更有效。 当树突向轴突移动时,由树突的兴奋电流产生的去极化必须沿着细胞体膜传递。 细胞体或轴突起始段的抑制作用会打开 Cl- 通道,从而增加 Cl- 电导并减少(通过分流)由扩散的兴奋电流产生的大部分去极化。 此外,当抑制输入针对细胞体而不是树突时,细胞体响应 IPSP 的超极化大小最大,这是由于树突的电缆特性导致树突 IPSP 衰减。

两类抑制性神经元,篮子细胞和枝形吊灯细胞,分别通过特异性靶向体细胞和轴突起始段来对神经元输出施加强大的控制(图 13-15)。 篮状细胞通常表达钙结合蛋白小白蛋白,是大脑中最常见的抑制性神经元类型。 也表达小白蛋白的枝形吊灯细胞具有轴突乔木,其具有分支模式和突触末端集群,类似于枝形吊灯的众多蜡烛。 在某些情况下,枝形吊灯细胞可能会反常地增强神经元放电,因为某些轴突中的 Cl- 反转电位可能与动作电位放电的阈值正相关。

第三类中间神经元,即 Martinotti 细胞,专门针对远端树突和棘。 除了 GABA 之外,这些常见的中间神经元还释放神经肽生长抑素。 树突远端部分的抑制作用可减少附近兴奋性输入产生的局部去极化,对其他树突分支上产生的 EPSP 影响较小。 生长抑素阳性的中间神经元响应兴奋性输入而缓慢激活,并生成随着重复激活(突触促进)而增加大小的 IPSP。 相比之下,表达小白蛋白的中间神经元会快速放电并生成 IPSP,IPSP 的大小会随着重复激活(突触抑制)而减小。 这些特性允许生长抑素和小清蛋白中间神经元分别控制神经信号后期和早期阶段通过神经回路的传播。

第四种主要类型的抑制性中间神经元表达神经肽血管活性肠肽 (VIP)。 这些中间神经元选择性地靶向其他中间神经元,从而降低神经回路中的抑制水平,从而增强整体兴奋,这一过程称为去抑制。

\subsection{树突是可以放大突触输入的电激发结构}
信号在树突中的传播最初被认为是纯粹被动的。 然而,1950 年代神经元细胞体和 1970 年代开始的树突的细胞内记录表明,树突可以产生动作电位。 事实上,我们现在知道,除了配体门控通道和静息通道外,大多数神经元的树突还包含电压门控 Na+、K+ 和 Ca2+ 通道。 事实上,树突电导的丰富多样性表明,中枢神经元依赖于复杂的电生理特性库来整合突触输入。

树突中电压门控 Na+ 和 Ca2+ 通道的功能之一是放大 EPSP。 在一些神经元中,树突膜中有足够密度的电压门控通道作为局部触发区。 这可以产生非线性电响应,增强到达树突远端部分的兴奋性输入产生的去极化。 当一个细胞有多个树突状触发区时,每个触发区都会汇总附近突触输入产生的局部兴奋和抑制; 如果净输入高于阈值,则可能会产生树突状动作电位,通常由电压门控 Na+ 或 Ca2+ 通道产生(图 13-17A)。 然而,树突中电压门控 Na+ 或 Ca2+ 通道的数量通常不足以支持动作电位到细胞体的全或无再生传播。 相反,树突中产生的动作电位通常是局部事件,这些事件以电紧张的方式传播到细胞体和轴突起始段,产生与细胞中其他输入信号整合的亚阈值体细胞去极化。

树突电压门控通道还允许在轴突初始段产生的动作电位向后传播到树突树中(图 13-17B)。 这些反向传播动作电位主要由树突状电压门控 Na+ 通道产生。 虽然这些动作电位的确切作用尚不清楚,但它们可能提供一种时间上精确的机制,通过提供去除 Mg2+ 阻滞所需的去极化来增强通过 NMDA 受体通道的电流,从而有助于诱导突触可塑性(图 13-10) )。

由于其电压依赖性,NMDA 受体能够介导树突中另一种类型的非线性整合。 适度的突触刺激能够激活足够数量的 AMPA 受体以产生中等水平的去极化,从而能够导致 Mg2+ 从一部分 NMDA 受体中排出。 当这些受体开始将阳离子传导到突触后树突中时,它们会产生进一步的去极化,从而导致更大程度的 Mg2+ 解阻,进一步增加 NMDA 受体 EPSC 的大小,从而导致更大的去极化。 在某些情况下,这会导致局部再生去极化,称为 NMDA 峰值。 这种 NMDA 尖峰纯粹是局部事件——它们不能在没有突触刺激的情况下主动传播,因为它们需要释放谷氨酸。 NMDA 峰值与不同形式的突触可塑性和增强突触输入的树突整合有关。

在什么条件下活性电导会影响树突整合? 现在有证据表明,树突可能会根据突触输入的精确时间和强度在被动和主动整合之间切换。 这种转换的一个有趣例子是一些皮层神经元对到达其远端和近端树突的输入作出反应的方式。 在许多神经元中,来自相对较近的神经元的输入到达树突的更近端区域,更靠近细胞体。 来自更远脑区的输入到达树突的远端尖端。 尽管远端树突的兴奋性突触输入通常只在体细胞产生非常小的去极化反应,但由于沿着树突电缆的电子衰减,这些输入与树突更近端区域的兴奋性输入配对时可以显着增强尖峰放电。 因此,近端部位的单个强 EPSP(或单个短暂的体细胞电流脉冲)通常会在轴突起始段产生单个动作电位,然后可以反向传播到树突中。 然而,当远端刺激与近端刺激配对时,反向传播尖峰与远端 EPSP 相加,触发一种持久类型的树突状尖峰,称为平台电位,这取决于电压门控 Ca2+ 通道和 NMDA 受体的激活。 当平台电位到达细胞体时,它可以以高达 100 Hz 的速率触发三个或更多尖峰的短暂爆发(图 13-17C)。 这些尖峰脉冲被认为提供了一种非常有效的方法来诱导长期突触可塑性并在脉冲传播到突触前末梢时释放递质。

一种更局部化的突触整合形式发生在树突棘中。 尽管一些兴奋性输入发生在树突状轴上,但大脑中接近 95\% 的兴奋性输入终止于棘突,令人惊讶地避免了树突状轴(见图 13-2)。 虽然棘的功能尚未完全了解,但它们的细颈为各种信号分子从棘头扩散到树突轴提供了屏障。 因此,通过 NMDA 受体的相对较小的 Ca2+ 电流可导致 [Ca2+] 相对较大的增加,该 [Ca2+] 位于被突触激活的单个脊柱头部(图 13-18A)。 此外,由于动作电位可以从细胞体反向传播到树突,因此棘也可以作为整合突触前和突触后活动信息的场所。

事实上,当反向传播动作电位与突触前刺激配对时,脊柱 Ca2+ 信号大于来自单独突触刺激或单独动作电位刺激的单个 Ca2+ 信号的线性总和。 这种“超线性”特定于激活的脊柱,并且发生是因为动作电位期间的去极化导致 Mg2+ 从 NMDA 受体通道中排出,使其能够将 Ca2+ 传导到脊柱中。 因此,由此产生的 Ca2+ 积累在单个突触处提供了输入 (EPSP) 和输出(反向传播动作电位)的近乎同时性的生化检测器,这被认为是记忆存储的关键要求(第 54 章)。

因为细的脊柱颈至少部分地限制了 Ca2+ 的升高,从而限制了接收突触输入的脊柱的长期可塑性,因此脊柱还确保突触功能的活动依赖性变化,以及记忆存储,是 仅限于被激活的突触。 脊柱实现这种特定于突触的局部学习规则的能力对于神经网络存储有意义信息的能力可能具有根本的重要性(第 54 章)。 最后,在一些棘中,局部突触电位在通过棘颈传播并进入树突时被过滤,从而减小细胞体处 EPSP 的大小。 这种电过滤的调节可以提供另一种控制给定突触电导能够激发细胞体的功效的方法。



\section{亮点}

1. 一个典型的中枢神经元整合了大量的兴奋性和抑制性突触输入。 氨基酸递质谷氨酸负责中枢神经系统中大多数兴奋性突触作用,抑制性氨基酸 GABA 和甘氨酸介导抑制性突触作用。 

2. 谷氨酸激活离子型和代谢型受体家族。 三种主要的离子型谷氨酸受体——AMPA、NMDA 和红藻氨酸——以激活它们的化学激动剂命名。 

3.离子型谷氨酸受体是由同源基因编码的亚基组成的四聚体。 每个亚基都有一个大的细胞外氨基末端,具有三个跨膜片段和一个大的细胞质尾部。 成孔环在第一和第二跨膜区段之间浸入和浸出膜。 

4. 谷氨酸与所有三种离子型受体的结合打开了一个对 Na+ 和 K+ 具有同等渗透性的非选择性阳离子通道。 NMDA 受体通道对 Ca2+ 也具有高渗透性。 

5. NMDA 受体充当符合检测器。 它通常被滞留在其孔隙中的细胞外 Mg2+ 阻断; 它仅在释放谷氨酸并且突触后膜充分去极化以通过静电排斥排出Mg2+离子时才进行。 

6. 在强烈的突触激活过程中,通过 NMDA 受体的钙流入可以触发细胞内信号级联,导致长期突触可塑性,这可以加强突触传递数小时至数天,为记忆存储提供潜在机制。 

7. 大脑中的抑制性突触作用是由 GABA 与离子型 (GABAA) 和代谢型 (GABAB) 受体的结合介导的。 GABAA 受体是五聚体,其亚基与烟碱 ACh 受体的亚基同源。 甘氨酸离子型受体在结构上类似于 GABAA 受体,主要局限于脊髓中的抑制性突触。 

8. GABA 或甘氨酸与其受体的结合会激活 Cl- 选择性通道。 在大多数细胞中,Cl- 平衡电位略低于静息电位。 结果,抑制性突触作用使细胞膜超极化,远离触发动作电位的阈值。 

9. 神经元是否激发动作电位的决定取决于各种兴奋性和抑制性输入的空间和时间总和,并由轴突起始段所产生的去极化的大小决定,神经元具有最低的区域 临界点。 

10. 树突也有电压门控通道,使它们能够在某些情况下激发局部动作电位。 这可以放大局部 EPSP 的大小,从而在细胞体处产生更大的去极化。


\section{选读}

\section{参考文献}




