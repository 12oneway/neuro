\chapter{中枢神经系统的突触整合}
% PDF所在目录: /data2/whd/win10/learn/neuro/neuro_神经科学原理_28_中枢神经系统的听觉处理.pdf

\section{中枢神经元接收兴奋性和抑制性输入}

\section{兴奋性和抑制性突触具有独特的超微结构并针对不同的神经元区域}

\section{兴奋性突触传递由可渗透阳离子的离子型谷氨酸受体通道介导}
\subsection{离子型谷氨酸受体由一个大基因家族编码}
\subsection{谷氨酸受体由一组结构模块构成}
\subsection{NMDA 和 AMPA 受体由突触后密度的蛋白质网络组织}
\subsection{NMDA 受体具有独特的生物物理和药理学特性}
\subsection{NMDA 受体的特性是长期突触可塑性的基础}
\subsection{NMDA 受体导致神经精神疾病}

\section{快速抑制性突触作用由离子型 GABA 和甘氨酸受体-可渗透氯离子的通道介导}
\subsection{离子型谷氨酸、GABA 和甘氨酸受体是由两个不同的基因家族编码的跨膜蛋白}
\subsection{通过 GABAA 和甘氨酸受体通道的氯离子电流通常会抑制突触后细胞}

\section{中枢神经系统中的一些突触动作依赖于其他类型的离子型受体}

\section{神经元将兴奋性和抑制性突触动作整合为单一输出}
\subsection{突触输入整合在轴突初始段}
\subsection{GABA 能神经元的亚类靶向其突触后靶神经元的不同区域以产生具有不同功能的抑制作用}
\subsection{树突是可以放大突触输入的电激发结构}


\section{亮点}

\section{选读}

\section{参考文献}




