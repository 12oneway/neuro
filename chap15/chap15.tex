\chapter{递质释放}
% PDF所在目录: /data2/whd/win10/learn/neuro/neuro_神经科学原理_28_中枢神经系统的听觉处理.pdf

\section{递质释放受突触前末梢去极化调控}

\section{钙离子内流激发释放}
\subsection{突触前钙浓度与释放的关系}
\subsection{几类钙通道介导递质释放}

\section{递质以量子单位释放}
\subsection{递质由突触囊泡存储和释放}
\subsection{突触小泡通过胞吐作用释放递质并通过胞吞作用回收}
\subsection{电容测量提供了对胞吐和胞吞动力学的洞察力}
\subsection{胞吐作用涉及临时融合孔的形成}
\subsection{突触小泡循环包括几个步骤}

\section{突触囊泡的胞吐仰仗高度保守的蛋白结构}
\subsection{突触蛋白对囊泡的抑制和动员很重要}
\subsection{SNARE 蛋白催化囊泡与质膜融合}
\subsection{钙与突触结合蛋白的结合触发递质释放}
\subsection{融合机器嵌入在活性区的保守蛋白质支架中}

\section{递质释放的调控是突触可塑性的基础}
\subsection{细胞内游离钙的活动依赖性变化可以在释放中产生持久的变化}
\subsection{突触前末端的轴突突触调节递质释放}

\section{亮点}

\section{选读}

\section{参考文献}

