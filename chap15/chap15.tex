\chapter{递质释放} \label{chap:chap15}
大脑的一些最显着的能力,例如学习和记忆,被认为源于化学突触的基本特性,其中突触前细胞释放化学递质,激活突触后细胞膜中的受体。 在大多数中央突触中,递质从突触前细胞的突触前细胞释放,沿着轴突的静脉曲张(像串珠一样)充满了突触小泡和其他接触突触后目标的细胞器。 在其他突触,包括神经肌肉接头处,递质从轴突末端的突触前末端释放。 为方便起见,我们将这两种类型的释放站点称为突触前终端。 在最后三章中,我们了解了突触后受体如何控制产生突触后电位的离子通道。 在这里,我们考虑突触前末梢的电和生化事件如何导致小分子神经递质的快速释放,例如乙酰胆碱 (ACh)、谷氨酸和 γ-氨基丁酸 (GABA),它们是快速突触传递的基础。 在下一章中,我们将研究神经递质本身以及生物胺(血清素、去甲肾上腺素和多巴胺)和神经肽的化学性质,它们是细胞间信号传导较慢形式的基础。

\section{递质释放受突触前末梢去极化调控}

突触前末梢的什么事件导致递质释放? Bernard Katz 和 Ricardo Miledi 首先证明了突触前膜去极化的重要性。 为此,他们使用了鱿鱼的巨大突触,这种突触大到足以让电极插入突触前和突触后结构。 两个电极插入突触前终端——一个用于刺激,一个用于记录——一个电极插入突触后细胞用于记录兴奋性突触后电位 (EPSP),它提供了递质释放的指标(图 15-1A)。

突触前神经元受到刺激并发出动作电位后,突触后细胞中会记录下足以触发动作电位的 EPSP。 Katz 和 Miledi 随后询问突触前动作电位如何触发递质释放。 他们发现,当电压门控 Na+ 通道被河豚毒素阻断时,连续动作电位会逐渐变小。 随着动作电位大小的减小,EPSP 相应减小(图 15-1B)。 当 Na+ 通道阻滞变得如此严重以至于将突触前尖峰的振幅降低到 40 mV 以下(相对于静息电位为正)时,EPSP 将完全消失。 因此,递质释放量(通过突触后去极化的大小测量)是突触前去极化量的陡峭函数(图 15-1C)。

Katz 和 Miledi 接下来研究了突触前去极化如何触发递质释放。 动作电位由 Na+ 流入和 K+ 通过电压门控通道流出产生。 为了确定触发递质释放是否需要 Na+ 流入或 K+ 流出,Katz 和 Miledi 首先用河豚毒素阻断了 Na+ 通道。 然后他们询问通过电流注入对突触前膜的直接去极化是否仍会触发递质释放。 事实上,突触前膜的去极化超过约 40 mV 的阈值(相对于静息电位为正)会在突触后细胞中引发 EPSP,即使 Na+ 通道被阻断。 超过该阈值,逐渐增加的去极化会导致逐渐增加的递质释放量。 该结果表明突触前 Na+ 流入不是释放所必需的; 它的重要性仅在于它使膜去极化足以发生发射器释放(图 15-2B)。

为了检查 K+ 流出对递质释放的贡献,Katz 和 Miledi 在用河豚毒素阻断电压敏感的 Na+ 通道的同时,用四乙铵阻断了电压门控 K+ 通道。 然后他们将去极化电流注入突触前末梢,发现 EPSP 大小正常,表明发生了正常的递质释放(图 15-2C)。 因此,递质释放不需要 Na+ 和 K+ 通量。

在存在四乙铵的情况下,电流脉冲在整个脉冲持续时间内引起突触前去极化,因为通常使突触前膜复极化的 K+ 电流被阻断。 因此,递质释放在整个电流脉冲中持续存在,这反映在突触后细胞的长期去极化中(图 15-2C)。 Katz 和 Miledi 使用持续去极化的量化来确定将突触前去极化与递质释放相关的完整输入-输出曲线(图 15-2D)。 他们证实了递质释放对突触前去极化的严重依赖。 在递质释放增加的去极化范围内(40-70 mV 正向静息水平),突触前去极化增加 10 mV 会使递质释放增加 10 倍。 超过上限的突触前膜去极化不会进一步增加突触后电位。

\section{钙离子内流激发释放}
Katz 和 Miledi 接下来将注意力转向了 Ca2+ 离子。 早些时候,Katz 和 José del Castillo 发现增加细胞外 Ca2+ 浓度会增强递质释放,而降低浓度会减少并最终阻断突触传递。 因为递质释放是一个细胞内过程,这些发现暗示 Ca2+ 必须进入细胞才能影响递质释放。

先前对鱿鱼巨型轴突膜的研究已经确定了一类电压门控 Ca2+ 通道,由于对 Ca2+ 的内向电化学驱动力大,因此打开该通道会导致大量 Ca2+ 流入。 细胞外 Ca2+ 浓度,在脊椎动物中约为 2 mM,通常比细胞内浓度高四个数量级,在静止时约为 10-7 M。 然而,由于这些 Ca2+ 通道沿轴突分布稀疏,它们本身无法提供足够的电流来产生再生动作电位。

Katz 和 Miledi 发现 Ca2+ 通道在突触前末梢更为丰富。 在那里,在四乙铵和河豚毒素存在的情况下,去极化电流脉冲有时能够触发再生去极化,这需要细胞外 Ca2+,即钙尖峰。 Katz 和 Miledi 因此提出 Ca2+ 具有双重功能。 它是动作电位期间去极化电荷的载体(如 Na+),并且是一种特殊的化学信号——第二信使——将有关膜电位变化的信息传递给负责递质释放的细胞内机制。 钙离子能够作为一种有效的化学信号,因为它们的细胞内静息浓度较低,比 Na+ 的静息浓度低约 105 倍。 因此,在动作电位期间进入或离开细胞的少量 Ca2+ 离子会导致细胞内 Ca2+ 发生较大百分比的变化,从而引发各种生化反应。 最近的实验证明了 Ca2+ 通道在释放中的重要性,这些实验表明阻断 Ca2+ 通道的特定毒素也会阻断释放。

Rodolfo Llinás 和他的同事测量了乌贼突触前末梢的电压门控 Ca2+ 通道的特性。 Llinás 使用电压钳使终端去极化,同时用河豚毒素阻断电压门控 Na+ 通道,用四乙基铵阻断 K+ 通道。 他发现分级去极化激活分级内向 Ca2+ 电流,这反过来导致递质分级释放(图 15-3)。 Ca2+ 电流分级,因为 Ca2+ 通道与电压门控 Na+ 和 K+ 通道一样具有电压依赖性。 然而,乌贼末端的钙离子通道与 Na+ 通道不同,因为它们不会迅速失活,而是在突触前去极化持续期间保持开放。

钙通道主要位于活动区的突触前末梢,神经递质释放的位置,正好与突触后受体相对(图 15-4)。 这种定位很重要,因为 Ca2+ 离子不会从其进入位点扩散很远的距离,因为游离 Ca2+ 离子会被 Ca2+ 结合蛋白快速缓冲。 结果,Ca2+ 流入导致活性区的 Ca2+ 浓度局部急剧上升。 使用 Ca2+ 敏感性荧光染料可以观察到突触前末梢 Ca2+ 的增加(图 15-4B)。 所有突触处递质释放的一个显着特征是它对 Ca2+ 流入的急剧非线性依赖性; Ca2+ 内流增加 2 倍可使递质释放量增加 16 倍以上。 这种关系表明,在某些调节位点,钙传感器,需要多个 Ca2+ 离子的协同结合才能触发释放。

\subsection{突触前钙浓度与释放的关系}
诱导神经递质释放需要多少 Ca2+? 为了解决这个问题,Bert Sakmann 和 Erwin Neher 及其同事测量了 Held 花萼中的突触传递,Held 是哺乳动物听觉脑干中的一个大突触,由从耳蜗核到梯形体内侧核的轴突组成。 这个突触专门用于非常快速和可靠的传输,以允许在环境中精确定位声音。

花萼形成一个杯状的突触前末端,吞没了一个突触后细胞体(图 15-5A)。 花萼突触包括近千个作为独立释放位点的活动区。 这使得突触前动作电位能够释放大量递质,从而导致可靠的大突触后去极化。 相比之下,大脑中典型神经元的单个突触按钮仅包含一个活动区。 因为萼末端很大,所以可以将电极插入突触前和突触后结构,就像乌贼巨型突触一样,并直接测量两个隔间之间的突触耦合。 这种配对记录可以精确确定突触前和突触后细胞活动的时间过程(图 15-5B)。

这些记录显示突触前动作电位和 EPSP 之间存在 1 到 2 毫秒的短暂滞后,这就是 Sherrington 所说的突触延迟。 因为 Ca2+ 通道比 Na+ 通道打开得更慢,并且内向 Ca2+ 驱动力随着神经元复极化而增加,所以 Ca2+ 直到膜开始复极化才开始全力进入突触前末梢。 令人惊讶的是,一旦 Ca2+ 进入终端,递质就会迅速释放,延迟只有几百微秒。 因此,突触延迟主要归因于打开 Ca2+ 通道所需的时间。 Ca2+ 作用的惊人速度表明,在 Ca2+ 流入之前,释放过程中的生化机制必须已经处于准备就绪状态。 这种快速动力学对于神经元信息处理至关重要,需要我们稍后考虑的优雅分子机制。

突触前动作电位通常只会使突触前 Ca2+ 浓度短暂升高,因为 Ca2+ 通道只打开很短的时间。 此外,Ca2+ 流入位于活性区。 这两个特性有助于形成集中的局部 Ca2+ 脉冲,从而引起递质释放的爆发(图 15-5B)。 正如我们将在本章后面看到的那样,动作电位的持续时间调节流入终端的 Ca2+ 量,从而调节递质释放量。

为了确定需要多少 Ca2+ 才能触发释放,Neher 和 Sakmann 小组将一种非活性形式的 Ca2+ 复合在光敏化学笼中,引入突触前末端。 他们还在终端上加载了 Ca2+ 敏感荧光染料,以测定细胞内游离 Ca2+ 浓度。 通过用闪光解笼 Ca2+ 离子,它们可以通过均匀且可量化的 Ca2+ 浓度增加来触发递质释放。 这些实验表明,Ca2+ 浓度升高小于 1 μM 就足以诱导某些递质的释放,但需要大约 10 至 30 μM Ca2+ 才能释放在动作电位期间通常观察到的量。 同样,Ca2+ 浓度和递质释放之间的关系是高度非线性的,与至少四个或五个 Ca2+ 离子必须与 Ca2+ 传感器结合才能触发释放的模型一致(图 15-5C,D)。

\subsection{几类钙通道介导递质释放}

钙通道存在于所有神经细胞和许多非神经元细胞中。 在骨骼肌和心肌细胞中,它们对兴奋-收缩偶联很重要; 在内分泌细胞中,它们调节激素的释放。 神经元包含五大类电压门控 Ca2+ 通道:L 型、P/Q 型、N 型、R 型和 T 型,它们由不同但密切相关的基因编码,这些基因可分为 基于氨基酸序列相似性的三个基因家族。 L 型通道由 CaV1 家族编码。 CaV2 家族的成员包括 P/Q- (CaV2.1)、N- (CaV2.2) 和 R 型 (CaV2.3) 通道。 最后,T 型通道由 CaV3 基因家族编码。 每种通道类型都具有特定的生物物理学和药理学特性以及生理功能(表 15-1)。

钙通道是多聚体蛋白质,其独特的特性由其成孔亚基 α1 亚基决定。 α1 亚基与电压门控 Na+ 通道的 α 亚基同源,由具有六个跨膜区段的结构域的四个重复组成,其中包括 S4 电压传感器和孔衬里 P 区(参见图 8 –10). 钙通道还具有辅助亚基(称为 α2、β、γ 和 δ),它们可以改变由 α1 亚基形成的通道的特性。 不同类型的钙通道在神经元中的亚细胞定位也不同。 N 型和 P/Q 型 Ca2+ 通道主要存在于突触前末梢,而 L 型、R 型和 T 型通道主要存在于胞体和树突中。

四种类型的电压门控 Ca2+ 通道——L 型、P/Q 型、N 型和 R 型——通常需要相当强的去极化才能被激活(电压正至 −40 至 −20 mV 必需),因此有时被宽泛地称为高压激活的 Ca2+ 通道(表 15-1)。 相反,T 型通道响应于产生动作电位(-60 至 -40 mV)阈值附近的小去极化而打开,因此被称为低电压激活的 Ca2+ 通道。 因为它们被膜电位的微小变化激活,所以 T 型通道有助于控制静息电位的兴奋性,并且是驱动大脑和心脏中某些细胞节律性起搏器活动的兴奋性电流的重要来源。

在神经元中,快速突触传递过程中常规递质的快速释放主要由 P/Q 型和 N 型 Ca2+ 通道介导,这些通道类型最集中在活动区。 N 型 Ca2+ 通道在青蛙神经肌肉接头处的定位已使用选择性结合这些通道的荧光标记蜗牛毒素进行可视化(见图 15-4A)。 L 型通道不存在于活性区中,因此通常不会促进传统递质(如 ACh 和谷氨酸)的快速释放。 然而,通过 L 型通道的 Ca2+ 流入对于不会发生在特定活性区的较慢释放形式很重要,例如神经元释放神经肽和内分泌细胞释放激素。 正如我们稍后将看到的,调节 Ca2+ 流入突触前末梢可控制递质释放量,从而控制突触传递的强度。

电压门控 Ca2+ 通道的突变是导致某些获得性和遗传性疾病的原因。 Timothy 综合征是一种发育障碍,其特征是严重的自闭症伴有认知功能受损和一系列其他病理生理变化,由 L 型通道的 α1 亚基突变改变其电压依赖性门控,从而影响树突状细胞引起 一体化。 P/Q 型通道 α1 亚基的不同点突变导致偏瘫性偏头痛或癫痫。 Lambert-Eaton 综合征患者是一种与肌肉无力相关的自身免疫性疾病,会产生针对 P/Q 型通道 α1 亚基的抗体,从而降低总 Ca2+ 电流(第 57 章)。

\section{递质以量子单位释放}
Ca2+的流入如何触发递质释放? Katz 和他的同事通过展示发射器以他们称为量子的离散量释放,提供了对这个问题的关键见解。 每个递质量子产生一个固定大小的突触后电位,称为量子突触电位。 总的突触后电位由大量的量子电位组成。 EPSP 似乎在振幅上平滑分级,只是因为每个量子(或单位)势相对于总势很小。

1951 年,Katz 和 Fatt 在青蛙的神经肌肉突触处观察到大约 0.5 mV 的自发突触后电位,从而获得了关于突触传递的量子性质的第一条线索。 就像神经刺激诱发的终板电位一样,这些小的去极化反应在神经肌肉接触部位最大,并随着距离的增加而衰减(见图 12-5)。 此后在哺乳动物肌肉和中枢神经元中观察到小的自发电位。 由于脊椎动物神经肌肉突触的突触后电位称为终板电位,Fatt 和 Katz 将这些自发电位称为微型终板电位。

几项结果使 Fatt 和 Katz 确信,微型终板电位代表了对少量 ACh 释放的反应,ACh 是神经肌肉突触中使用的神经递质。 微型终板电位的时间过程和各种药物对其的影响与终板电位的特性无法区分。 与终板电位一样,微型终板电位被前列斯的明增强和延长,前列斯的明是一种阻止乙酰胆碱酯酶水解 ACh 的药物。 相反,它们会被阻断 ACh 受体的药物(如箭毒)消除。 微型终板电位代表对在没有动作电位的情况下从突触前神经末梢自发释放的小包递质的反应。 它们的频率可以通过突触前末梢的小去极化来增加。 如果突触前运动神经退化,它们就会消失,并在新的运动突触形成时重新出现。

什么可以解释微型终板电位的小而固定的尺寸? Del Castillo 和 Katz 首先测试了每个事件代表对单个 ACh 受体通道开放的反应的可能性。 然而,将非常少量的 ACh 应用于青蛙肌肉终板引起的去极化突触后反应远小于微型终板电位的 0.5 mV 反应。 这一发现清楚地表明,微型终板电位代表了不止一个 ACh 受体通道的开放。 事实上,Katz 和 Miledi 后来能够估计通过单个 ACh 受体通道对基本电流的电压响应仅为大约 0.3 μV(第 12 章)。 基于这一估计,0.5 mV 的微型终板电位将代表大约 2,000 个通道的基本电流的总和。 后来的研究表明,微型终板电位是对大约 5,000 个乙酰胆碱分子同步释放的反应。

神经刺激诱发的大终板电位与小的、自发的微型终板反应有什么关系? 这个问题首先由 del Castillo 和 Katz 在研究浸泡在低 Ca2+ 溶液中的神经肌肉突触的突触信号时提出。 在这种情况下,终板电位显着降低,从正常的 70 mV 降至约 0.5 至 2.5 mV。 此外,每个连续的终板电位的振幅现在从一个刺激到下一个刺激随机变化; 通常,根本无法检测到任何响应(称为故障)。 然而,零以上的最小响应——响应突触前动作电位的单位终板电位——在振幅(大约 0.5 mV)和形状上与自发的微型终板电位相同。 重要的是,每个终板电位的振幅是单位电位的整数倍(图 15-6)。

现在 del Castillo 和 Katz 可能会问:伴随每个动作电位的细胞内 Ca2+ 的升高如何影响递质的释放? 他们发现增加外部 Ca2+ 浓度不会改变单位突触电位的振幅。 然而,失败的比例降低,更高振幅响应(由多个量子单元组成)的发生率增加。 这些观察结果表明,外部 Ca2+ 浓度的增加不会增加递质量子的大小(即每个量子中 ACh 分子的数量),而是会增加响应于 突触前动作电位。 流入终端的 Ca2+ 越多,释放的递质量子数就越多。

因此,三个发现导致 del Castillo 和 Katz 得出结论,递质是在具有固定量递质的数据包中释放的,一个量子: 每一步增加都是单位电位的整数倍,单位电位与自发微型终板电位具有相同的平均振幅和形状。 此外,通过分析终板电位振幅的统计分布,del Castillo 和 Katz 以及其他后来的研究人员能够表明,单个动作电位会导致给定量子递质释放概率的瞬态增加 过程,类似于决定抛硬币结果的过程(方框 15-1)。

在没有动作电位的情况下,量子释放率很低——每秒只有一个量子在终板自发释放。 相比之下,动作电位的发射释放大约 150 个量子,每个量子的振幅大约为 0.5 mV,从而产生大的终板电位。 因此,在动作电位期间,Ca2+ 流入突触前末梢使量子释放速率显着增加 150,000 倍,从而在约 1 毫秒内触发约 150 个量子的同步释放。


\subsection{递质由突触囊泡存储和释放}
细胞的哪些形态特征可能会导致发射体的量子释放? 表明发射器以固定量子释放的生理观察与通过电子显微镜发现的小透明囊泡在突触前末端的积累相吻合。 Del Castillo 和 Katz 推测小泡是储存递质的细胞器,每个小泡储存一个量的递质(相当于数千个分子),每个小泡以全有或全无的方式将其全部内容物释放到突触间隙中 在专门用于发布的站点。

释放位点,即活性区,包含一团突触小泡,这些小泡聚集在附着在突触前膜内表面的模糊电子致密材料上方(见图 15-4A)。 在所有突触中,囊泡通常是透明的、小的、卵形的,直径约为 40 nm(与第 16 章中描述的大的致密核囊泡不同)。 虽然大多数突触小泡不接触活动区,但有些是物理结合的。 这些被称为停靠囊泡,被认为是可立即释放的囊泡(有时称为易释放池)。 在神经肌肉接头处,活动区是线性结构(见图 15-4),而在中央突触中,它们是圆盘状结构,面积约为 0.1 μm2,具有指向细胞质的密集突起。 活动区通常与包含神经递质受体的突触后膜片精确并列(见图 13-2)。 因此,突触前和突触后特化在功能和形态上相互协调,有时在结构“纳米柱”中精确对齐。 正如我们稍后将了解到的,现在已经鉴定和表征了涉及递质释放的几种关键活性区蛋白。

到目前为止,已在所有化学突触中证明了量子传输。 然而,递质从单个突触前细胞释放到单个突触后细胞的功效在神经系统中变化很大,取决于几个因素:(1) 一对突触前细胞和突触后细胞之间的个体突触数量(即, 接触突触后细胞的突触前神经节数); (2) 单个突触终端的活动区数; (3) 突触前动作电位触发激活区释放一个或多个递质量子的概率。 正如我们稍后将看到的,释放概率可以作为神经元活动的函数进行强有力的调节。

在中枢神经系统中,大多数突触前神经节只有一个活动区,其中动作电位通常以全有或全无的方式最多释放一个递质量。 然而,在一些中央突触处,例如 Held 的花萼,递质从可能包含许多活动区域的大型突触前终端释放,因此可以释放大量量子以响应单个突触前动作电位。 典型的突触前细胞与典型的突触后细胞形成的突触数量也不同,中枢神经元也不同。 大多数中枢神经元仅与任何一个突触后细胞形成少数突触,而来自下橄榄中神经元的单个攀爬纤维在小脑中的单个浦肯野神经元上形成多达 10,000 个末端! 最后,从单个活动区释放递质的平均概率在突触前末梢也有很大差异,从小于 0.1(即突触前动作电位触发囊泡释放的概率为 10\%)到大于 0.9。 这种广泛的概率甚至可以在特定类型的突触前细胞和特定类型的突触后细胞之间的单个突触的 boutons 中看到。

因此,中枢神经元在突触传递的功效和可靠性方面差异很大。 突触可靠性被定义为突触前细胞中的动作电位在突触后细胞中导致某种可测量反应的概率——即突触前动作电位将释放一个或多个递质量子的概率。 功效是指突触反应的平均幅度,它取决于突触传递的可靠性和突触传递确实发生时反应的平均大小。

大多数中枢神经元在递质释放概率较低的突触处进行通信。 大多数中央突触的高释放失败率(即它们的低释放概率)不是设计缺陷,而是有目的的。 正如我们稍后讨论的那样,此功能允许在较宽的动态范围内调节发射器释放,这对于使神经信号适应不同的行为需求非常重要。 在释放概率低对功能有害的突触连接中,可以通过简单地在一个突触中设置多个活动区来克服这种限制,就像 Held 的花萼和神经肌肉突触的情况一样。 两者都包含数百个独立的活动区,因此一个动作电位可靠地释放 150 到 250 个量子,确保突触前信号始终跟随突触后动作电位。 神经肌肉接头处的可靠传输对于生存至关重要。 如果动物远离捕食者的能力受到低概率反应的阻碍,那么动物将无法生存。 另一个提高可靠性的策略是使用多泡释放,多个泡在单个活性区同时融合,以确保突触后受体始终暴露在饱和浓度的神经递质中(见方框 15-1)。

并非神经元之间的所有化学信号都依赖于前面描述的突触机制。 某些物质,例如某些脂质代谢物和气体一氧化氮(第 14 章),可以扩散穿过膜的脂质双层。 如果细胞内浓度足够高,其他细胞可以被载体蛋白移出神经末梢。 谷氨酸或 GABA 的质膜转运体通常在突触前动作电位后从突触间隙将递质带入细胞(第 13 章)。 然而,在视网膜的一些神经胶质细胞中,谷氨酸转运的方向在一定条件下可以逆转,导致谷氨酸通过转运蛋白离开细胞进入突触间隙。 还有一些物质只是以低速率从神经末梢漏出。 令人惊讶的是,大约 90\% 的离开神经肌肉接头处突触前末梢的 ACh 是通过持续泄漏来完成的。 然而,这种泄漏是无效的,因为它是弥散的并且不针对终板区域的受体,并且因为它是连续的和低水平的而不是同步的和集中的。


\subsection{突触小泡通过胞吐作用释放递质并通过胞吞作用回收}
del Castillo 和 Katz 的量子假设已被直接实验证据充分证实,即突触小泡确实包装神经递质,并且它们通过直接与突触前膜融合释放其内容物,这一过程称为胞吐作用。

四十年前,Victor Whittaker 发现电鱼 Torpedo 的电器官运动神经末梢的突触小泡中含有高浓度的 ACh。 后来,Thomas Reese 和 John Heuser 及其同事获得了电子显微照片,这些照片捕捉到了囊泡的胞吐作用。 为了观察短暂的胞吐事件,他们在突触前神经受到刺激后,将神经肌肉突触以精确定义的时间间隔浸入液氦中,从而迅速冻结神经肌肉突触。 此外,他们通过使用药物 4-氨基吡啶(一种阻断某些电压门控 K+ 通道的化合物)增加每次神经冲动释放的递质量子数,从而增加动作电位的持续时间并增强 Ca2+ 流入。 (这种药物干预产生的尖峰增宽类似于重复放电过程中 K+ 通道的累积失活导致的尖峰增宽;参见图 15-15C。)在这两种情况下,延长的动作电位都会引起突触前 Ca2+ 通道的更大开放。

这些技术提供了胞吐过程中活动区突触小泡的清晰图像。 使用一种称为冷冻断裂电子显微镜的技术,Reese 和 Heuser 注意到突触活动后突触前膜沿活动区立即发生变形,他们将其解释为突触小泡融合引起的细胞膜内陷。 这些变形沿着一排或两排异常大的膜内颗粒分布,沿着突触前密度的两个边缘可见。 许多这些颗粒现在被认为是电压门控 Ca2+ 通道(图 15-7)。 颗粒密度(约 1,500 个/μm2)与被认为存在于活性区突触前质膜中的 Ca2+ 通道密度相似。 此外,粒子与释放位点的接近程度与 Ca2+ 电流开始与递质释放之间的短时间间隔一致。

最后,Heuser 和 Reese 发现这些变形是短暂的; 它们仅在囊泡排出时发生,并且在递质释放后不会持续存在。 薄截面电子显微照片显示许多 omega 形 (Ω) 结构,在囊泡膜完全塌陷到质膜之前,突触囊泡的外观刚刚与膜融合(图 15-7B)。 Heuser 和 Reese 通过证明当他们改变 4-氨基吡啶的浓度以改变递质释放量时,Ω 形结构的数量与 EPSP 的大小直接相关,从而证实了这一想法。 这些形态学研究提供了显着的证据,证明递质是通过胞吐作用从突触小泡中释放出来的。

在胞吐作用之后,回收添加到突触前末端的多余膜。 在刺激后 10 到 20 秒的突触前末梢图像中,Heuser 和 Reese 观察到质膜上的新结构,即涂层凹坑,这些凹坑由蛋白质网格蛋白形成,有助于通过内吞作用介导膜恢复(图 15-7C) ). 几秒钟后,可以看到涂层凹坑从膜上夹断,并在细胞质中显示为涂层囊泡。 正如我们稍后将看到的,通过包被坑形成的内吞作用代表了囊泡膜修复的几种方法之一。

\subsection{电容测量提供了对胞吐和胞吞动力学的洞察力}
在具有大突触前末端的某些神经元中,胞吐过程中质膜表面积的增加可以在电测量中检测为膜电容的增加。 正如我们在第 9 章中看到的,膜的电容与其表面积成正比。 Erwin Neher 发现可以使用电容测量来监测分泌细胞的胞吐作用。

在肾上腺嗜铬细胞(释放肾上腺素和去甲肾上腺素)和大鼠腹膜的肥大细胞(释放组胺和血清素)中,单个致密核心囊泡足够大,可以测量与单个囊泡融合相关的电容增加 . 这些细胞中发射器的释放伴随着电容的逐步增加,随后是逐步减少,这反映了多余膜的回收和再循环(图 15-8)。

在神经元中,由单个小突触囊泡融合引起的电容变化通常太小而无法分辨。 在某些释放大量囊泡的有利突触制剂中(例如视网膜中双极神经元的巨大突触前末梢),由于胞吐作用和恢复,膜去极化触发了末端总电容的短暂平滑上升和下降 来自数百个单独的突触小泡的膜(图 15-8C)。 这些结果提供了膜融合和修复率的直接测量。

\subsection{胞吐作用涉及临时融合孔的形成}
使用快速冷冻对肥大细胞的形态学研究表明,胞吐作用取决于跨越囊泡膜和质膜的临时融合孔的形成。 在肥大细胞电容增加的电生理学研究中,在囊泡和细胞膜完全融合之前,在电生理学记录中检测到通道状融合孔。 该融合孔以大约 200 pS 的单通道电导开始,类似于间隙连接通道,它也桥接两个膜。 在胞吐过程中,孔迅速扩张,直径可能从 5 纳米到 50 纳米左右,电导急剧增加(图 15-9A)。

融合孔不仅仅是导致递质胞吐的中间结构,因为递质可以在孔扩张和囊泡塌陷之前通过孔释放。 这首先通过安培法显示,这种方法使用细胞外碳纤维电极检测某些胺类神经递质,例如血清素,基于发射器和电极之间的电化学反应,产生与局部发射器浓度成比例的电流。 5-羟色胺能细胞中动作电位的激发导致电极电流的大量瞬时增加,对应于单个致密核囊泡内容物的胞吐作用。 在某些情况下,这些大的瞬态增加之前是较小、持续时间更长的电流信号,这些信号反映了发射器通过融合孔的泄漏,融合孔在完全融合之前闪烁打开和关闭几次(图 15–9B)。

递质也可能仅通过短暂的融合孔释放,这些融合孔短暂地连接囊泡腔和细胞外空间,而囊泡膜不会完全塌陷到质膜中。 神经内分泌细胞中大致密核囊泡胞吐作用的电容测量表明,融合孔可以快速且可逆地打开和关闭。 融合孔的可逆打开和关闭代表了一种非常快速的膜回收方法。 与全膜塌陷相反,快速突触处的小透明囊泡通过融合孔排出递质的情况尚不确定。

\subsection{突触小泡循环包括几个步骤}

当以高频发射时,典型的突触前神经元能够保持发射器释放的高速率。 随着时间的推移,这可能导致大量囊泡的胞吐作用,超过突触前末端形态学上明显的数量。 为了防止囊泡的供应在快速突触传递过程中迅速耗尽,使用过的囊泡会被迅速回收和回收。 因为神经末梢通常与细胞体有一段距离,所以通过细胞体中的合成来补充囊泡并运输到末端会太慢而不适用于快速突触。

突触小泡在一个简单的循环中被释放和重复使用。 囊泡充满神经递质并聚集在神经末梢。 然后它们停靠在活性区,在那里它们经历复杂的启动过程,使囊泡能够响应触发融合过程的 Ca2+ 信号(图 15-10A)。 存在许多机制用于在胞吐作用后恢复突触小泡膜,每个机制都有不同的时间过程(图 15-10B)。

第一个也是最快速的机制涉及融合孔的可逆打开和关闭,而囊泡膜与质膜没有完全融合。 在亲吻和停留途径中,融合孔关闭后囊泡保留在活性区,为第二次释放事件做好准备。 在 kiss-andrun 通路中,囊泡在融合孔关闭后离开活性区,但能够快速再释放。 这些通路被认为在低频刺激期间优先使用。

Jorgensen 及其同事描述了超快网格蛋白非依赖性内吞作用的第二条通路,该通路比经典的网格蛋白介导通路快 200 倍。 在胞吐作用后仅 50 毫秒开始,超快胞吞作用就发生在活性区外。

较高频率的刺激会招募第三条较慢的回收途径,该途径使用网格蛋白在与质膜融合后回收囊泡膜。 网格蛋白在内吞作用期间形成围绕膜的晶格状结构,从而在 Heuser 和 Reese 观察到的涂层凹坑周围出现涂层。 在此途径中,回收的囊泡膜必须通过内体隔室再循环,然后才能重复使用囊泡。 网格蛋白介导的循环需要最多一分钟才能完成,并且似乎也从活性区转移到活性区周围的膜上(见图 15-7)。 第四种机制在长时间的高频刺激后起作用。 在这些条件下,可以看到大量膜内陷进入突触前末端,这被认为反映了膜通过称为批量回收的过程进行再循环。

\section{突触囊泡的胞吐仰仗高度保守的蛋白结构}
突触小泡的许多关键蛋白及其在质膜中的相互作用伙伴已被分离和纯化。 分离的突触小泡的蛋白质组学分析提供了它们包含的多种蛋白质的普查(图 15-11)。 两种最丰富的蛋白质,synaptobrevin 和 synaptotagmin-1,参与囊泡融合,稍后讨论。 另一类重要的囊泡蛋白是神经递质转运蛋白(第 16 章)。 这些跨膜蛋白(以谷氨酸转运蛋白 v-GluT 为例)利用存储在电化学梯度中的能量使质子从细胞质中逆浓度梯度泵送递质分子进入囊泡。 质子动力由囊泡 H+ 泵 V-ATPase 产生,它将质子从细胞质泵入囊泡腔,导致囊泡 pH 值约为 5.0 的酸性。

其他突触小泡蛋白将小泡引导至它们的释放位点,通过胞吐作用参与递质的释放,并介导小泡膜的再循环。 这三个步骤中涉及的蛋白质机制在从蠕虫到人类的整个进化过程中都得到了保护,并构成了神经递质调节释放的基础。 我们依次考虑这些步骤中的每一个。


\subsection{突触蛋白对囊泡的抑制和动员很重要}
活性区外的囊泡代表递质储备池。 Paul Greengard 发现了一个蛋白质家族,即突触蛋白,它们被认为是囊泡储备池的重要调节剂。 突触蛋白是与突触小泡的细胞质表面结合的外周膜蛋白。 突触蛋白包含一个保守的中央 ATPase 结构域,占其结构的大部分,但其功能仍然未知。 此外,synapsin-1 结合肌动蛋白。

突触蛋白是蛋白激酶 A 和 Ca2+/钙调蛋白依赖性蛋白激酶 II 的底物。 当神经末梢去极化并且 Ca2+ 进入时,突触蛋白被激酶磷酸化,从而从囊泡中释放出来。 引人注目的是,刺激突触蛋白磷酸化、突触蛋白基因缺失或细胞内注射突触蛋白抗体会导致神经末梢突触小泡数量减少,从而导致神经末梢保持高速率递质释放的能力下降 在重复刺激期间。

\subsection{SNARE 蛋白催化囊泡与质膜融合}
因为膜双层是稳定的结构,突触小泡和质膜的融合必须克服大的不利活化能。 这是通过现在称为 SNARE(可溶性 N-乙基马来酰亚胺-敏感因子附着受体)的融合蛋白家族实现的(图 15-12)。

从酵母到人类,SNARE 普遍参与膜融合。 它们介导蛋白质从内质网到高尔基体再到质膜的运动过程中的组成型膜运输,以及对调节胞吐作用很重要的突触小泡运输。 SNARE 具有一个保守的蛋白质序列,即 SNARE 基序,其长度为 60 个残基。 它们有两种形式。 囊泡陷阱或 v-陷阱(也称为 R-陷阱,因为它们含有重要的中央精氨酸残基)位于囊泡膜中。 靶膜陷阱或 t-SNARE(也称为 Q-SNARE,因为它们含有重要的谷氨酰胺残基)存在于靶膜中,例如质膜。

每个突触小泡都包含一个称为 synaptobrevin(也称为小泡相关膜蛋白或 VAMP)的 v-SNARE。 相比之下,突触前活动区包含两种类型的 t-SNARE 蛋白,syntaxin 和 SNAP-25。 (Synaptobrevin 和 syntaxin 有一个 SNARE 基序;SNAP-25 有两个。)synaptobrevin、syntaxin 和 SNAP-25 都参与突触小泡与质膜融合的第一个线索来自以下发现:所有三种蛋白质都是 肉毒杆菌毒素和破伤风毒素的底物,细菌蛋白酶是递质释放的有效抑制剂。 James Rothman 随后提供了重要的见解,即这三种蛋白质在紧密的生化复合物中相互作用。 在溶液中使用纯化的 v-SNARE 和 t-SNARE 的实验中,四个 SNARE 基序彼此紧密结合形成 α-螺旋卷曲螺旋复合体(图 15-12B)。

SNARE 复合体的形成如何驱动突触小泡融合? 在胞吐过程中,突触小泡上 synaptobrevin 的 SNARE 基序与质膜上的 SNAP-25 和 syntaxin 的 SNARE 基序形成紧密的复合体(图 15-12B)。 SNARE 复合物的晶体结构表明该复合物将膜聚集在一起。 synaptobrevin、syntaxin 和 SNAP-25 的三元复合物非常稳定。 其组装中释放的能量被认为会吸引囊泡和质膜的带负电荷的磷脂紧密并置,迫使它们进入融合前的中间状态(图 15-12)。 这种不稳定状态可能会开始形成融合孔,并产生在电生理测量中观察到的融合孔的快速打开和关闭(闪烁)。

然而,SNARE 并不能完全解释突触小泡和质膜的融合。 脂质囊泡中纯化蛋白的重组实验表明突触小蛋白、syntaxin 和 SNAP-25 可以催化融合,但体外反应显示 Ca2+ 几乎没有调节作用,并且反应比真实突触中的囊泡融合慢得多,效率也低得多。 突触小泡胞吐所需的一种重要的额外蛋白质是 Munc18(哺乳动物 unc18 同系物)。 Munc18 的同系物,称为 SM 蛋白(sec1/Munc18 样蛋白),对于所有 SNARE 介导的细胞内融合反应都是必不可少的。 Munc18 在 SNARE 复合体组装之前与突触融合蛋白结合。 删除 Munc18 会阻止神经元中的所有突触融合。 因此,核心融合机制由 SNARE 和 SM 蛋白组成,这些蛋白由特定融合反应特异的各种辅助因子调节。 最后,突触 SNARE 复合物还与一种称为复合蛋白的小可溶性蛋白相互作用,后者抑制递质的自发释放,但增强 Ca2+ 依赖性诱发释放。

融合后,必须拆解 SNARE 复合体才能进行有效的囊泡回收。 Rothman 发现一种称为 NSF(N-乙基马来酰亚胺敏感融合蛋白)的细胞质 ATP 酶通过一种称为 SNAP(可溶性 NSF 附着蛋白,与 SNARE 蛋白 SNAP-25 无关)的衔接蛋白与 SNARE 复合物结合。 NSF 和 SNAP 使用 ATP 水解的能量来解离 SNARE 复合物,从而再生游离 SNARE(图 15-12A)。 SNARE 和 NSF 还参与树突棘中突触后 AMPA 型谷氨酸受体的循环。

\subsection{钙与突触结合蛋白的结合触发递质释放}
因为突触小泡与质膜的融合必须在几分之一毫秒内发生,所以人们认为大多数负责融合的蛋白质在 Ca2+ 流入之前组装。 根据这一观点,一旦 Ca2+ 进入突触前末端,它就会与囊泡上的 Ca2+ 传感器结合,立即触发膜融合。

一个密切相关的蛋白质家族的成员,即突触结合蛋白,已被确定为触发突触小泡融合的主要 Ca2+ 传感器。 突触结合蛋白是具有单个 N 末端跨膜区的膜蛋白,可将它们固定在突触小泡上(图 15–13A、B)。 每个突触结合蛋白的细胞质区域主要由两个结构域组成,即 C2 结构域,这是与蛋白激酶 C 的 Ca2+ 和磷脂结合 C2 结构域同源的常见蛋白质基序。发现 C2 结构域不仅结合 Ca2+,而且结合 磷脂也与其在 Ca2+ 依赖性胞吐作用中的重要性一致。 Synaptotagmin-1、-2 和 -9 已被确定为用于快速同步囊泡融合的 Ca2+ 传感器。 每个都表现出不同的 Ca2+ 结合亲和力和动力学,根据表达的特定突触结合蛋白亚型赋予不同的突触不同的释放特性。 相比之下,synaptotagmin-7 介导一种较慢形式的 Ca2+ 触发的胞吐作用,这种胞吐作用对于重复触发动作电位的长时间活动期间的突触传递很重要。 所有这些突触结合蛋白还在其他形式的胞吐作用中充当 Ca2+ 传感器,例如内分泌细胞中的胞吐作用和 AMPA 型谷氨酸受体在 NMDA 受体依赖性长期增强期间从细胞内囊泡池插入突触后细胞膜。

对突触结合蛋白-1 被删除或通过基因工程改变其 Ca2+ 亲和力的突变小鼠的研究提供了重要证据,证明突触结合蛋白是生理 Ca2+ 传感器。 当突触结合蛋白对 Ca2+ 的亲和力降低两倍时,递质释放所需的 Ca2+ 会发生相同数量的变化。 当小鼠、果蝇或蠕虫中的 synaptotagmin-1 被删除时,动作电位不再能够触发快速同步释放。 然而,Ca2+ 仍然能够刺激一种较慢形式的递质释放,称为异步释放(图 15-13A),由 synaptotagmin-7 介导。 因此,几乎所有 Ca2+ 触发的神经递质释放都依赖于突触结合蛋白。

Ca2+ 与突触结合蛋白的结合如何触发囊泡融合? 两个 C2 结构域结合总共五个 Ca2+ 离子,与触发发射体量子释放所需的最少 Ca2+ 离子数量相同(图 15–13B)。 然而,由于多个突触结合蛋白可能参与触发释放,超过五个结合的 Ca2+ 离子可能分布在单个囊泡上的多个突触结合蛋白分子中。

Ca2+ 离子与突触结合蛋白的结合被认为充当开关,促进 C2 结构域与磷脂的相互作用。 突触结合蛋白的 C2 结构域也与 SNARE 蛋白和复合蛋白相互作用。 突触结合蛋白的晶体结构揭示了与相关 SNARE 复合体的保守主界面。 此外,第二个突触结合蛋白分子与相同的 SNARE 复合物和络合物形成三方相互作用。 Brunger 及其同事发现,主要的 SNARE 复合物/突触结合蛋白界面和三方 SNARE 复合物/突触结合蛋白/复合蛋白界面对于快速 Ca2+ 触发的融合都是必不可少的。 这些发现导致了以下假设:(1) 在静止时,启动囊泡的突触结合蛋白存在于具有部分预拉链的 SNARE 蛋白和络合物的复合物中; (2) 在动作电位触发的 Ca2+ 流入后,Ca2+ 与突触结合蛋白结合。 这触发了突触结合蛋白和质膜之间的相互作用,导致复合物整体旋转,从而诱导复合物部分从 SNARE 复合物上解离; (3) 这种旋转导致质膜凹陷,其面向细胞质的脂质重新排列,并最终导致质膜和囊泡膜融合(图 15-13C)。 通过这种方式,可以利用突触结合蛋白、Ca2+ 和膜的有利相互作用的能量来解除络合物介导的融合锁定并促进囊泡膜与质膜的能量不利合并。

\subsection{融合机器嵌入在活性区的保守蛋白质支架中}

正如我们所见,快速突触传递的一个决定性特征是神经递质在活动区通过胞吐作用释放。 其他类型的胞吐作用,例如发生在肾上腺髓质中的胞吐作用,不需要质膜的专门区域。 活性区被认为可以协调和调节突触小泡的对接和启动,以实现释放的速度和严格调节。 这是通过一组保守的蛋白质来实现的,这些蛋白质在活性区形成一个大的大分子结构。

杰克·麦克马汉 (Jack MacMahan) 使用一种称为电子显微断层扫描的强大超微结构技术,获得了青蛙神经肌肉接头处活动区的精美细节视图。 该技术展示了突触小泡如何通过一系列独特的结构实体(称为肋和梁)束缚在膜上,这些结构实体附着在小泡上的特定位置和突触前膜中的颗粒(钉子)上,这些颗粒可能对应于电压门控 Ca2+ 通道(图 15–14)。

了解各种突触小泡和活性区蛋白在胞吐过程中如何协调的一个关键目标是将已识别的各种蛋白质与该电子显微结构的元素相匹配。 已经鉴定出几种细胞质蛋白,它们被认为是活性区结构基质的组分。 这些包括三个大的细胞质多结构域蛋白,Munc13(与前面讨论的 Munc18 蛋白无关)、RIM 和 RIM 结合蛋白 (RIM-BPs),它们彼此形成紧密的复合体,可能构成肋和梁的一部分 . 突触小泡与 RIM 和 Munc13 的结合对于启动小泡进行胞吐作用至关重要。 cAMP 依赖性蛋白激酶对 RIM 的磷酸化与增强与可能有助于学习和记忆的某些形式的长期突触可塑性相关的递质释放有关。 正如我们稍后将看到的,第二信使对 Munc13 的调节涉及突触可塑性的短期形式。

RIM 结合突触小泡蛋白 Rab3 和 Rab27,它们是低分子量鸟苷三磷酸酶 (GTPases) 家族的成员。 Rab3 和 Rab27 蛋白作为 GTP 结合的 Rab3 复合物与突触小泡瞬时结合(图 15-11B)。 RIM 与 Rab3 或 Rab27 的结合被认为在 SNARE 复合体组装之前的囊泡周期中将突触囊泡束缚在活性区。 此外,RIM 和 RIM-BP 共同调节 Ca2+ 通道向活性区的募集,从而使 Ca2+ 流入与囊泡释放紧密耦合。 这种通用机制在进化过程中得以保存,并且存在于无脊椎动物中,尽管有所修改。


\section{递质释放的调控是突触可塑性的基础}
化学突触的有效性可以被显着和快速地调节——在几秒钟内调节几倍——而且这种变化可以维持几秒钟、几小时、甚至几天或更长时间,这种特性称为突触可塑性。

突触强度可以通过改变神经递质的释放在突触前进行修改,也可以在突触后通过调节对递质的反应(如第 13 章所述)进行修改,或两者兼而有之。 突触前和突触后机制的长期变化对于发育过程中突触连接的完善(第 49 章)以及学习和记忆过程中的信息存储(第 53 和 54 章)至关重要。 在这里,我们关注如何通过调制释放的递质数量来改变突触强度。 原则上,递质释放的变化可以通过两种不同的机制介导:Ca2+ 流入的变化或响应给定 Ca2+ 浓度而释放的递质数量的变化。 正如我们稍后将看到的,两种类型的机制都有助于不同形式的可塑性。

突触强度通常会因突触前神经元的活动模式而改变。 动作电位序列在一些突触处产生连续较大的突触后电流,在其他突触处产生连续较小的电流(图 15-15A)。 对重复刺激的突触后反应大小的减小称为突触抑制(图 15-15A,上图); 相反,通过重复刺激增强传递,称为突触促进或增强(图 15-15A,下图,15-15E)。 各种突触表现出这些不同形式的短期突触可塑性——有时重叠,有时以一种为主——导致个别突触类型的短期动力学特征模式(图 15-15A)。

突触是促进还是抑制通常取决于响应火车的第一动作电位的释放概率。 具有初始高释放概率的突触通常会受到抑制,因为高释放率会暂时耗尽活动区的对接囊泡。 初始释放概率较低的突触会经历突触促进,部分原因是细胞内 Ca2+ 在训练过程中的积累增加了释放的概率(见下文)。 从基因突变的影响可以看出释放概率在控制可塑性符号中的重要性。 由细胞培养物中的海马神经元形成的突触最初具有较高的释放概率,因此通常会响应 20 Hz 的刺激而抑制。 然而,将 synaptotagmin-1 的 Ca2+ 结合亲和力降低大约两倍的突变,从而降低了释放的初始概率,将抑制突触转化为促进突触(图 15-15B)。

影响突触前末端游离 Ca2+ 浓度的机制也会影响释放的递质数量。 例如,高频放电期间某些电压门控 K+ 通道失活的累积导致动作电位持续时间逐渐增加。 动作电位的延长会增加电压门控 Ca2+ 通道保持开放的时间,从而导致 Ca2+ 的进入增强和随后递质释放的增加,从而导致更大的突触后电位(图 15-15C)。

大多数关于短期突触动力学的功能影响的研究都是在体外进行的或基于计算结果。 然而,最近的体内实验开始揭示短期可塑性的行为重要性。 例如,来自丘脑皮质突触的啮齿类动物的体内记录表明,突触抑制可能有助于反复刺激胡须期间的感觉适应。 这种感觉适应的时间过程与皮质尖峰信号对晶须刺激的减弱和丘脑皮质突触中 EPSP 的突触抑制平行(图 15-15D)。

突触前神经元的高频刺激在某些细胞中每秒可产生高达 500 至 1,000 个动作电位,称为强直性刺激。 如此强烈的刺激会引起突触强度的显着变化。 强直刺激期间 EPSP 大小的增加称为增强; 强直刺激后持续增加称为强直后增强(图 15-15E)。 与持续几毫秒到几秒的突触促进相反,强直后增强通常持续几分钟,但它可以在某些突触处持续一个小时或更长时间。

突触利用包含 Munc13 和 RIM(前面讨论的两种活性区蛋白)的复合物来抵消高频刺激期间的囊泡耗竭。 破伤风刺激期间突触前 Ca2+ 的升高会激活磷脂酶 C,从而产生肌醇 1,4,5-三磷酸 (IP3) 和甘油二酯。 二酰甘油直接与 Munc13 上称为 C1 结构域的蛋白质结构域相互作用(与蛋白激酶 C 中的二酰甘油结合结构域同源,但不同于突触结合蛋白的 C2 结构域),从而加速突触小泡的再循环速率。 同时,IP3 导致细胞内储存的 Ca2+ 额外释放,Ca2+ 的增加通过与其 C2 结构域结合进一步激活 Munc13,该结构域类似于突触结合蛋白的 C2 结构域,但充当短期突触可塑性的代理。

\subsection{细胞内游离钙的活动依赖性变化可以在释放中产生持久的变化}
几种 Ca2+ 依赖机制导致递质释放发生更持久的变化,这种变化在高频破伤风终止后持续存在。 通常情况下,突触前末梢 Ca2+ 响应动作电位而升高,会被细胞质 Ca2+ 结合蛋白和线粒体迅速缓冲。 钙离子也通过泵和转运蛋白主动从神经元中转出。 然而,在强直刺激期间,太多的 Ca2+ 流入轴突末端,以至于 Ca2+ 缓冲和清除系统可能变得饱和。

这会导致暂时过量的 Ca2+,称为残留 Ca2+。 残留的游离 Ca2+ 通过激活某些对静息 Ca2+ 水平升高敏感的酶(包括 Ca2+/钙调蛋白依赖性蛋白激酶)增强突触传递,持续数分钟或更长时间。 这种 Ca2+ 依赖性酶通路的激活被认为会增加末端突触小泡的启动。 那么,这就是最简单的细胞记忆! 突触前细胞可以以其末端残留的游离 Ca2+(或与传感器蛋白结合的残留 Ca2+)的形式存储有关其活动历史的信息。

这种 Ca2+ 通过具有不同半衰期的多种途径发挥作用。 在第 13 章中,我们看到了某些突触的强直后增强是如何伴随更持久的过程(也由 Ca2+ 流入启动),称为长时程增强,可以持续数小时甚至数天。 第 53 章和第 54 章将讨论长时程增强对学习和记忆的重要性。

\subsection{突触前末端的轴突突触调节递质释放}
突触在轴突末端以及神经元的细胞体和树突上形成(见第 13 章)。 尽管轴体突触作用影响突触后神经元轴突的所有分支(因为它们影响神经元激发动作电位的概率),但轴突作用选择性地控制轴突的各个末端。 轴突-轴突突触的一项重要作用是增加或减少 Ca2+ 流入突触后细胞的突触前末端,从而分别增强或抑制递质释放。

正如我们在第 13 章中看到的,当一个神经元释放使另一个细胞体(或树突)超极化的递质时,它会降低突触后细胞放电的可能性; 这种作用称为突触后抑制。 相反,当一个神经元在另一个细胞的轴突末端形成突触时,它可以减少突触后细胞释放到第三个细胞的递质数量; 这种行为称为突触前抑制(图 15-16A)。 其他轴突-轴突突触动作可以增加突触后细胞释放的递质数量; 此动作称为突触前促进(图 15-16B)。 突触前抑制和促进都可以响应突触前末梢膜中离子型或代谢型受体的激活而发生。


在无脊椎动物神经元和脊椎动物机械感受器神经元(其轴突投射到脊髓中的神经元)中发现了突触前抑制和促进的最佳分析机制。 已在这些细胞中鉴定出三种突触前抑制机制。 一种取决于抑制性中间神经元的激活,这些中间神经元在感觉神经元突触前末梢形成轴突-轴突突触,在那里它们激活离子型 GABAA 受体通道。 由于突触前末端的 Cl- 反转电位相对正,因此由 GABAA 通道激活引起的 Cl- 电导增加使突触前末端去极化。 这种电压变化,称为初级传入去极化,被认为会使电压门控 Na+ 通道失活,降低突触前动作电位的幅度,从而减少电压门控 Ca2+ 通道的激活,从而减少递质释放量。

突触前抑制的其他两种机制均由突触前 G 蛋白偶联代谢受体的激活引起。 一种类型的作用源于离子通道的调节。 正如我们在第 14 章中看到的,G 蛋白的 βγ-亚基复合体可以同时关闭电压门控 Ca2+ 通道和打开 K+ 通道。 这减少了 Ca2+ 的流入并增强了动作电位后突触前末梢的复极化,从而减少了递质释放。 第二种类型的 G 蛋白依赖性作用取决于 βγ-亚基复合物对释放机制本身的直接作用,与离子通道活性或 Ca2+ 流入的任何变化无关。 这第二个作用被认为涉及降低释放机制的 Ca2+ 敏感性。

相反,突触前易化可由 Ca2+ 流入增加引起。 在某些软体动物神经元中,5-羟色胺通过 cAMP 依赖性蛋白磷酸化作用关闭突触前末梢的 K+ 通道(包括第 14 章讨论的海兔 S 型 K+ 通道)。 此操作增加了突触前动作电位的持续时间,从而通过使电压依赖性 Ca2+ 通道保持开放更长时间来增加 Ca2+ 流入。 在其他细胞中,突触前离子型受体的激活会增加递质释放。 突触前 Ca2+ 可渗透离子型受体通道(包括突触前 NMDA 型谷氨酸受体)的激活可以通过直接增强 Ca2+ 流入来增加释放。 激活不能渗透 Ca2+ 的突触前离子型受体通道可以通过使末端去极化和激活电压门控 Ca2+ 通道间接增加突触前 Ca2+ 水平。

因此,突触前终端具有多种机制,可以微调突触传递的强度。 尽管我们对突触强度的短期变化机制(持续数秒、数分钟和数小时的变化)了解相当多,但我们才刚刚开始了解它们的功能作用。 支持持续数天、数周甚至更长时间的变化的机制也仍然是个谜。 这些长期变化通常需要改变基因表达和突触前和突触后结构的生长,此外还需要改变 Ca2+ 流入和增强现有终端的递质释放。 我们将在第 53 章和第 54 章中讨论这些变化如何促进不同形式的长期学习和记忆。


\section{亮点}

1. 化学神经传递是神经元交流和处理信息的主要机制; 它发生在整个神经系统中。 神经递质的释放受到突触前神经末梢中一系列电和生化过程的刺激。 

2. 神经递质释放高度依赖于突触前末梢的去极化。 虽然动作电位由钠和钾电导控制,但触发释放的是去极化本身,而不是电压门控钠或钾通道的打开。 

3. 突触前末端去极化打开电压门控 Ca2+ 通道 (VGCC),导致 Ca2+ 流入。 这些通道集中在突触前“活跃区”,非常靠近发生释放的位置。 Ca2+ 流入和神经递质释放之间的关系是紧密耦合的并且是非线性的。 Ca2+ 峰值进入略微滞后于动作电位峰值,并迅速导致递质释放速率显着升高。 

4. VGCC 是异质的——已描述了具有不同生物物理、生物化学和药理学特性的五类。 多类 VGCC 可促进单个神经末梢的神经递质释放,并且是致病突变的靶标。 P/Q- 和 N- 型 Ca2+ 通道在中枢神经系统的活动区尤为突出。 

5. 化学传输通常涉及释放神经递质的量子包,其量子对应于单个突触小泡的内容。 在减少递质释放的条件下,例如细胞外 Ca2+ 降低,突触前动作电位会触发一些量子的概率释放,这会产生可变幅度的突触后反应,这些反应是对单个量子的单一反应的整数倍,中间穿插着完全失败 传输。 

6.单一事件是由大小和递质内容相对均匀的单个突触小泡融合驱动的。 可视化为小而清晰的球形膜细胞器,单个囊泡包含数以千计的小分子神经递质。 其他神经递质,包括生物胺和神经肽,被包装到一类不同的较大的致密核心囊泡中,这些囊泡介导较慢形式的突触传递。 哺乳动物中枢神经系统中参与快速突触传递的典型突触前末端包含 100 到 200 个囊泡。 少量小泡停靠在活性区的突触前膜上,最容易融合。

7. 在许多突触连接处,突触后电位的振幅可以描述为多种因素的产物:(1) 容易释放的囊泡占据的突触前位点的数量 (n),(2) 单个位点的释放概率 (p) 和 (3) 对释放单个囊泡 (a) 的突触后反应的大小。 在个别试验中,释放的囊泡数量可以用反映释放零个、一个、两个或更多个囊泡的可能性的二项分布来描述,就好像我们要计算抛 n 个硬币时正面朝上的数量。 

8. 胞吐作用是囊泡与突触前膜融合的过程,内吞作用是恢复囊泡的过程,在神经末梢和其他分泌结构中快速连续发生。 这些事件在形态学研究中很明显,并通过膜表面积的电测量进行实时研究。 

9. 胞吐作用由进化上保守的 SNARE 蛋白介导。 突触前质膜蛋白 syntaxin 和 SNAP-25 以及突触小泡膜蛋白 synaptobrevin 共同构成了 SNARE 复合体,即一组四个螺旋结构域。 这种复合物的形成对于囊泡融合至关重要,如各种神经毒素通过裂解 SNARE 蛋白阻断递质释放的能力所示。 SNARE 复合体装配由 SM 蛋白家族调节,例如 Munc18。 

10. 突触结合蛋白,如突触结合蛋白-1 (syt1),是丰富的囊泡蛋白,可作为调节囊泡释放的 Ca2+ 传感器。 Syt1 结合多个 Ca2+ 离子,因此在 Ca2+ 流入后与质膜形成紧密结合。 通过甚至在突触前 Ca2+ 升高之前就与 SNARE 复合体结合,它可能使该复合体能够快速引起融合。 

11. 突触小泡胞吐非常精确和快速,因为它的分子机制嵌入在由 RIM、RIM-BP 和 Munc-13 组成的活性区蛋白支架中。 复合物: 1. 通过 RIM 与 Rab3 和 Rab27 囊泡蛋白的结合将囊泡系在质膜上; 2. 通过与 RIM 和 RIM-BP 结合,将钙通道募集到栓系囊泡附近; 3. 通过与 Munc13 的交互促进 SNARE 复合体的组装。 活动区复合体还介导多种形式的短期和长期突触可塑性。 

12. 释放后囊泡膜的快速内吞作用使囊泡能够快速回收以在长时间刺激期间持续供应。 

13. 突触终端多种多样,释放特性各不相同。 活性区支架蛋白因突触和物种而异,突触前 Ca2+ 通道和突触结合蛋白的表达也不同。 在某些突触处,囊泡和 Ca2+ 通道似乎由复杂的结构网络对齐。 

14. 递质释放可以作为突触可塑性的一个方面进行内在或外在的调节。 在称为“抑郁”和“易化”的现象中,放电模式会从本质上强烈影响突触强度。 此外,外在神经调节剂可以通过调节 Ca2+ 通道或 Ca2+ 进入下游的事件来改变释放动力学。

\section{选读}

\section{参考文献}

