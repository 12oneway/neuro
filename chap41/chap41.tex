\chapter{小丘脑:生存的自主、激素和行为控制}
% PDF所在目录: /data2/whd/win10/learn/neuro/neuro_神经科学原理_28_中枢神经系统的听觉处理.pdf

\section{体内平衡将生理参数保持在一个狭窄的范围内,对生存至关重要}

\section{下丘脑协调稳态调节}
\subsection{下丘脑通常分为三个 Rostrocaudal 区域}
\subsection{模态特异性下丘脑神经元将内感受性感觉反馈与控制适应性行为和生理反应的输出联系起来}
\subsection{模态特异性下丘脑神经元也接收关于预期稳态挑战的下行前馈输入}

\section{自主系统将大脑与生理反应联系起来}
\subsection{自主系统中的内脏运动神经元被组织成神经节}
\subsection{节前神经元位于脑干和脊髓的三个区域}
\subsection{交感神经节投射到全身的许多目标}
\subsection{副交感神经节支配单个器官}
\subsection{肠神经节调节胃肠道}
\subsection{乙酰胆碱和去甲肾上腺素是自主运动神经元的主要递质}
\subsection{自主反应涉及自主部门之间的合作}

\section{内脏感觉信息被传递到脑干和高级脑结构}
\section{自主神经功能的中央控制可能涉及导水管周围灰质、内侧前额叶皮层和杏仁核}

\section{神经内分泌系统通过激素将大脑与生理反应联系起来}
\subsection{垂体后叶的下丘脑轴突末端将催产素和加压素直接释放到血液中}
\subsection{垂体前叶中的内分泌细胞响应下丘脑神经元释放的特定因子而分泌激素}

\section{专用的下丘脑系统控制特定的稳态参数}
\subsection{体温由中位视前核中的神经元控制}
\subsection{水平衡和相关的口渴驱动由终板、中位视前核和穹隆下器官的血管器官中的神经元控制}
\subsection{能量平衡和相关的饥饿驱动由弓状核中的神经元控制}

\section{下丘脑中的性二态性区域控制性、攻击性和育儿}
\subsection{性行为和攻击性受视前下丘脑区和下丘脑腹内侧核的一个分区控制}
\subsection{父母行为受视前下丘脑区控制}

\section{要点}
\subsection{选读}
\subsection{参考文献}