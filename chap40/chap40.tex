\chapter{脑干}
% PDF所在目录: /data2/whd/win10/learn/neuro/neuro_神经科学原理_28_中枢神经系统的听觉处理.pdf

\section{颅神经与脊神经同源}
\subsection{颅神经调节面部和头部的感觉和运动功能以及身体的自主功能}
\subsection{颅神经成群离开颅骨,常一起受伤}

\section{脑神经核团的组织遵循与脊髓的感觉和运动区域相同的基本计划}
\subsection{胚胎脑神经核具有节段性组织}
\subsection{成人脑神经核具有柱状组织}
\subsection{脑干的组织在三个重要方面不同于脊髓}

\section{脑干网状结构中的神经元群协调稳态和生存所需的反射和简单行为}
\subsection{颅神经反射涉及单突触和多突触脑干中继}
\subsection{模式发生器协调更复杂的刻板行为}
\subsection{呼吸控制提供了模式发生器如何集成到更复杂行为中的示例}

\section{脑干中的单胺能神经元调节感觉、运动、自主神经和行为功能}
\subsection{许多调节系统使用单胺作为神经递质}
\subsection{单胺能神经元具有许多细胞特性}
\subsection{自主调节和呼吸由单胺能途径调节}
\subsection{痛觉受单胺镇痛途径的调节}
\subsection{单胺能途径促进运动活动}
\subsection{上升的单胺能投射调节前脑系统的动机和奖励}
\subsection{单胺能和胆碱能神经元通过调节前脑神经元维持唤醒}

\section{要点}
\subsection{选读}
\subsection{参考文献}