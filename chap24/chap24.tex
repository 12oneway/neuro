\chapter{高层视觉处理:从视觉到认知}
% PDF所在目录: /data2/whd/win10/learn/neuro/neuro_神经科学原理_28_中枢神经系统的听觉处理.pdf

\section{高级视觉处理与对象识别有关}

\section{下颞叶皮层是物体识别的主要中心}
\subsection{临床证据表明颞下皮层对于物体识别至关重要}
\subsection{下颞叶皮层中的神经元编码复杂的视觉刺激,并按功能专门的列组织}
\subsection{灵长类动物的大脑包含用于面部处理的专用系统}
\subsection{下颞叶皮层是参与物体识别的皮层区域网络的一部分}

\section{物体识别依赖于感知恒常性}

\section{目标的分类感知简化了行为}

\section{视觉记忆是高级视觉处理的一个组成部分}
\subsection{隐式视觉学习导致神经元反应选择性的变化}
\subsection{视觉系统与工作记忆和长期记忆系统相互作用}

\section{视觉记忆的联想回忆依赖于处理视觉刺激的皮层神经元的自上而下的激活}

\section{要点}
\subsection{选读}
\subsection{参考文献}