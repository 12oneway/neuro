\chapter{海马体和外显记忆存储的神经基础}
% PDF所在目录: /data2/whd/win10/learn/neuro/neuro_神经科学原理_28_中枢神经系统的听觉处理.pdf

\section{哺乳动物的外显记忆涉及海马体的突触可塑性}
\subsection{不同海马通路的长期增强对于外显记忆存储至关重要}
\subsection{不同的分子和细胞机制有助于长期增强的表达形式}
\subsection{长期增强有早期和晚期}
\subsection{尖峰时间依赖性可塑性为改变突触强度提供了更自然的机制}
\subsection{海马体中的长期增强具有使其可用作记忆存储机制的特性}
\subsection{空间记忆取决于长期增强}

\section{显式记忆存储也依赖于突触传递的长期抑制}
\subsection{内存存储在单元组件中}

\section{海马体的不同分区处理外显记忆的不同方面}
\subsection{齿状回对于模式分离很重要}
\subsection{CA3 区域对于模式完成很重要}
\subsection{CA2 区域编码社会记忆}

\section{海马体形成外部世界的空间地图}
\subsection{内嗅皮层神经元提供独特的空间表征}
\subsection{位置细胞是空间记忆基质的一部分}
\subsection{海马体功能紊乱导致的自传体记忆障碍}

\section{亮点}
\subsection{选读}
\subsection{参考文献}


