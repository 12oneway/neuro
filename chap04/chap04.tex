\chapter{神经回路介导行为的神经解剖学基础}
人脑以当前计算机无法开始的方式执行操作。 仅仅是看——看世界并识别一张脸或面部表情——就需要惊人的计算成就。 事实上,我们所有的感知能力——看、听、闻、尝和触——都是分析的胜利。 同样,我们所有的自愿行动都是工程学的胜利。 感觉和运动虽然本身很奇妙,但与形成记忆或理解社会习俗等复杂的认知行为相比就显得苍白无力了。

大脑完成这些计算壮举是因为它的神经细胞在非常精确的功能回路中连接在一起。 大脑是分层组织的,因此在一个级别处理的信息被传递到更高级别的电路,以进行更复杂和精细的处理。 从本质上讲,大脑是一个网络的网络。 不同的大脑区域以整合的方式工作以完成有目的的行为。

在本章中,我们概述了一些回路的神经解剖结构,这些回路使大脑能够处理感觉输入并产生运动输出。 我们将触觉作为一种感觉方式来关注,因为体感系统特别容易理解,并且因为触觉清楚地说明了从脊髓到大脑皮层的多个级别的感觉处理回路的相互作用。 我们在这里的目的是说明电路如何控制行为的基本原理。 在下一章中,我们将考虑这些电路的功能特性,包括它们处理信息的计算。 在随后的章节中,我们将更详细地讨论各种感觉方式的解剖结构和功能,以及感觉输入如何调节运动。

最后,我们提供了大脑回路的预览,这些回路有助于产生我们日常生活的记忆,称为外显记忆(见第 52 章和第 54 章)。 我们这样做是为了说明,虽然记忆回路中的许多神经元与感觉和运动回路中的神经元相似,但并非所有神经元都相似。 此外,记忆系统中电路之间通路的组织与运动和感觉系统中的不同。 这突出了一个基本的神经生物学原则,即大脑的不同回路已经进化出一个组织,以最有效地执行特定功能。

理解大脑的功能组织乍一看似乎令人生畏。 但正如我们在前一章中看到的那样,大脑的组织通过三个解剖学考虑得到了简化。 首先,神经元的种类相对较少。 成千上万的脊髓运动神经元或数百万个新皮质锥体细胞中的每一个都具有相似的解剖结构并具有相似的功能。 其次,大脑和脊髓中的神经元聚集在称为核或大脑皮层离散区域的功能组中,形成网络或功能系统。 第三,大脑皮层的离散区域专门负责感觉、运动或记忆等联想功能。

\section{局部回路执行特定的神经计算,这些计算被协调以调制复杂的行为}

神经元相互连接形成功能电路。 例如,在脊髓内,简单的反射回路接收来自拉伸感受器的感觉信息,并将输出发送到各个肌肉群。 对于更复杂的行为功能,信息处理的不同阶段在神经系统不同区域的网络中进行。 神经系统内神经元之间的连接可以有不同的长度。

在大脑区域内,可能是兴奋性或抑制性的局部连接将许多神经元整合到功能网络中。 这样的局部网络然后可以通过更长的投射向一个或多个其他大脑区域提供输出。 许多这些较长的路径都有名称。 例如,从丘脑的外侧膝状体核到视觉皮层的投射称为视辐射。 从大脑的一侧到大脑另一侧的新皮质(大脑皮层中最靠近大脑表面的区域)的连接形成了胼胝体。 这些长路径携带的信息整合了许多本地电路的输出(图 4-1)。

考虑一下击打网球的简单动作(图 4-2)。 在视觉系统中分析有关接近球运动的视觉信息,视觉系统本身是一个分层组织的系统,从视网膜延伸到丘脑的外侧膝状体核,再到枕叶和颞叶的数十个皮质区域(第 21 章) . 此信息在运动皮层中与有关手臂、腿和躯干位置的本体感受信息相结合,以计算拦截球所需的运动。 一旦开始挥杆,其他专门负责运动的大脑区域(例如小脑)就会根据关于接近的球的轨迹和手臂位置的源源不断的感官信息,对运动程序进行许多细微的调整。

像大多数运动行为一样,打网球并不是硬连接到大脑回路中的,而是需要学习和记忆。 运动任务的记忆,称为程序记忆或内隐记忆,需要修改运动皮层、基底神经节和小脑中的回路。 最后,这整个行为是可以被意识接受的,并且可以引起对过去类似经历的有意识回忆,称为外显记忆和情绪。 外显记忆取决于海马体中的回路(第 52 和 54 章),而情绪则由杏仁核(第 42 和 53 章)和部分眶额皮质、扣带回和岛叶皮质调节。 当然,在执行挥杆动作时,大脑也会通过同样复杂的网络协调球员的心率、呼吸和其他稳态功能。



\section{体感系统中的感官信息回路}
复杂的行为,例如区分抓球和抓书所需的运动行为,需要多个核和皮层区域的综合行动。 信息在大脑中以分层方式处理。 因此,有关刺激的信息通过一系列皮层下区域然后是皮层区域传递; 在每个处理级别,信息变得越来越复杂。

此外,不同类型的信息,即使是在单一的感觉形态中,也会在几个解剖学上离散的通路中被处理。 在体感系统中,对同一皮肤区域的轻触和针刺疼痛是由皮肤中连接到大脑中不同通路的不同感觉受体介导的。 用于精细触觉、压力和本体感觉的系统称为表观系统,而用于疼痛和温度的系统称为原发病系统。


\subsection{来自躯干和四肢的体感信息被传送到脊髓}
来自躯干和四肢的各种形式的体感信息进入脊髓,脊髓有一个核心的 H 形灰质区域,神经元细胞体位于该区域。 灰质被有髓轴突形成的白质包围,这些轴突构成短连接和长连接。 脊髓两侧的灰质分为背角(或后角)和腹角(或前角)(图 4-3)。

背角包含一组次级感觉神经元(感觉核),其树突从支配身体皮肤、肌肉和关节的初级感觉神经元接收刺激信息。 腹角包含运动神经元组(运动核),其轴突离开脊髓并支配骨骼肌。 脊髓具有介导从牵张反射到肢体运动协调等行为的回路。

正如我们在第 3 章中讨论的那样,在考虑膝跳反射时,灰质中各种类型的中间神经元调节脊髓运动神经元的输出(见图 3-5)。 这些中间神经元中的一些是兴奋性的,而另一些是抑制性的。 这些中间神经元调节流向大脑的感觉信息和从大脑下传至脊髓运动神经元的运动指令。 运动神经元还可以通过中间神经元调节其他运动神经元的输出。 当我们在第 32 章讨论脊髓时,将更详细地考虑这些回路。

灰质周围的白质包含成束的上升轴突和下降轴突,它们分为背侧柱、侧柱和腹侧柱。 位于灰质两个背角之间的背柱仅包含将体感信息传递到脑干的上升轴突(图 4-1)。 侧柱包括来自脑干和新皮质的上升和下降轴突,它们支配脊髓中间神经元和运动神经元(图 4-3)。 这证明了关于中枢神经系统连接的一般原则。 处理往往是分层的:从较低到较高处理区域的投射被称为前馈,而下降投射可以调节脊髓反射并被认为是反馈。 A 区投射到 B 区并反过来接收 B 区的返回投射的主题在整个神经系统中重复出现。 腹侧柱还包括上升和下降轴突。 侧柱和腹侧柱中的上升体感轴突构成平行通路,将有关疼痛和热感觉的信息传递到中枢神经系统的更高水平。 下行轴突控制轴向肌肉和姿势。

脊髓沿其长度分为四个主要区域:颈椎、胸椎、腰椎和骶椎(图 4-4)。 从这些区域产生的连接根据肌肉、骨骼和身体其他成分发育的胚胎体节进行分离(第 45 章)。 从脊髓突出到在同一节段发育的身体结构的轴突与在椎间孔中进入脊髓的轴突连接形成脊神经。 颈椎水平的脊神经与头后部、颈部和手臂的感觉知觉和运动功能有关; 胸椎神经支配上躯干; 腰椎和骶椎神经支配下躯干、背部和腿部。

脊髓的四个区域中的每一个区域都包含多个节段,这些节段大致对应于每个区域中的不同椎骨; 有8个颈段,12个胸段,5个腰段,5个骶段。 成熟脊髓的实际物质看起来并不分段,但四个脊柱区域的分段仍然由进入或离开脊髓的背根和腹根的数量和位置来定义。 由于两个组织特征,脊髓沿其 rostrocaudal 轴的大小和形状各不相同。

首先,相对较少的感觉轴突进入骶骨水平的脊髓。 进入脊髓的感觉轴突的数量在逐渐升高的水平(腰椎、胸椎和颈椎)上增加。 相反,大多数来自大脑的下行轴突终止于颈椎水平,逐渐减少到脊髓的较低水平。 因此,白质中的纤维数量在颈椎水平最高(上行和下行纤维数量最多),而在骶骨水平最低。 因此,脊髓骶骨层的白质比灰质少得多,而颈髓的白质多于灰质(图 4-4)。

第二个组织特征是腹角和背角大小的差异。 腹角在运动神经支配手臂和腿部的水平较大。 专用于身体区域的腹侧运动神经元的数量大致与该区域运动的灵巧性平行。 因此,与躯干相比,需要更多的运动神经元来支配更多的肌肉并调节四肢更复杂的运动。 同样,背角较大,四肢的感觉神经进入脊髓。 四肢具有更高密度的感觉感受器来介导更精细的触觉辨别,从而将更多的感觉纤维发送到脐带。 脊髓的这些区域被称为腰骶和颈椎增大(图 4-4)。

\subsection{躯干和四肢的初级感觉神经元聚集在背根神经节}
将信息从四肢和躯干的皮肤、肌肉和关节传递到脊髓的感觉神经元聚集在脊柱内的背根神经节中,紧邻脊髓(图 4-5)。 这些神经元的形状是伪单极的; 它们有一个带有中央和外围分支的分叉轴突。 周围分支作为游离神经末梢支配皮肤、肌肉或其他组织,或与用于感知触觉、本体感觉(拉伸感受器)、疼痛和温度的专门感受器相关联。

体感系统及其从感受器到知觉的通路在第 17 章到第 20 章中有更全面的描述。在这一点上,基本上有两条来自外围的体感通路,它们要么携带触摸和拉伸(epicritic 系统),要么携带疼痛和 温度(原病系统)。 Epicritic 纤维在后柱-内侧丘系系统中移动(图 4-6)。 来自背根神经节神经元的中央定向轴突在背(或后)柱白质中上升并终止于髓质的薄核或楔形核。 疼痛和温度通路的中枢轴突形成脊髓丘脑通路。 它们终止于脊髓背角的灰质内。 二级神经元穿过脊髓的另一侧并在脊髓丘脑前束和侧束中上升(图 4-6)。 两条通路最终都终止于丘脑,丘脑将投射发送到大脑皮层的初级体感区域。 在下一节中,我们将重点关注史诗批评系统。

来自触觉和本体感觉神经元的局部和上升分支为体感信息从背根神经节细胞进入脊髓提供了两条功能途径。 局部分支可以激活调节运动输出的局部反射回路,而上行分支将信息传送到大脑,这些信息在丘脑和大脑皮层中进一步处理。


\subsection{脊髓中背根神经节神经元的中央轴突末端产生体表图}
背根神经节细胞的中央轴突终止于脊髓的方式形成了体表的神经图谱。 来自身体表面不同部分的输入的这种有序的躯体分布在整个上升的体感通路中保持不变。 这种排列说明了神经组织的另一个重要原则。 在任何特定水平上构成神经回路的神经元通常以系统的方式连接并且在个体之间看起来相似。 同样,连接神经系统不同层次不同处理区域的纤维束也以高度组织化和刻板的方式排列。

进入骶骨区脊髓的轴突在靠近中线的背柱中上升,而那些依次进入更高水平的轴突在背柱内逐渐向外侧位置上升。 因此,在颈髓中,来自身体所有部分的轴突已经进入,来自下半身的感觉纤维位于背柱的内侧,而来自躯干、手臂和肩部,最后是颈部的纤维逐渐占据更多 横向区域。 在颈脊髓中,形成背柱的轴突分为两束:位于内侧的纤细束和位于更外侧的楔形束(图 4-1)。


\subsection{每个躯体亚模态都在从外围到大脑的不同子系统中处理}

躯体感觉的子模态——触觉、痛觉、温度觉和位置觉——在大脑中通过终止于不同大脑区域的不同通路进行处理。 我们通过触摸子模态的信息路径来说明这些平行路径的特殊性。

携带触觉信息的初级传入纤维进入同侧背柱并上升到髓质。 来自下半身的纤维在细束中运行并终止于细核,而来自上半身的纤维在楔形束中运行并终止于楔形核。 纤细核和楔形核中的神经元产生轴突,轴突穿过大脑的另一侧并在称为内侧丘系的长纤维束中上升到丘脑(图 4-1)。

与脊髓的背柱一样,内侧丘系的纤维按体表排列。 因为携带感觉信息的纤维穿过中线到达大脑的另一侧,大脑的右侧接收来自身体左侧的感觉信息,反之亦然。 内侧丘系的纤维终止于丘脑的特定部分,称为腹侧后外侧核(图 4-1)。 在那里,纤维保持它们的躯体组织,使得那些从下半身传递信息的纤维在横向结束,而那些从上半身传递信息的纤维在中间终止。

\section{丘脑是感觉受体和大脑皮层之间的重要纽带}

丘脑是一个蛋形结构,构成间脑的背侧部分。 它包含一类称为丘脑中继细胞的兴奋性神经元,可将感觉输入传递到大脑皮层的初级感觉区域。 然而,丘脑不仅仅是一个中继站。 它充当信息到大脑皮层的守门人,根据生物体的行为状态阻止或增强特定信息的通过。

大脑皮层的反馈投射部分终止于丘脑的一个特殊部分,称为丘脑网状核。 这个细胞核在丘脑周围形成一层薄片,几乎完全由突触到中继细胞上的抑制性神经元组成。 它根本不投射到新皮质。 除了接收来自新皮质的反馈投射外,网状核还接收来自轴突的输入,离开丘脑前往新皮质,使丘脑能够调节其中继细胞对传入的感觉信息的反应。

丘脑是一个很好的例子,大脑区域由几个明确定义的核组成。 已识别出多达 50 个丘脑核团(图 4-7)。 一些细胞核接收特定于感觉方式的信息并投射到新皮层的特定区域。 例如,腹侧后外侧核(内侧丘系终止处)中的细胞处理体感信息,它们的轴突投射到初级体感皮层(图 4-1 和 4-7)。 来自视网膜神经节细胞的投射终止于丘脑的另一部分,称为外侧膝状体核(图 4-7)。 该核中的神经元依次投射到视觉皮层。 丘脑的其他部分参与运动功能,将信息从小脑和基底神经节传递到额叶的运动区域。 来自投射到新皮层的丘脑细胞的轴突在辐射冠中行进,辐射冠是携带大部分轴突进出大脑半球的大纤维束。 通过与额叶和海马体的联系,丘脑可能在记忆等认知功能中发挥作用。 一些可能在注意力投射中起作用的核弥漫性地投射到皮层的大但明显不同的区域。

丘脑的细胞核最常见的分为四组——前部、内侧、腹外侧和后部——相对于内髓板,一种片状纤维束,贯穿丘脑的尾部长度(图 4-7) ). 因此,内侧核团位于髓内板的内侧,而腹外侧核和后核核位于其外侧。 在丘脑的喙极,内髓板分裂并包围前部。 丘脑的尾极被后组占据,以枕核为主。 神经元组也位于内髓板的纤维内,统称为板内核。

前组从下丘脑的乳头状核和海马结构的前下托接收其主要输入。 前组的作用尚不确定,但由于其联系,被认为与记忆和情感有关。 前组主要与扣带回和额叶皮质区域相互连接。

内侧组主要由内侧核组成。 这个大的丘脑核分为三个部分,每个部分都与额叶皮层的特定部分相连。 细胞核接收来自基底神经节、杏仁核和中脑部分的输入,并与记忆和情绪处理有关。

腹外侧核的命名是根据它们在丘脑内的位置命名的。 腹侧前核和腹侧核对运动控制很重要,并将信息从基底神经节和小脑传递到运动皮层。 腹侧后核将体感信息传递给新皮质。 如前所述,腹后外侧核传递来自脊髓束的信息。 腹后内侧核传递来自面部的信息,主要通过三叉神经(颅神经V)进入脑干。

后组包括内侧和外侧膝状体核、外侧后核和枕核。 内侧膝状体是听觉系统的一个组成部分,根据其输入携带的声音频率信息按音调组织; 它将听觉信息传递到颞叶颞上回的初级听觉皮层。 外侧膝状体接收来自视网膜的信息并将其传送到枕叶的初级视觉皮层。 与啮齿类动物相比,灵长类大脑中的枕骨不成比例地扩大,尤其是在人类大脑中,其发育似乎与顶叶、枕叶和颞叶皮质联合区域的扩大平行。 它已被分为至少三个分支,并与顶叶、颞叶和枕叶的广泛区域以及与视觉相关的脑干的上丘和其他核团广泛互连。

如前所述,丘脑不仅投射到新皮质(前馈连接),而且还接收来自新皮质的大量返回输入(反馈连接)。 例如,在外侧膝状核中,轴突从视觉皮层的反馈投射形成的突触数量实际上比外侧膝状核从视网膜接收到的突触数量还要多! 这种反馈被认为在感觉信息的处理中起着重要的调节作用,尽管确切的功能尚不清楚。 尽管这种反馈主要来自双眼激活的皮层神经元,但外侧膝状体核中的神经元仅对一只或另一只眼睛有反应。 这意味着它们主要是由来自视网膜(来自不同层的不同眼睛)的输入驱动的,而不是来自皮质的反馈,尽管它在数量上有优势。 丘脑的大多数核团从大脑皮层接收到类似显着的返回投射,这些投射的意义是神经科学未解之谜之一。

到目前为止描述的丘脑核团被称为中继(或特定)核团,因为它们与新皮质的特定部分具有特定和选择性的关系。 其他丘脑核团,称为非特异性核团,投射到几个皮质和皮质下区域。 这些核位于丘脑的中线(中线核)或内髓层内(板内核)。 最大的中线核是室旁核、脑旁核和 reuniens 核; 最大的层内细胞群是中央核。 椎板内核投射到内侧颞叶结构,如杏仁核和海马体,但也投射到部分基底神经节。 这些核从脊髓、脑干和小脑中的各种来源接收输入,并被认为可以调节皮质觉醒。

丘脑是感觉处理层次结构中的重要一步,而不是将信息简单地传递到新皮质的被动中继站。 它是一个复杂的大脑区域,在这里进行大量信息处理(图 4-1)。 仅举一个例子,来自腹侧后外侧核的体感信息的输出受到四种类型的处理:(1)核内的局部处理; (2) 脑干输入的调节,例如来自去甲肾上腺素能和血清素能系统; (3) 来自网状核的抑制性输入; (4) 来自新皮层的调节反馈。


\section{感觉信息处理在大脑皮层达到顶峰}

来自腹侧后外侧核的体感信息主要传递到初级体感皮层(图 4-1)。 这里的神经元对皮肤表面的触觉刺激非常敏感。 与触觉感觉处理的早期阶段一样,体感皮层是按体位组织的(图 4-8)。

当神经外科医生 Wilder Penfield 在 20 世纪 40 年代末和 50 年代初刺激接受脑部手术的患者的体感皮层表面时,他发现来自下肢的感觉是由位于大脑中线附近的神经元介导的,而来自上肢的感觉 身体、手和手指、脸、嘴唇和舌头由位于侧面的神经元介导。 Penfield 发现,虽然身体的所有部分都在皮质体表中表示,但用于每个身体部位的皮质表面积与其质量不成正比。 相反,它与身体部位的辨别精细度成正比,而这又与感觉纤维的神经分布密度有关(第 19 章)。 因此,专用于手指的皮质区域比手臂的皮质区域大。 同样,嘴唇和舌头的表征比面部其余部分占据更多的皮质表面(图 4-8)。 正如我们将在第 53 章中看到的那样,用于特定身体部位的皮层数量不是固定的,而是可以根据经验进行修改,就像在音乐会小提琴手身上看到的那样,用于手指的体感皮层区域有所扩展 用来拨弦的手。 这说明了大脑回路的一个重要方面:它能够响应使用或停用而发生可塑性变化。 这些变化对于各种形式的学习都很重要,包括中风后恢复功能的能力。

最靠近大脑表面的大脑皮层区域被组织成层和列,这种排列提高了计算效率。 皮质在进化过程中经历了戏剧性的扩张。 最近的新皮质包括哺乳动物的大部分皮质。 在较大的灵长类动物和鲸类动物的大脑中,新皮质表面是一张折叠起来有很深皱纹的薄片,因此可以将三倍多的皮质表面装入一个适度扩大的头部。 事实上,大约三分之二的新皮质沿着皮质的深层皱纹,称为脑沟。 新皮质的其余部分位于薄片的外部褶皱处,称为脑回。 新皮质接收来自丘脑、大脑两侧的其他皮质区域和其他皮质下结构的输入。 它的输出被定向到皮质、基底神经节、丘脑、桥脑核和脊髓的其他区域。

这些复杂的输入-输出关系在皮质神经元的有序分层中得到有效组织; 每层包含不同的输入和输出。 新皮质的许多区域,特别是初级感觉区域,包含六层,从大脑外表面到白质编号(图 4-9)。

第一层,即分子层,由位于更深层的细胞的树突和穿过该层以在皮层的其他区域建立连接的轴突占据。

第二层和第三层主要包含小的金字塔形细胞。 第二层,外部颗粒细胞层,是包含小球形神经元的两层之一。 第三层称为外锥体细胞层(内锥体细胞层位于更深的水平)。 位于第三层较深的神经元通常比位于较浅层的神经元大。 II 层和 III 层锥体神经元的轴突局部投射到同一皮质区域内的其他神经元以及其他皮质区域,从而调节皮质内通讯(图 4-10)。

第四层包含大量小球形神经元,因此称为内部颗粒细胞层。 它是丘脑感觉输入的主要接受者,在初级感觉区最为突出。 例如,作为初级视觉皮层的枕叶皮层区域有一个非常突出的第四层。 该区域的第 IV 层由神经元组成,非常复杂,通常分为三个子层。 具有突出的第 IV 层的区域称为颗粒状皮层。 相比之下,初级运动皮层所在的中央前回基本上没有第四层,因此是所谓的无颗粒额叶皮层的一部分。 这两个皮质区域是组织学切片中最容易识别的区域之一(图 4-11)。

第 V 层,即内部锥体细胞层,主要包含金字塔形细胞,这些细胞通常比第 III 层中的细胞大。 该层中的锥体神经元产生皮质的主要输出通路,投射到其他皮质区域和皮质下结构(图 4-9)。

VI 层中的神经元在形状上相当异质,因此这一层称为多态层或多形式层。 它融入形成皮层深层界限的白质中,并携带轴突进出皮层区域。

各个层的厚度及其功能组织的细节在整个皮质中各不相同。 大脑皮层的早期研究者 Korbinian Brodmann 使用第 IV 层上下各层的相对突出程度、细胞大小和包装特征来区分新皮层的不同区域。 基于这种细胞结构差异,1909 年,Brodmann 将大脑皮层分为 47 个区域(图 4-11)。

尽管 Brodmann 的分界部分与新皮层局部功能的信息一致,但细胞构造方法本身并不能捕捉到皮层所有不同区域功能的微妙之处或多样性。 例如,Brodmann 确定了五个区域(区域 17-21)与猴子的视觉功能有关。 相比之下,现代连接神经解剖学和电生理学已经在 Brodmann 识别的五个区域中识别出超过 35 个功能不同的皮层区域。

在新皮质内,信息通过前馈和反馈连接从一个突触中继传递到另一个突触中继。 例如,在视觉系统中,从初级视觉皮层到二级和三级视觉区域的前馈投射主要起源于第三层,主要终止于目标皮层区域的第四层。 相反,对早期处理阶段的反馈投射源自第 V 层和第 VI 层的细胞,并终止于第 I、II 和 VI 层(图 4-12)。

大脑皮层在功能上被组织成从白质延伸到皮层表面的细胞柱。 (这种柱状组织在标准组织学制备中并不特别明显,最初是在电生理学研究中发现的。)每根柱状物的直径约为三分之一毫米。 每列中的单元格形成一个具有高度专业化功能的计算模块。 列内的神经元往往具有非常相似的响应特性,大概是因为它们形成了局部处理网络。 专用于某个功能的皮质区域越大,专用于该功能的计算列的数量就越多(第 23 章)。 手指具有高度辨别力的触觉是大面积皮层中许多专门处理来自手的体感信息的皮层柱的结果。

除了皮层柱的识别之外,早期电生理学研究的第二个主要见解是体感皮层包含的不是一个而是几个身体表面的体表图。 初级躯体感觉皮层(前顶叶皮层)有四张完整的皮肤图,布罗德曼区 3a、3b、1 和 2 各一张。丘脑并行发送大量深层受体信息(例如,来自肌肉) 区域 3a 和它的大部分皮肤信息到区域 3b 和 1。区域 2 接收来自这些丘脑皮质区域的输入,并且可能负责我们对三维立体物体的综合感知,称为立体视觉。 初级体感皮层中的神经元投射到邻近区域的神经元,而这些神经元又投射到其他邻近的皮层区域(图 4-13)。 在皮层连接层次结构的更高层次上,体感信息用于运动控制、手眼协调和与触觉相关的记忆。

参与感觉处理早期阶段的皮层区域主要与单一感觉方式有关。 这些区域称为初级感觉或单峰(感觉)关联区域。 来自单峰关联区域的信息会聚在与组合感觉方式有关的皮层多峰关联区域(图 4-13)。 这些与海马体密切相关的多模态关联区域似乎对两个功能特别重要:(1) 统一知觉的产生和 (2) 知觉在记忆中的表征(我们将在后面讨论这个问题) 本章结束)。

因此,从皮肤感受器上的机械压力到手指被握手的朋友触摸的感知,体感信息在从背根神经节到大脑的一系列越来越复杂的回路(网络)中进行处理。 体感皮层,到单峰关联区,最后到多峰关联区。 体感信息的主要目的之一是指导定向运动。 正如人们想象的那样,皮层的体感和运动功能之间存在着密切的联系。

\section{自主运动是由皮层和脊髓之间的直接连接介导的}

正如我们将在第 25 章和第 30 章中看到的那样,知觉系统的一个主要功能是为运动系统调节的动作提供必要的感觉信息。 初级运动皮层的组织结构类似于体感皮层(图 4-8B)。 运动皮层的特定区域影响特定肌肉群的活动(第 34 章)。

初级运动皮层 V 层神经元的轴突提供新皮层的主要输出以控制运动。 一些 V 层神经元通过皮质脊髓束中投射到脊髓腹角中的运动神经元直接影响运动。 其他人通过突触到髓质中的运动输出核或基底神经节中的纹状体神经元上来影响运动控制。 人类皮质脊髓束由大约一百万个轴突组成,其中大约 40\% 起源于运动皮层。 这些轴突向下穿过皮层下白质、内囊和中脑的大脑脚(图 4-14)。 在髓质中,纤维在腹面形成突出的突起,称为髓质金字塔,因此整个突起有时称为锥体束。

与上行体感系统一样,下行皮质脊髓束穿过脊髓的另一侧。 大多数皮质脊髓纤维在称为锥体交叉的位置穿过髓质中线。 然而,大约 10\% 的纤维在到达它们将终止的脊髓水平之前不会交叉。 皮质脊髓纤维与运动神经元建立单突触连接,这种连接对于个性化的手指运动尤为重要。 它们还与脊髓中的兴奋性和抑制性中间神经元形成突触,这些连接对于在伸手和行走等行为中协调较大的肌肉群很重要。

皮质脊髓束中携带的运动信息受到感觉信息和来自其他运动区域的信息的显着调节。 需要连续不断的触觉、视觉和本体感受信息流,才能使随意运动既准确又顺序正确。 此外,运动皮层的输出受到大脑其他运动区域的重大影响,包括小脑和基底神经节,这些结构对于顺利执行运动至关重要。 这两个皮层下区域在第 37 章和第 38 章中有详细描述,它们提供对熟练动作的顺利执行必不可少的反馈,因此对于通过练习提高运动技能也很重要(图 4-15)。

\section{大脑中的调节系统影响动机、情绪和记忆}
大脑的某些区域既不是纯粹的感觉也不是纯粹的运动,而是调节特定的感觉或运动功能。 调节系统通常涉及对饥饿、口渴或睡眠等基本需求做出反应的行为。 例如,下丘脑中的感觉和调节系统决定血糖水平(第 41 章)。 当血糖降至某个临界水平以下时,我们就会感到饥饿。 为了满足饥饿感,大脑中的调节系统将视觉、听觉和嗅觉集中在与进食相关的刺激上。

脑干内不同的调节系统调节注意力和觉醒(第 40 章)。 脑干中的小核含有合成和释放调节性神经递质去甲肾上腺素(蓝斑)和血清素(中缝背核)的神经元。 这些神经元通过与前脑结构的广泛联系来设定动物的一般觉醒水平。 一组胆碱能调节神经元,即 Meynert 基底核,参与觉醒或注意力(第 40 章)。 该核位于端脑基底前脑部分的基底神经节下方。 其神经元的轴突基本上投射到新皮质的所有部分。

如果捕食者发现潜在的猎物,各种皮层和皮层下系统将决定猎物是否可食用。 一旦识别出食物,其他皮层和皮层下系统就会启动一个全面的自愿运动程序,使动物与猎物接触、捕获猎物并将其放入口中、咀嚼和吞咽。

最后,动物在进食中的生理满足感强化了导致成功捕食的行为。 中脑中的一组多巴胺能神经元对于监测强化和奖励很重要。 多巴胺能调节系统的力量已经通过实验证明,在实验中,将电极植入大鼠的奖励区域,让动物自由地按下杠杆以电刺激它们的大脑。 他们更喜欢这种自我刺激,而不是获取食物或水、进行性行为或任何其他自然有益的活动。 第 38 章描述了多巴胺能调节系统在通过强化探索行为进行学习中的作用。

与奖励、注意力和动机有关的大脑调节系统如何与感觉和运动系统相互作用是神经科学中最有趣的问题之一,也是我们理解学习和记忆存储的基础(第 40 章)。


\section{周围神经系统在解剖学上与中枢神经系统不同}
周围神经系统为中枢神经系统提供有关身体外部环境和内部环境的连续信息流。 它具有躯体和自主神经分裂(图 4-16)。


躯体分裂包括从皮肤、肌肉和关节接收信息的感觉神经元。 这些感觉神经元的细胞体位于背根神经节和颅神经节中。 与这些细胞相关的受体提供有关肌肉和肢体位置以及体表触摸和压力的信息。 在第四部分(感知)中,我们将看到这些感受器在将一种或另一种物理能量(例如深压力或热量)转换为神经系统使用的电信号方面是多么的专业化。 在第五部分(运动)中,我们将看到肌肉和关节中的感觉感受器对于塑造身体的连贯运动至关重要。

周围神经系统的自主神经分裂介导内脏感觉以及内脏、血管系统和外分泌腺的运动控制。 它由交感神经系统、副交感神经系统和肠道系统组成。 交感神经系统参与身体对压力的反应,而副交感神经系统则起到保护身体资源和恢复体内平衡的作用。 肠神经系统,神经元细胞体位于或邻近内脏,控制平滑肌和肠道分泌物的功能。 第 41 章描述了自主神经系统的功能组织,第 42 章描述了它在情绪和动机中的作用。


\section{记忆是一种复杂的行为,由不同于执行感觉或运动的结构介导}
过去 50 年的研究提供了大脑中记忆系统的复杂视图。 我们现在知道不同形式的记忆(例如,恐惧记忆与技能记忆)由不同的大脑区域调节。 在这里,我们对比了负责编码和存储我们对其他个人、地点、事实和事件的体验的系统组织,这个过程称为外显记忆。

我们知道称为海马体的结构(或更准确地说是海马体结构,因为它包含多个皮层区域)是编码和存储我们生活记忆的内侧颞叶记忆系统的关键组成部分(图 4-17)。 这种理解主要基于对著名患者亨利·莫莱森(Henry Molaison,生前研究过他的科学家简称为 HM)的分析,他在 1950 年代初进行了双侧颞叶手术以减轻危及生命的癫痫症。 与六层新皮质相比,海马体连同嗅觉皮质(梨状皮质)是一种三层皮质结构,称为原皮质,是系统发育较早的皮质区域之一。

我们在本章中简要描述海马结构的原因是要强调并非所有大脑回路都是相似的。 事实上,无论是谈论嗅觉开始处理的嗅球,还是细化精细运动的小脑,一般原则是回路的结构特定于它所调节的功能 . 海马体回路与人们想象的调节感官知觉或运动运动的回路截然不同。 大脑的海马体回路将在后面的章节中更详细地讨论。 第 5 章介绍了海马体对动物在其环境中的空间位置信息进行编码以及外显记忆(包括空间记忆)的编码需要突触功能发生可塑性变化的观点。 第 52 章和第 54 章分别探讨了人类记忆功能以及外显记忆和空间表征的细胞和分子基础。

\subsection{海马系统与最高级别的多感觉皮层区域相互连接}

感觉系统是分层的,并且在更高层次上逐渐处理更复杂的刺激,特别是新皮质。 此外,从每个模态的最高级别,电路连接位于皮层周围不同位置的多感觉皮层区域,来自许多感觉模态的信息会聚到单个神经元上。 海马体系统从几个特定的多感觉区域接收大部分输入,即制造记忆的原材料。 这些包括位于内侧颞叶的鼻周和海马旁皮质,以及位于扣带回尾部的脾后皮质。 这些多感觉区域汇聚在海马系统的入口结构上,即内嗅皮层(图 4-18)。 进入内嗅皮层的多感觉信息可以被认为是对即时经验的总结。

\subsection{海马结构由几个不同但高度集成的电路组成}
海马结构由许多不同的皮质区域组成,这些区域的组织比新皮质更简单——至少它们的层数更少。 这些区域包括齿状回、海马体、下托和内嗅皮质。 这些区域中的每一个都由包含许多神经元细胞类型的子区域组成。 海马结构中最简单的子区域是齿状回,它有一个称为颗粒细胞的主要神经元。 海马体的亚区域称为 CA1、CA2 和 CA3,由单层锥体细胞组成,其树突在细胞体层上方和下方延伸,并接收来自多个区域的输入。 下托(分为下托、前托和副托)是另一个主要由锥体细胞组成的区域。 最后,海马结构中最复杂的部分是内嗅皮质,它有多层,但仍然具有与新皮质明显不同的组织。 例如,它缺少第 IV 层,而具有更突出的第 II 层。

\subsection{海马结构主要由单向连接组成}

在这里,我们描述了海马结构的基本电路。 该电路在第 54 章中有更详细的描述。图 4-19 中显示的海马电路的简化版本强调了它对多模式感觉信息的逐步串行处理,每个海马区域都有助于外显记忆的形成。 这种串行处理意味着对该系统的任何一个组件的损坏都会导致记忆障碍。 事实上,另一名姓名缩写为 R.B. 的著名患者确实因缺血发作后 CA1 区域的细胞丢失而遭受严重的记忆障碍。

事实证明,虽然海马体的形成对于我们生活记忆的初始形成至关重要,但这些记忆最终存储在大脑的其他地方。 在像 HM 这样的患者中,内嗅皮质和大部分海马系统的其余部分都被切除了,手术前的记忆基本完好无损。 因此,为了实现我们生活记忆的创造和长期储存,海马体和内嗅皮层必须与大脑皮层中的回路进行交流。 发生这种情况的地点和具体方式仍然是个谜。


\section{亮点}

1. 单个神经元不能进行行为。 它们必须被纳入由不同类型的神经元组成的电路中,这些神经元通过兴奋性、抑制性和调节性连接相互连接。

2. 感觉和运动信息在大脑中同时活跃的各种不同的大脑区域中进行处理。

3. 功能通路由可识别的脑区串联而成,每个脑区的回路比前一个脑区处理更复杂或更具体的信息。

4. 触觉和痛觉是由脊髓、脑干、丘脑和新皮质中不同回路之间的通路介导的。 

5. 所有的感觉和运动系统都遵循分层和相互处理信息的模式,而海马体记忆系统主要是为非常复杂的多感觉信息的串行处理而组织的。 一般原则是大脑中的电路具有适合其执行功能的组织结构。 

6. 与对我们个人经历的直觉分析相反,感知并不是我们周围世界的精确复制品。 感觉是对现实的抽象,而不是复制。 在首先分析外部物理事件的各种特征之后,大脑的回路构建了外部物理事件的内部表征。 当我们把一个物体拿在手里时,不同的大脑区域会根据大脑自身的规律,同时对物体的形状、运动和质地进行分析,并将结果整合到有意识的体验中。 

7. 感觉如何整合到有意识的体验中——绑定问题——以及有意识的体验如何从大脑对传入的感觉信息的分析中浮现出来,这是认知神经科学中最有趣的两个问题(第 56 章)。 一个更复杂的问题是这些有意识的印象是如何编码成存储数十年的记忆的。

\section{选读}
\section{参考文献}























