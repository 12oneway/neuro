\chapter{神经回路介导行为的神经解剖学基础}
% PDF所在目录: /data2/whd/win10/learn/neuro/neuro_神经科学原理_28_中枢神经系统的听觉处理.pdf

\section{局部回路执行特定的神经计算,这些计算被协调以调制复杂的行为}


\section{体感系统中的感官信息回路}
\subsection{来自躯干和四肢的体感信息被传送到脊髓}
\subsection{躯干和四肢的初级感觉神经元聚集在背根神经节}
\subsection{脊髓中背根神经节神经元的中央轴突末端产生体表图}
\subsection{每个躯体亚模态都在从外围到大脑的不同子系统中处理}

\section{丘脑是感觉受体和大脑皮层之间的重要纽带}

\section{感觉信息处理在大脑皮层达到顶峰}

\section{自主运动是由皮层和脊髓之间的直接连接介导的}

\section{大脑中的调节系统影响动机、情绪和记忆}

\section{周围神经系统在解剖学上与中枢神经系统不同}

\section{记忆是一种复杂的行为,由不同于执行感觉或运动的结构介导}
\subsection{海马系统与最高级别的多感觉皮层区域相互连接}
\subsection{海马结构由几个不同但高度集成的电路组成}
\subsection{海马结构主要由单向连接组成}


\section{亮点}
\section{选读}
\section{参考文献}























