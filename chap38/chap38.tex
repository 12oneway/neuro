\chapter{基底神经节}
% PDF所在目录: /data2/whd/win10/learn/neuro/neuro_神经科学原理_28_中枢神经系统的听觉处理.pdf

\section{基底神经节网络由三个主输入核、两个主输出核和一个内在核组成}
\subsection{纹状体、丘脑底核和黑质致密部/腹侧被盖区是基底神经节的三个主要输入核}
\subsection{黑质网状部和内部苍白球是基底神经节的两个主要输出核}
\subsection{外部苍白球主要是基底神经节的内在结构}

\section{基底神经节的内部电路调节组件如何相互作用}
\subsection{基底神经节的传统模型强调直接和间接通路}
\subsection{详细的解剖分析揭示了一个更复杂的组织}

\section{基底神经节与外部结构的连接以重入环为特征}
\subsection{输入定义基底神经节的功能区域}
\subsection{输出神经元投射到提供输入的外部结构}
\subsection{重入环是基底神经节电路的基本原理}

\section{生理信号为基底神经节的功能提供了更多线索}
\subsection{纹状体和丘脑底核主要接收来自大脑皮层、丘脑和中脑腹侧的信号}
\subsection{腹侧中脑多巴胺神经元接收来自外部结构和其他基底神经节核的输入}
\subsection{去抑制是基底神经节输出的最终表达}

\section{在整个脊椎动物进化过程中,基底神经节一直高度保守}

\section{动作选择是基底神经节研究中反复出现的主题}
\subsection{所有脊椎动物都面临从多个竞争选项中选择一种行为的挑战}
\subsection{动机、情感、认知和感觉运动处理需要选择}
\subsection{基底神经节的神经结构被配置为进行选择}
\subsection{基底神经节促进选择的内在机制}
\subsection{基底神经节的选择功能受到质疑}

\section{强化学习是选择架构的固有属性}
\subsection{内在增强是由基底神经节核内的相位多巴胺信号介导的}
\subsection{外在强化可以通过在传入结构中操作来偏向选择}

\section{基底神经节的行为选择受目标导向和习惯控制}

\section{基底神经节疾病可能与选择障碍有关}
\subsection{选择机制可能容易受到多种潜在故障的影响}
\subsection{帕金森病可以部分地视为未能选择感觉运动选项}
\subsection{精神分裂症可能与抑制非选择选项的普遍失败有关}
\subsection{注意缺陷多动障碍和图雷特综合症也可能以非选择性选项的侵入为特征}
\subsection{强迫症反映了病态主导选项的存在}
\subsection{成瘾与强化机制和习惯性目标的紊乱有关}

\section{要点}
\subsection{荐读}
\subsection{参考文献}
