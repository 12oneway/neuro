\chapter{小脑}



% 参考:https://www.dxy.cn/bbs/newweb/pc/post/40268362

\section{小脑损伤导致明显的症状和体征}
\subsection{损伤导致运动和姿势的特征性异常}
\subsection{损伤会影响特定的感觉和认知能力}

\section{小脑通过其他大脑结构间接控制运动}
\subsection{小脑是一个大的皮层下脑结构}
\subsection{小脑通过循环回路与大脑皮层相连}
\subsection{不同的运动由纵向功能区控制}

\section{小脑皮层由具有相同基本微电路的重复功能单元组成}
\subsection{小脑皮层分为三个功能专门层}
\subsection{攀缘纤维和苔藓纤维传入系统编码和处理信息的方式不同}
\subsection{小脑微电路架构建议进行规范计算}

\section{假设小脑执行几种一般计算功能}
\subsection{小脑有助于前馈感觉运动控制}
\subsection{小脑结合了运动装置的内部模型}
\subsection{小脑整合感觉输入和必然放电}
\subsection{小脑有助于时间控制}

\section{小脑参与运动技能学习}
\subsection{在几种不同的运动系统中,小脑是运动学习所必需的}
\subsection{学习发生在小脑的几个部位}

\section{要点}
\subsection{选读}
\subsection{参考文献}

