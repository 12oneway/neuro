\chapter{凝视的控制}
% PDF所在目录: /data2/whd/win10/learn/neuro/neuro_神经科学原理_28_中枢神经系统的听觉处理.pdf

\section{眼球被六块眼外肌所移动}
\subsection{眼球运动使眼球在轨道上旋转}
\subsection{六块眼外肌形成三个主动-拮抗对}
\subsection{两只眼睛的运动是协调的}
\subsection{眼外肌由三个颅神经控制}

\section{六种神经元控制系统保持目标的瞄准}
\subsection{主动固定系统将中央凹固定在固定目标上}
\subsection{扫视系统将中央凹指向感兴趣的对象}

\section{扫视的运动环路位于脑干}
\subsection{桥脑网状结构产生水平扫视}
\subsection{中脑网状结构中产生垂直扫视}
\subsection{脑干病变导致眼球运动的特征性缺陷}

\section{扫视通过上丘由大脑皮层进行控制}
\subsection{上丘将视觉和运动信息整合到脑干的动眼神经信号中}

