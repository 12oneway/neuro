\chapter{膜电位和神经元的被动电特性}
% PDF所在目录: /data2/whd/win10/learn/neuro/neuro_神经科学原理_28_中枢神经系统的听觉处理.pdf


\section{跨细胞膜的电荷分离产生静息膜电位}

\section{静息膜电位由非门控和门控离子通道决定}
\subsection{神经胶质细胞中的开放通道仅可渗透钾}
\subsection{静息神经细胞中的开放通道可渗透三种离子}
\subsection{钠、钾和钙的电化学梯度是由离子的主动传输建立的}
\subsection{氯离子也被主动运输}

\section{静息膜中离子通量的平衡在动作电位期间被取消}

\section{不同离子对静息膜电位的贡献可以通过 Goldman 方程量化}

\section{神经元的功能特性可以表示为等效电路}

\section{神经元的被动电特性影响电信号}
\subsection{薄膜电容减缓了电信号的时程}
\subsection{膜和细胞质电阻影响信号传导的效率}
\subsection{大轴突比小轴突更容易兴奋}
\subsection{被动膜特性和轴突直径影响动作电位传播速度}

\section{亮点}

\section{选读}

\section{参考文献}







