\chapter{低层视觉处理\_视网膜}
% PDF所在目录: /data2/whd/win10/learn/neuro/neuro_神经科学原理_28_中枢神经系统的听觉处理.pdf

\section{光感层对视觉图像进行采样}
\subsection{眼科光学限制了视网膜图像的质量}
\subsection{有两种类型的光感受器:杆状和锥状}

\section{光转导将光子的吸收与膜电导的变化联系起来}
\subsection{光激活光感受器中的色素分子}
\subsection{兴奋的视紫红质通过 G 蛋白转导蛋白激活磷酸二酯酶}
\subsection{多重机制关断级联}
\subsection{光转导缺陷导致疾病}

\section{神经节细胞将神经图像传输到大脑}
\subsection{神经节细胞的两种主要类型是ON细胞和OFF细胞}
\subsection{许多神经节细胞对图像中的边缘反应强烈}
\subsection{神经节细胞的输出强调刺激的时间变化}
\subsection{视网膜输出强调移动物体}
\subsection{几种神经节细胞类型通过平行通路投射到大脑}

\section{中间神经元网络塑造视网膜输出}
\subsection{平行通路起源于双极细胞}
\subsection{空间过滤是通过横向抑制实现的}
\subsection{时间过滤发生在突触和反馈电路中}
\subsection{色觉始于锥形选择性电路}
\subsection{先天性色盲有多种形式}
\subsection{杆状和锥状电路在视网膜内部合并}

\section{视网膜的灵敏度适应光照的变化}
\subsection{光适应在视网膜处理和视觉感知中很明显}
\subsection{多重增益控制发生在视网膜内}
\subsection{光适应改变空间处理}

\section{亮点}
\section{选读}
\section{参考文献}