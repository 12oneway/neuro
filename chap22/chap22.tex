\chapter{低层视觉处理\_视网膜} \label{chap:chap22}

视网膜是大脑观察世界的窗口。 所有的视觉体验都基于眼睛中这个神经回路处理的信息。 视网膜的输出仅通过一百万个视神经纤维传送到大脑,但几乎一半的大脑皮层用于处理这些信号。 由于设计或缺陷而丢失在视网膜中的视觉信息永远无法恢复。 由于视网膜处理对可以看到的内容设置了基本限制,因此人们对了解视网膜的功能非常感兴趣。

从表面上看,脊椎动物的眼睛看起来很像照相机。 瞳孔形成可变孔径,角膜和晶状体提供折射光学器件,将外部世界的小图像投射到眼球后部的光敏视网膜上(图 22-1)。 但这就是类比结束的地方。 视网膜是一层薄薄的神经元,厚几百微米,由五种主要细胞类型组成,这些细胞排列在由两个突触层隔开的三个细胞层中(图 22-2)。

最外层的感光细胞吸收光并将其转化为神经信号,这一过程称为光转导。 这些信号通过突触传递到双极细胞,双极细胞又连接到最内层的视网膜神经节细胞。 视网膜神经节细胞是视网膜的输出神经元,它们的轴突形成视神经。 除了从感觉神经元到输出神经元的这种直接通路外,视网膜回路还包括许多由外部突触层中的水平细胞和内部突触层中的无长突细胞提供的横向连接(图 22-3)。

视网膜回路执行低级视觉处理,这是视觉图像分析的初始阶段。 它从眼睛的原始图像中提取某些空间和时间特征,并将它们传送到更高的视觉中心。 这种处理的规则适应环境条件的变化。 特别是,视网膜必须调整其对不断变化的照明条件的敏感度。 尽管在每天的过程中遇到的光强度范围很大,但这种适应使我们的视力或多或少保持稳定。

在本章中,我们依次讨论了视网膜功能的三个重要方面:光转导、预处理和适应。 我们将说明实现它们的神经机制及其对视觉感知的影响。



\section{光感层对视觉图像进行采样}


\subsection{眼科光学限制了视网膜图像的质量}
视网膜图像的清晰度由几个因素决定:瞳孔孔径处的衍射、角膜和晶状体的屈光不正以及光路中材料引起的散射。 外界的一个点通常会聚焦在视网膜上一个模糊的小圆圈中。 与其他光学设备一样,这种模糊在光轴附近最小,图像质量接近光瞳衍射所施加的极限。 远离轴,由于角膜和晶状体的像差,图像会显着退化,并且可能会因光散射白内障或近视等屈光不正等异常情况而进一步退化。

靠近光轴的视网膜区域,中央凹,是视力最敏锐的地方,对应于我们指向我们关注的物体的注视中心。 感光细胞、双极细胞和神经节细胞的密度在中央凹处最高(图 22-1B)。 感光器之间的间距与光学模糊圈的大小非常匹配,因此以理想的方式对图像进行采样。 光在到达感光器之前通常必须穿过几层细胞,但在称为中央凹的中央凹中心,其他细胞层被推到一边以减少光散射造成的额外模糊(图 22-1B)。 最后,眼睛后部衬有一层黑色色素上皮,它吸收光线并防止光线散射回眼睛。

视网膜包含另一个特殊部位,即视盘,视网膜神经节细胞的轴突会聚并延伸穿过视网膜,从眼睛后部出现,成为视神经(图 22-1A)。 必然地,该区域没有光感受器,因此对应于每只眼睛视野中的盲点。 由于圆盘位于每只眼睛的中央凹的鼻侧,来自一个点的光不会同时落在两个盲点上,因此通常我们不会注意到它们。 我们可以只用一只眼睛体验盲点(图 22-4)。 盲点展示了盲人的体验——不是黑暗,而是什么都没有。 这就解释了为什么周围视网膜的损伤经常被忽视。 通常是由于意外事故,例如撞到未引起注意的物体,或通过临床测试发现视力缺陷。

盲点是视网膜由内而外设计的必然结果,这让几代生物学家感到困惑和好笑。 该组织的目的可能是使光感受器与视网膜色素上皮细胞紧密并置,这在视网膜色素更新和通过吞噬作用回收光感受器膜中起着重要作用。

\subsection{有两种类型的光感受器:杆状和锥状}

所有的感光细胞都有一个共同的结构,有四个功能区域:外节,位于神经视网膜的远端表面; 内段,位于更近端; 细胞体; 和突触末端(图 22-5A)。

大多数脊椎动物有两种类型的光感受器,即视杆细胞和视锥细胞,以它们的形态来区分。 杆有一个长的圆柱形外段,其中圆盘堆叠与质膜分开,而圆锥通常有一个较短的锥形外段,圆盘与外膜连续(图 22-5B)。

视杆细胞和视锥细胞在功能上也有所不同,最重要的是它们对光的敏感性。 视杆细胞可以发出吸收单个光子的信号,并负责在月光等昏暗光照下的视觉。 但是随着接近黎明的光照水平增加,杆状细胞的电反应变得饱和,细胞停止对强度变化做出反应。 视锥细胞对光的敏感度要低得多; 它们对夜视没有贡献,但只负责白天的视力。 它们的反应比杆状体快得多。 灵长类动物只有一种杆状体,但有三种锥形光感受器,根据它们响应的波长范围来区分:L(长波)、M(中波)和 S(短波)锥体(图 22-6)。

人类视网膜包含大约 1 亿个视杆细胞和 500 万个视锥细胞,但这两种细胞类型的分布不同。 中央凹不包含杆状体,但密集地排列着小锥体。 在中央凹外几毫米处,视杆细胞的数量远远多于视锥细胞。 所有的光感受器都变得更大,并向视网膜周边分布得更广。 S 锥体仅占所有锥体的 10\%,并且不存在于中央凹中。

视网膜注视中心显然专用于白天视觉。 中央凹中锥形光感受器的密集堆积设定了我们视力的极限。 事实上,我们可以在医生的视力表上读到的最小字母具有笔画,其图像在视网膜上只有一到两个锥体直径宽,视角约为 1 分弧(图 22-1C)。 到了晚上,由于没有视杆细胞,中央凹是盲的。 天文学家知道,必须只看一颗昏暗恒星的侧面才能看到它。 夜间在森林里散步时,我们这些非天文学家往往会跟随白天的反射,直视可疑声音的来源。 神秘的是,这个物体消失了,只是在我们移开视线时跳回了我们的周边视野。

\section{光转导将光子的吸收与膜电导的变化联系起来}
与许多其他神经元一样,光感受器的膜电位受膜电导与 Na+ 和 K+ 离子平衡的调节,其跨膜梯度由代谢活性泵维持(第 9 章)。 在黑暗中,Na+ 离子通过非选择性阳离子通道流入光感受器,该通道由第二信使环鸟苷 3'-5' 单磷酸 (cGMP) 激活。

色素蛋白对光子的吸收启动了生化级联反应,最终降低了 cGMP 的浓度,从而关闭了 cGMP 门控通道,使细胞更接近 K+ 平衡电位。 通过这种方式,光使感光器超极化(图 22-7)。 在这里,我们详细描述了这一系列事件。 这些知识大部分来自对杆状细胞的研究,但锥体中的机制非常相似。

\subsection{光激活光感受器中的色素分子}
视紫红质是视杆细胞中的视觉色素,有两种成分。 蛋白质部分视蛋白嵌入椎间盘膜中,本身不吸收可见光。 吸光部分,视黄醛,是一种小分子,其 11-顺式异构体与视蛋白的赖氨酸残基共价连接(图 22-8A)。 视网膜对光子的吸收导致它从 11-cis 翻转到全反式构型。 这种反应是视觉中唯一依赖光的步骤。

视网膜分子形状的变化导致视蛋白的构象变化为激活状态,称为变视紫红质 II,从而触发光转导的第二步。 Metarhodopsin II 不稳定,会在几分钟内分裂,产生视蛋白和游离的全反式视网膜。 然后全反式视黄醛从视杆细胞运送到色素上皮细胞,在那里它被还原为全反式视黄醇(维生素 A),即 11-顺式视黄醛的前体,随后被运回视杆细胞。

因此,全反式视网膜是视觉系统中的重要化合物。 它的前体,如维生素 A,不能由人类合成,因此必须作为日常饮食的一部分。 维生素 A 缺乏会导致夜盲症,如果不加以治疗,还会导致受体外节的退化并最终导致失明。

人类视网膜中的每种视锥细胞都会产生一种视蛋白变体。 这三种锥形颜料的区别在于它们的吸收光谱,即光吸收效率对波长的依赖性(见图 22-6)。 光谱由蛋白质序列通过结合口袋附近的视黄醛和某些氨基酸侧链之间的相互作用来确定。 红光比 M 视锥细胞更能激发 L 视锥细胞,而绿光更能激发 M 视锥细胞。 因此,这些锥体类型中的相对激发度包含有关光的光谱的信息,与其强度无关。 大脑对来自不同视锥细胞类型的信号进行比较是色觉的基础。

在夜视中,只有视杆细胞处于活动状态,因此所有功能性光感受器都具有相同的吸收光谱。 因此,绿光对视觉系统的影响与强度更高的红光完全相同。 因为单光感受器系统无法区分光的光谱和强度,所以“晚上所有的猫都是灰色的”。 通过比较杆对不同波长光的敏感度,可以获得视紫红质的吸收光谱。 一个了不起的事实是,只需向人类受试者询问各种彩色光的外观,就可以准确地测量这种分子特性(图 22-9)。 知觉或心理物理学的定量研究为大脑处理的其他机制提供了类似的见解(第 17 章)。

\subsection{兴奋的视紫红质通过 G 蛋白转导蛋白激活磷酸二酯酶}
活化的视紫红质以变视紫红质 II 的形式扩散到视盘膜内,在那里它遇到转导蛋白,G 蛋白家族的一员(第 14 章)。 与其他 G 蛋白的情况一样,无活性形式的转导蛋白结合鸟苷二磷酸 (GDP) 分子。 与 metarhodopsin II 的相互作用促进 GDP 与三磷酸鸟苷 (GTP) 的交换。 这导致转导蛋白的亚基解离为携带 GTP (Tα-GTP) 以及 β 和 γ 亚基 (Tβγ) 的活性 α 亚基。 Metarhodopsin II 可以激活数百个额外的转导蛋白分子,从而显着增强细胞的反应。

活性转导蛋白亚基 Tα-GTP 与环核苷酸磷酸二酯酶(另一种与椎间盘膜相关的蛋白质)形成复合物。 这种相互作用大大提高了酶将 cGMP 水解为 5'-GMP 的速率。 每个磷酸二酯酶分子每秒可水解1000多个cGMP分子,从而增加放大程度。

cGMP 的浓度控制外段质膜中 cGMP 门控通道的活性。 在黑暗中,当 cGMP 浓度高时,大量 Na+ 通过开放通道流入,使细胞保持在大约 –40 mV 的去极化水平。 结果,细胞的突触末端不断释放递质谷氨酸。 光诱发的 cGMP 减少导致 cGMP 门控通道关闭,从而减少 Na+ 离子的内流并使细胞超极化(图 22-7B1)。 超极化减缓了光感受器末端神经递质的释放,从而启动了神经信号。


\subsection{多重机制关断级联}
感光器对单个光子的反应必须终止,这样细胞才能对另一个光子做出反应。 Metarhodopsin II 通过特定视紫红质激酶的磷酸化作用失活,然后结合可溶性蛋白抑制蛋白,从而阻断与转导蛋白的相互作用。

活性转导蛋白 (Tα-GTP) 具有内在的 GTP 酶活性,最终将结合的 GTP 转化为 GDP。 然后 Tα-GDP 释放磷酸二酯酶并与 Tβγ 重组,再次准备好被视紫红质激发。 一旦磷酸二酯酶失活,cGMP 浓度就会通过从 GTP 产生 cGMP 的鸟苷酸环化酶恢复。 此时,膜通道打开,Na+ 电流恢复,光感受器去极化回到其暗电位。

除了这些关闭级联的各个元素的独立机制之外,重要的反馈机制可确保更快地终止大量响应。 这是由细胞中 Ca2+ 浓度的变化介导的。 钙离子通过 cGMP 门控通道进入细胞,并被快速阳离子交换剂排出。 在黑暗中,细胞内 Ca2+ 浓度较高,但在细胞对光反应期间,当 cGMP 门控通道关闭时,Ca2+ 水平迅速下降至黑暗水平的百分之几。

Ca2+ 浓度的降低以三种方式调节生化反应(图 22-7B2)。 钙结合蛋白恢复素对视紫红质激酶的作用可加速视紫红质磷酸化,从而减少转导蛋白的激活。 钙依赖性鸟苷酸环化酶激活蛋白可加速鸟苷酸环化酶的活性。 最后,cGMP 门控通道对 cGMP 的亲和力通过 Ca2+-钙调蛋白的作用而增加。 所有这些作用促进光感受器返回到黑暗状态。

\subsection{光转导缺陷导致疾病}
毫不奇怪,光转导机制中的缺陷会产生严重的后果。 一个突出的缺陷是色盲,这是由于视锥细胞色素基因缺失或异常引起的,稍后将对此进行讨论。

当视杆细胞功能丧失但视锥细胞功能保持完好时,就会导致静止性夜盲症。 这种疾病是可遗传的,并且已经在光转导级联的许多组分中发现了突变:视紫红质、杆状转导蛋白、杆状磷酸二酯酶、视紫红质激酶和抑制蛋白。 在某些情况下,视杆细胞似乎会永久激活,就好像暴露在持续的致盲光下一样。

不幸的是,许多光转导缺陷会导致色素性视网膜炎,这是一种视网膜进行性退化,最终导致失明。 该疾病有多种形式,其中许多与影响视杆细胞信号转导的突变有关。 为什么这些功能变化会导致视杆细胞死亡以及随后的视锥细胞退化尚不清楚。


\section{神经节细胞将神经图像传输到大脑}
感光层产生视觉场景的相对简单的神经表征:明亮区域的神经元超极化,而黑暗区域的神经元去极化。 由于视神经的轴突数量仅为受体细胞数量的 1\%,因此视网膜回路必须在将信息传送到大脑之前编辑光感受器中的信息。

此步骤构成低级视觉处理,是从落在视网膜上的光模式中获得视觉感知的第一阶段。 要了解这个过程,我们必须首先了解视网膜输出的组织以及视网膜神经节细胞如何响应各种光模式。


\subsection{神经节细胞的两种主要类型是ON细胞和OFF细胞}
许多视网膜神经节细胞即使在黑暗或持续光照下也会自发地激发动作电位。 如果光强度突然增加,所谓的 ON 细胞会更快地放电。 其他神经节细胞,OFF 细胞,放电更慢或完全停止放电。 当强度再次降低时,ON 细胞的放电减少而 OFF 细胞的放电增加。 因此,视网膜输出包括两个互补的表示,它们对光的反应极性不同。

这种布置用于快速传达视觉场景中的变亮和变暗。 如果视网膜只有 ON 细胞,黑暗物体将通过降低放电率来编码。 如果神经节细胞以每秒 10 个脉冲的维持速率放电,然后降低其速率,则突触后神经元需要大约 100 毫秒才能注意到动作电位频率的变化。 相比之下,发射率增加到每秒 200 个尖峰仅在 5 毫秒内就很明显。

\subsection{许多神经节细胞对图像中的边缘反应强烈}
为了更详细地探测神经节细胞的反应,可以测试细胞的放电如何随着聚焦在视网膜不同部分的小光点的位置和时间过程而变化。

典型的神经节细胞对靠近细胞体的视网膜紧凑区域(称为细胞感受野)中的光敏感。 在那个区域内,人们通常可以区分中心区域和周围区域,光在细胞中产生相反的反应。 例如,当亮点聚焦在细胞的感受野中心时,ON 细胞会发射得更快,但当亮点聚焦在周围时会减少发射。 如果光线覆盖中心和周围,则响应比仅中心照明弱得多。 中心的亮点与覆盖周围的黑色环形相结合,会引发非常强烈的射击。 对于 OFF 电池,这些关系是相反的; 细胞被暗点和亮环强烈激发(图 22-10)。

因此,由一群视网膜神经节细胞产生的输出增强了输入中的空间对比度区域,例如两个不同强度区域之间的边缘,并且不太强调均匀照明区域。

\subsection{神经节细胞的输出强调刺激的时间变化}
当出现有效的光刺激时,神经节细胞的放电通常会从静息水平急剧增加到峰值,然后放松到中间速率。 当刺激关闭时,放电率急剧下降,然后逐渐恢复到静止水平。

从峰值到静息水平下降的速度因神经节细胞类型而异。 瞬时神经元仅在刺激开始时产生尖峰脉冲,而持续神经元在刺激期间保持几乎稳定的放电率几秒钟(图 22-10)。

然而,一般来说,神经节细胞的输出有利于视觉输入在恒定光强度期间的时间变化。 事实上,当图像通过眼动追踪设备稳定在视网膜上时,它会在几秒钟内从视野中消失。 幸运的是,这在正常视力中永远不会发生; 即使当我们试图固定我们的视线时,小的自动眼球运动(扫视)也会持续扫描视网膜上的图像并防止世界消失(第 25 章)。

\subsection{视网膜输出强调移动物体}
基于这些观察,我们可以更普遍地理解神经节细胞对视觉输入的反应。 例如,移动物体的边缘会在神经节细胞群中引起强烈的放电,因为这些是空间对比的唯一区域,也是光强度随时间变化的唯一区域(图 22-11)。

我们很容易理解为什么视网膜会选择性地响应这些特征。 物体的轮廓对于推断其形状和身份特别有用。 同样,突然移动或改变的物体比那些不移动或改变的物体更值得立即关注。 因此,视网膜处理提取了有助于指导行为的场景低级特征,并有选择地将这些特征传输到大脑。 事实上,拒绝在空间或时间上保持不变的特征解释了人类感知的时空敏感性(方框 22-1)。

\subsection{几种神经节细胞类型通过平行通路投射到大脑}
根据它们的形态和对光的反应,已经鉴定出几种不同类型的神经节细胞。 ON 和 OFF 细胞出现在每个脊椎动物的视网膜中,而在灵长类动物的视网膜中,两大类细胞,P 细胞和 M 细胞,分别包括 ON 和 OFF 类型(见图 22-2B)。 在距中央凹的任何给定距离处,M 细胞(拉丁语 magno,大)的感受野远大于 P 细胞(拉丁语 parvo,小)的感受野。 M 细胞也比 P 细胞具有更快和更瞬态的反应。 由于视觉色素黑视蛋白的表达,一些神经节细胞本质上对光敏感。

总共描述了 20 多种类型的神经节细胞。 每种类型的细胞群以平铺方式覆盖视网膜,使得视网膜上的任何点位于至少一个神经节细胞的感受野中心内。 可以设想来自每个群体的信号一起向大脑发送视野的独特神经表征。 在这个观点中,视神经传达了 20 种或更多的神经表征,它们在极性(开或关)、空间分辨率(精细或粗糙)、时间响应(持续或瞬态)、光谱过滤(宽带或由红色、绿色、 或蓝色),以及对其他图像特征(例如运动)的选择性。

这些神经表征指向大脑中的各种视觉中心,包括丘脑的外侧膝状体核,它是视觉皮层的中继; 上丘,一个涉及空间注意力和定向运动的中脑区域; pretectum,参与对瞳孔的控制; 辅助光学系统,分析自身运动以稳定凝视; 和视交叉上核,一个指导昼夜节律的中央时钟,其相位可以通过光提示设置(第 44 章)。 在许多情况下,一种神经节细胞的轴突将侧枝延伸到中枢神经系统的多个区域。 例如,M 细胞投射到丘脑和上丘。


\section{中间神经元网络塑造视网膜输出}
我们现在更详细地考虑视网膜回路以及它如何解释视网膜神经节细胞的复杂反应特性。

\subsection{平行通路起源于双极细胞}
光感受器与双极细胞和水平细胞形成突触(见图 22-3A)。 在黑暗中,光感受器的突触末端不断释放谷氨酸。 当受光刺激时,光感受器超极化,较少的钙进入末端,末端释放较少的谷氨酸。 光感受器不发射动作电位; 像双极细胞一样,它们使用特殊结构(带状突触)以分级方式释放神经递质。 事实上,大多数视网膜处理是通过分级膜电位完成的:动作电位主要发生在某些无长突细胞和视网膜神经节细胞中。

双极细胞的两种主要变体,ON 细胞和 OFF 细胞,通过不同的机制对突触处的谷氨酸作出反应。 OFF 细胞使用离子型受体,即 AMPA-红藻氨酸变体(AMPA = α-amino-3-hydroxy-5-methylisoxazole-4-propionate)的谷氨酸门控阳离子通道。 在黑暗中释放的谷氨酸使这些细胞去极化。 ON 细胞使用与 G 蛋白相连的促代谢受体,G 蛋白的作用最终会关闭阳离子通道。 因此,这些受体的谷氨酸激活使黑暗中的细胞超极化。

双极 ON 和 OFF 细胞的形状不同,尤其是在轴突终止的内丛状层内的水平上。 ON 细胞的轴突终止于近端(下)一半,而 OFF 细胞的轴突终止于远端(上)一半(图 22-15)。 在那里,它们在无长突细胞和神经节细胞的树突上形成特定的突触连接。 ON 双极细胞激发 ON 神经节细胞,而 OFF 双极细胞激发 OFF 神经节细胞(见图 22-3A)。 因此,视网膜输出的两个主要细分,ON 和 OFF 通路,已经在双极细胞水平上建立。

双极细胞也可以通过其树突的形态来区分(图 22-15)。 在灵长类动物视网膜的中央区域,小型双极细胞接收来自单个视锥细胞的输入并激发 P 型神经节细胞。 这解释了为什么 P 细胞感受野的中心如此之小。 弥漫性双极细胞接收来自许多视锥细胞的输入并激发 M 型神经节细胞。 因此,M 细胞的感受野中心要大得多。 因此,神经节细胞群中的刺激表征起源于专用的双极细胞通路,这些通路的区别在于它们与光感受器和突触后目标的选择性连接。

\subsection{空间滤波是通过横向抑制实现的}
平行开和关通路中的信号通过与水平细胞和无长突细胞的相互作用而改变(参见图 22-3A)。 水平细胞具有广泛的树枝状树突,这些树突在外丛状层中横向分布。 光感受器在与双极细胞共享的谷氨酸能末端接触这些乔木的尖端。 此外,水平单元通过间隙连接彼此电耦合。

水平细胞有效地测量了广阔区域中光感受器群体的平均激发水平。 该信号通过抑制性突触反馈到感光器终端。 因此,感光器终端受到两种相反的影响:落在受体上的光使其超极化,但落在周围区域的光通过水平细胞的符号反转突触使其去极化。 因此,双极细胞具有拮抗的感受野结构。

感受野中的这种空间对抗通过内部视网膜中无长突细胞的侧向抑制得到增强。 无长突细胞是神经元,其突起仅在内丛状层中分支。 大约有 30 种类型的无长突细胞是已知的,一些具有只有几十微米宽的小乔木,而另一些则具有延伸到整个视网膜的突起。 无长突细胞通常在谷氨酸能突触处接收来自双极细胞的兴奋信号。 一些无长突细胞在相互抑制性突触处直接反馈给突触前双极细胞。 一些无长突细胞与其他相同类型的无长突细胞电耦合,形成与水平细胞非常相似的电网络。

通过这个抑制网络,双极细胞末端可以接受来自远处双极细胞的抑制,其方式与光感受器末端的横向抑制非常相似(见图 22-3A)。 无长突细胞也直接抑制视网膜神经节细胞。 这些横向抑制连接对视网膜神经节细胞的拮抗性感受野成分有很大贡献。

\subsection{时间滤波发生在突触和反馈电路中}
对于许多神经节细胞,光强度的阶跃变化会产生瞬态响应,放电的初始峰值会下降到较小的稳定速率(见图 22-10)。 这种敏感性部分源于涉及水平细胞和无长突细胞的负反馈电路。 例如,光强度的突然降低使锥体末端去极化,从而激发水平细胞,进而使锥体末端重新极化(见图 22-3A)。 因为这个反馈回路涉及一个短暂的延迟,锥体的电压响应突然达到峰值,然后稳定到一个较小的稳定水平。 类似的过程发生在视网膜内部双极细胞和无长突细胞之间的相互突触处。

在这两种情况下,延迟抑制电路有利于快速变化的输入而不是缓慢变化的输入。 这种过滤的效果可以在视觉感知中观察到,对于最有效地驱动水平和无长突细胞网络的大刺激最为明显。 例如,当一个大光点以 10 Hz 的频率闪烁时很容易看到,但以低频率闪烁时却看不到(见图 22-14)。

除了这些电路特性外,某些细胞过程还有助于塑造时间响应。 例如,AMPA-红藻氨酸类型的谷氨酸受体会经历强烈的脱敏作用。 双极或神经节细胞树突处谷氨酸浓度的逐步增加导致额外的谷氨酸受体立即打开。 随着这些受体脱敏,突触后电导再次降低。 效果是使阶跃响应更加瞬态。

视网膜回路似乎竭尽全力加快他们的反应并强调时间变化。 一个可能的原因是视网膜回路中的第一个细胞,即光感受器,异常缓慢(见图 22-7C)。 在一道闪光之后,视锥细胞需要大约 40 毫秒才能达到峰值响应,这对于正常的视觉功能来说是无法忍受的延迟。 通过视网膜回路中的各种过滤机制,后续神经元在视锥细胞反应的上升阶段反应最强烈。 事实上,一些神经节细胞在闪光后仅 20 毫秒就有一个响应峰值。 视网膜中的时间处理显然有助于减少视觉反应时间,这是一种延长寿命的特性,在高速公路交通中和在我们祖先的热带草原上一样重要。

\subsection{色觉始于锥形选择性电路}
纵观有记载的历史,哲学家和科学家一直对色彩感知着迷。 这种兴趣最初是由颜色与艺术的相关性驱动的,后来是由它与光的物理特性的关系驱动的,最后是由电视和摄影中的商业利益驱动的。 19 世纪出现了大量解释色彩感知的理论,其中有两个在现代审查中幸存下来。 它们基于仔细的心理物理学,对潜在的神经机制施加了强烈的限制。

早期的实验表明,任何给定的自然光都可以通过将适量的三种原色光混合在一起来进行颜色匹配。 这导致了基于三种机制对光的吸收的颜色感知的三原色理论,每种机制具有不同的灵敏度光谱。 这些对应于三种锥体类型(见图 22-6),其测量的吸收光谱充分解释了正常个体和色素基因遗传异常者的配色结果。

提出了所谓的对立过程理论来解释我们对不同色调的感知。 根据这一理论,色觉涉及三个过程,它们对不同颜色的光以相反的方式做出反应:(y–b) 会被黄色光刺激并被蓝光抑制; (r–g) 受红色刺激,受绿色抑制; 和 (w–bk) 受白色刺激,受黑色抑制。 我们认识到视网膜后感受器回路中的一些 19 世纪假设。

在人类视网膜的中央 10°,接收来自单个视锥细胞输入的单个小型双极细胞会激发每个 P 型神经节细胞。 例如,L-ON 神经节细胞的感受野中心由单个 L 锥体和包含 L 和 M 锥体混合的拮抗环绕组成。 当这个神经元的感受野受到一个延伸到中心和周围的大而均匀的光点的刺激时,这个神经元被红光去极化并被绿光超极化。 类似的拮抗作用适用于其他三个 P 细胞:L-OFF、M-ON 和 M-OFF。 这些 P 细胞将它们的信号发送到外侧膝状体核的小细胞层。

一种专用类型的 S-ON 双极细胞选择性地收集 S 锥体的信号,并将它们传输到小双层型神经节细胞。 因为这个神经节细胞也接受来自 L-OFF 和 M-OFF 双极细胞的激发,所以它被蓝光去极化并被黄光超极化。 另一种类型的神经节细胞显示出相反的特征:S-OFF 和 (L + M)-ON。 这些信号被传输到外侧膝状体核的 koniocellular 层。

M 细胞被弥漫性双极细胞激发,这些细胞反过来收集来自许多视锥细胞的输入,无论色素类型如何。 因此,这些神经节细胞具有宽光谱灵敏度的大感受野。 它们的轴突投射到外侧膝状体核的大细胞层。

以这种方式,彩色信号被视网膜组合和编码,以传输到丘脑和皮质。 在初级视觉皮层的电路中,这些信号以不同的方式重新组合,从而导致各种各样的感受野布局。 只有大约 10\% 的皮层神经元优先受颜色对比而不是亮度对比的驱动。 这可能反映了这样一个事实,即色觉——尽管它具有巨大的审美吸引力——对我们的整体健康只做出了很小的贡献。 作为这方面的一个例子,回想一下色盲个体,他们在某种意义上已经失去了一半的色彩空间,可以在成长过程中从未注意到这一缺陷。

\subsection{先天性色盲有多种形式}
很少有人是真正的色盲,即完全无法区分颜色的变化和光强的变化,但许多人的色觉受损,并且在区分对我们大多数人来说微不足道的东西时遇到困难,例如 在红色和绿色之间。 大多数此类色觉异常是先天性的,并且已被详细描述; 其他一些异常是由视觉通路的损伤或疾病引起的。

有些人只有两类视锥细胞而不是三类。 这些二色视者发现很难或不可能区分某些颜色对三色视者来说明显不同的表面。 二色视者的问题是每个表面反射函数都由二值描述而不是三值描述表示,这种简化的描述导致二色视者比三色视者混淆更多的表面。 色盲的简单测试利用了这一事实(图 22-16)。

虽然存在三种形式的双色性,对应于三种类型视锥细胞中每一种的损失,但有两种比第三种更常见。 常见形式对应于 L 视锥细胞或 M 视锥细胞的丧失,分别称为红色盲和绿色盲。 红色盲和绿色盲几乎总是发生在男性身上,各占 1\% 左右。 这些病症由自身未受影响的女性传播,因此与 X 染色体上的基因有关。 第三种形式的双色性,蓝色盲,涉及 S 锥体的丧失或功能障碍。 它仅影响大约 10,000 人中的 1 人,以相同的频率折磨女性和男性,并且涉及 7 号染色体上的一个基因。

由于 L 和 M 视锥细胞大量存在,人们可能会认为其中一种或另一种视锥细胞的缺失会对视力造成更广泛的损害,而不仅仅是削弱色觉。 事实上,这不会发生,因为二色视视网膜中 L 和 M 视锥细胞的总数没有改变。 所有注定要成为 L 或 M 视锥细胞的细胞可能在绿色盲中转化为 L 视锥细胞,在红色盲中转化为 M 视锥细胞。

除了以二色性为代表的相对严重的色盲形式外,还有较温和的形式,同样主要影响男性。 这些所谓的异常三色视者的视锥细胞的光谱灵敏度与正常三色视者不同。 异常三色性是由于一种正常的视锥细胞色素被具有不同光谱灵敏度的改变蛋白质所取代所致。 两种常见形式,protanomaly 和 deuteranomaly,共同影响大约 7\% 的男性,分别代表 L 或 M 视锥细胞被具有某种中间光谱敏感性的色素取代。

色觉缺陷的遗传学已广为人知。 L 和 M 色素的基因以头尾排列的形式位于 X 染色体上(图 22-17A)。 色素蛋白具有非常相似的结构,仅 4\% 的氨基酸不同。 具有正常色觉的人拥有 L 色素基因的单个拷贝,以及 1 到 3 个——有时多达 5 个——几乎相同的 M 色素基因拷贝。

这些基因的接近性和相似性使它们易于发生各种形式的重组,导致基因丢失或形成杂交基因,从而导致常见的红-绿缺陷形式(图 22-17B)。 对二色视者中这些基因的检查揭示了红色盲中 L 色素基因的缺失和绿色盲中一种或多种 M 色素基因的缺失。 异常三色视者具有 L-M 或 M-L 混合基因,这些基因编码具有偏移光谱灵敏度的视觉色素; 转变的程度取决于重组点。 在 tritanopes 中,S 锥体功能的丧失是由 S 色素基因突变引起的。

\subsection{杆状和锥状电路在视网膜内部合并}
对于弱光条件下的视觉,哺乳动物视网膜有一个 ON 双极细胞,专门连接到视杆(见图 22-3B)。 通过收集来自多达 50 个视杆的输入,这种视杆双极细胞可以将分散的单光子吸收效应集中在一小块视网膜中。 没有相应的专用于杆的 OFF 双极电池。

与所有其他双极细胞不同,杆状双极细胞不直接接触神经节细胞,而是激发专用神经元,即 AII 无长突细胞。 该无长突细胞接收来自多个杆状双极细胞的输入,并将其输出传送至锥状双极细胞。 它通过间隙连接向 ON 双极细胞提供兴奋信号,并向 OFF 双极细胞提供甘氨酸能抑制信号。 如前所述,这些锥形双极细胞依次激发 ON 和 OFF 神经节细胞。 因此,棒状信号在绕道后被馈入锥体系统,为开和关路径产生适当的信号极性。 添加的中间神经元的目的可能是允许更多的杆状信号汇集而不是锥状信号。

杆信号也通过另外两条途径进入锥体系统。 杆可以直接通过电接头驱动相邻的锥体,并且它们与主要为锥体服务的 OFF 双极电池连接。 一旦视杆细胞信号通过这些通路到达视锥细胞双极,它就可以利用内部视网膜同样复杂的电路。 因此,哺乳动物视网膜的视杆系统可能是进化后添加到视锥回路中的想法。


\section{视网膜的灵敏度适应光照的变化}
视觉在许多不同的照明条件下运行。 来自物体的光的强度取决于环境照明的强度和物体表面反射的光的比例,称为反射率。 一天中遇到的强度范围很大,变化跨越 10 个数量级,但这种变化的大部分对于指导行为是无用的。

照明强度变化约九个数量级,主要是因为我们的星球每天围绕其轴旋转一次,而物体反射率变化小得多,在典型场景中变化约一个数量级。 但是这种反射率对于视觉来说是一个有趣的量,因为它表征了物体并将它们与背景区分开来。 事实上,我们的视觉系统非常擅长独立于环境光照计算表面反射率(图 22-18)。

随着环境照明的整体增加,视觉场景中的所有点都以相同的因子变亮。 如果眼睛可以简单地将其灵敏度降低相同的因素,则图像的神经表征将在神经节细胞水平上保持不变,并且可以由大脑的其余部分以与光照变化之前相同的方式进行处理。 此外,由于不同的物体反射率,视网膜神经节细胞只需要编码 10 倍范围的图像强度,而不是包括环境照明变化的 100 亿倍范围。 部分灵敏度调整是由瞳孔执行的,瞳孔在强光下收缩,将视网膜照明度降低多达 10 倍。此外,视网膜本身执行自动增益控制,称为光适应,接近理想的正常化 我们在这里想象过。

\subsection{光适应在视网膜处理和视觉感知中很明显}
当不同强度的闪光呈现恒定的背景照明时,视网膜神经节细胞的反应符合 S 形曲线(图 22-19A)。 最弱的闪光不会引起反应,闪光强度的分级增加会引起分级反应,最亮的闪光会引起饱和。 当背景照明增加时,响应曲线保持相同的形状,但转向更高的闪光强度。 为了补偿背景照明的增加,神经节细胞现在对光变化不太敏感:在较高背景下,需要更大的变化才能引起相同的反应。 这种刺激-反应关系的横向移动是视网膜光适应的标志。

这种增益变化对人类视觉感知的影响在心理物理学实验中显而易见。 当人类受试者被要求在恒定照明的背景场中检测闪光时,在更亮的背景上检测需要更亮的闪光(图 22-19B)。 在前面讨论的理想增益控制机制下,如果两个刺激引起背景强度的相同分数变化,它们将产生相同的响应。 在那种情况下,阈值闪光强度应与背景强度成正比,这种关系称为韦伯适应定律,我们在考虑体细胞受体敏感性时遇到过这种关系(第 17 章)。 视觉系统大致遵循韦伯定律:在整个视觉范围内,随着背景强度的增加,灵敏度的下降幅度有所缓和(图 22-19B)。

\subsection{多重增益控制发生在视网膜内}
光适应所需增益的巨大变化出现在视网膜内的多个部位。 在星光下,单个视杆细胞每隔几秒就会受到一个光子的刺激,这个速度不足以改变细胞的适应状态。 然而,视网膜神经节细胞结合来自许多视杆细胞的信号,从而接收稳定的光子信号流,从而在细胞中引起光依赖性增益变化。

在稍高的光照强度下,杆状双极细胞开始适应,根据平均光照水平改变其响应能力。 接下来,我们达到单个杆细胞的增益逐渐降低的光强度。 除此之外,视杆细胞饱和:它们所有依赖于 cGMP 的通道都关闭了,膜电位不再对光刺激做出反应。 到这个时候,大约在黎明时分,敏感度低得多的视锥细胞得到有效刺激,并逐渐取代视杆细胞。 随着环境光进一步增加,接近中午,光适应主要来自视锥细胞内的增益变化。

光感受器的细胞机制在光感受器中得到了最好的理解。 前面讨论的钙依赖性反馈通路具有突出的作用。 回想一下,当闪光关闭 cGMP 门控通道时,细胞内 Ca2+ 的减少会加速几种生化反应,从而终止对闪光的反应(参见图 22-7B)。 然而,当光照持续时,Ca2+ 浓度仍然很低,因此所有这些反应都处于稳定状态,既降低了增益又加速了受体对光的反应的时间进程(图 22-19C)。 因此,光适应感光器可以更快地响应强度的快速变化。 这对人类视觉感知有重要影响; 对高频闪烁的对比敏感度随着强度的增加而增加,这也是在灵长类动物视网膜神经节细胞中观察到的效果(见图 22-14)。

\subsection{光适应改变空间处理}

除了视网膜反应的灵敏度和速度之外,光适应还改变了空间处理的规则。 在明亮的光线下,许多神经节细胞在它们的感受野中有一个尖锐的中心环绕结构(见图 22-10)。 随着光线变暗,对立的环绕声变得宽广而微弱,最终消失。 在这些条件下,视网膜的电路只是简单地积累稀有光子,而不是计算局部强度梯度。 感受野特性的这些变化是由于水平和无长突细胞网络产生的横向抑制的变化而发生的(见图 22-3)。 这些过程的一个重要调节剂是多巴胺,由专门的无长突细胞以依赖光的方式释放。

这些视网膜效应在人类感知上留下了印记。 在明亮的光线下,我们的视觉系统更喜欢精细的光栅而不是粗糙的光栅。 但在昏暗的光线下,我们对粗光栅最敏感:随着中心环绕拮抗作用的丧失,低空间频率不再衰减(见方框 22-1 和图 22-13)。

总之,光照适应有两个重要作用。 一种是丢弃有关环境光强度的信息,同时保留有关物体反射率的信息。 另一个是将视网膜神经节细胞中的小动态放电范围与环境中的大范围光强度相匹配。 这些大的增益变化必须在视神经纤维中产生动作电位之前通过分级神经元信号来完成,因为这些纤维的放电率只能在两个数量级内有效变化。 事实上,光适应的关键需求可能就是为什么这种神经回路位于眼睛而不是视神经另一端的大脑中。

\section{亮点}

1. 视网膜将投射到光感受器上的光模式转化为神经信号,这些信号通过视神经传送到大脑中专门的视觉中心。 不同的神经节细胞群沿着平行通路传输视网膜图像的多种神经表征。 

2. 视网膜丢弃了受体水平上可用的大部分刺激信息,并提取了对中央视觉系统有用的视野的某些低级特征。 精细的空间分辨率仅在凝视中心的狭窄区域中保持。 图像中的强度梯度,例如物体边缘,在空间均匀的部分上得到强调; 场景的不变部分增强了时间变化。 

3. 视网膜能灵活适应视觉条件的变化,尤其是昼夜光照变化较大。 关于绝对光照水平的信息大部分被丢弃,有利于随后对场景内物体反射率的分析。 

4. 当色素分子吸收光子时,光刺激的转导开始于感光细胞的外段。 这启动了一个放大的 G 蛋白级联反应,最终降低了膜电导,使光感受器超极化,并减少了突触处的谷氨酸释放。 多种反馈机制,其中细胞内 Ca2+ 具有重要作用,用于关闭级联中的酶并终止光响应。 

5. 杆状感光器是高效的光收集器并服务于夜间视觉。 视锥细胞的敏感性要低得多,并且全天都在发挥作用。 视锥突触到双极细胞上,进而激发神经节细胞。 视杆连接到专门的视杆双极细胞,其信号通过无长突细胞传递到视锥双极细胞。 

6. 垂直兴奋通路由主要是抑制性的水平连接调节。 通过这些横向网络,神经节细胞感受野周围的光抵消了中心光的影响。 同样的负反馈电路也加强了神经节细胞的瞬态反应。 

7. 将信息隔离到平行通路中以及通过抑制性横向连接形成反应特性是视觉系统中普遍存在的组织原则。


\section{选读}
\section{参考文献}