\chapter{成像和行为}
% PDF所在目录: /data2/whd/win10/learn/neuro/neuro_神经科学原理_28_中枢神经系统的听觉处理.pdf

\section{功能性 MRI 实验测量神经血管活动}

\subsection{fMRI 取决于磁共振物理学}
\subsection{fMRI 依赖于神经血管耦合的生物学}

\section{可以通过多种方式分析功能性 MRI 数据}
\subsection{fMRI 数据首先需要通过以下预处理步骤准备分析}
\subsection{fMRI 可用于将认知功能定位到特定的大脑区域}
\subsection{fMRI 可用于解码大脑中代表的信息}
\subsection{fMRI 可用于测量大脑网络中的相关活动}

\section{功能性 MRI 研究带来了基本的见解}
\subsection{人类的 fMRI 研究启发了动物的神经生理学研究}
\subsection{fMRI 研究挑战了认知心理学和系统神经科学的理论}
\subsection{fMRI 研究检验了动物研究和计算模型的预测}

\section{功能性 MRI 研究需要仔细解读}

\section{未来的进步取决于技术和概念的进步}

\section{亮点}

\section{推荐读物}

\section{参考文献}
