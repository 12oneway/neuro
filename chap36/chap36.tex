\chapter{姿态}
% PDF所在目录: /data2/whd/win10/learn/neuro/neuro_神经科学原理_28_中枢神经系统的听觉处理.pdf

\section{平衡和定向是姿势控制的基础}

\subsection{姿势平衡控制身体的重心}
\subsection{姿势定向能预示平衡障碍}

\section{姿势反应和预期姿势调整使用刻板的策略和协同作用}
\subsection{自动姿势反应补偿突然的干扰}
\subsection{预期姿势调整补偿随意运动}
\subsection{姿势控制与运动相结合}

\section{必须整合和解释体感、前庭和视觉信息以保持姿势}
\subsection{体感信号对于自动姿势反应的时间和方向很重要}
\subsection{前庭信息对于在不稳定表面和头部运动期间的平衡很重要}
\subsection{视觉输入为姿势系统提供方向和运动信息}
\subsection{来自单一感官形态的信息可能是模棱两可的}
\subsection{姿势控制系统使用结合了内部平衡模型的身体模式}

\section{姿势的控制取决于任务}
\subsection{任务要求决定了每个感觉系统在姿势平衡和定向中的作用}

\section{姿势控制分布在神经系统中}
\subsection{脊髓回路足以维持反重力支持但不足以维持平衡}
\subsection{脑干和小脑整合姿势的感觉信号}
\subsection{脊髓小脑和基底神经节在姿势适应中很重要}
\subsection{大脑皮层中心有助于姿势控制}

\section{要点}
\subsection{荐读}
\subsection{参考文献}