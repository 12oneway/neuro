\chapter{神经系统的性别分化}


\section{基因和激素决定男性和女性之间的生理差异}
\subsection{染色体性别指导胚胎的性腺分化}
\subsection{性腺合成促进性别分化的激素}
\subsection{类固醇激素生物合成障碍影响性别分化}

\subsection{神经系统的性别分化产生两性异形行为}
\subsection{勃起功能由脊髓中的性二态回路控制}
\subsection{鸟类的鸣叫是由前脑中的性二态回路控制的}
\subsection{哺乳动物的交配行为受下丘脑中的性二态神经回路控制}

\section{环境线索调节两性异形行为}
\subsection{信息素控制小鼠的伴侣选择}
\subsection{早期经验改变了后来的母性行为}
\subsection{一组核心机制是大脑和脊髓中许多性别二态性的基础}

\section{人脑是两性异形的}
\subsection{人类的性别二态性可能源于荷尔蒙作用或经验}
\subsection{大脑中的双态结构与性别认同和性取向相关}

\section{要点}
\subsection{选读}
\subsection{参考文献}
