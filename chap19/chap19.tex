\chapter{触觉}
% PDF所在目录: /data2/whd/win10/learn/neuro/neuro_神经科学原理_28_中枢神经系统的听觉处理.pdf

\section{主动和被动触摸有不同的目标}

\section{手有四种类型的机械感受器}
\subsection{细胞的感受野决定了它的触觉敏感区}
\subsection{两点辨别测试测量触觉敏锐度}
\subsection{缓慢适应的纤维检测物体压力和形状}
\subsection{快速适应纤维检测运动和振动}
\subsection{缓慢和快速适应的纤维对握力控制都很重要}

\section{触觉信息在中央触摸系统中处理}
\subsection{脊髓、脑干和丘脑回路分离触觉和本体感觉}
\subsection{体感皮层被组织成功能专门的列}
\subsection{皮层柱是按体位组织的}
\subsection{皮层神经元的感受野整合来自邻近受体的信息}

\section{触摸信息在连续的中央突触中变得越来越抽象}
\subsection{认知触觉由次级躯体感觉皮层中的神经元介导}
\subsection{主动触摸参与后顶叶皮层的感觉运动回路}

\section{大脑体感区的病变会产生特定的触觉缺陷}

\section{亮点}
\section{选读}
\section{参考文献}
