\chapter{突触连接的经验和细化}

\section{人类心理功能的发展受早期经验的影响}
\subsection{早期经历对社会行为有终生影响}
\subsection{视觉感知的发展需要视觉体验}

\section{视觉皮层双眼回路的发育取决于产后活动}
\subsection{视觉体验影响视觉皮层的结构和功能}
\subsection{电活动的模式组织双眼电路}

\section{关键时期视觉回路的重组涉及突触连接的改变}
\subsection{皮质重组取决于兴奋和抑制的变化}
\subsection{突触结构在关键时期发生改变}
\subsection{丘脑输入在关键时期被重塑}
\subsection{突触稳定有助于结束关键期}

\section{独立于经验的自发神经活动导致早期电路完善}

\section{依赖于活动的连接细化是大脑回路的一个普遍特征}
\subsection{视觉系统开发的许多方面都依赖于活动}
\subsection{感官模式在关键时期得到协调}
\subsection{不同的功能和脑区有不同的发展关键期}

\section{关键时期可以在成年期重新开启}
\subsection{视觉和听觉地图可以在成人中对齐}
\subsection{双眼电路可以在成人中重塑}

\section{要点}
\subsection{选读}
\subsection{参考文献}
