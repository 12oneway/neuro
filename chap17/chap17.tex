\chapter{感觉编码}

% 参考:https://www.dxy.cn/bbs/newweb/pc/post/40268362

\section{感受器对单一类型的刺激进行应答}
人们常说,大脑的独特能力在于数百万个神经元并行对信息进行处理。
然而,这种表述并没有捕捉到大脑和身体其他所有器官之间的本质区别。
肾脏或肌肉的能力在于许多细胞的平行作用,每个细胞都做同样的事情; 
如果我们了解肌细胞,我们就基本上了解整个肌肉的运作方式。
大脑的独特能力在于,数以百万计的细胞的平行作用,每个细胞都做不同的事情; 
了解大脑我们需要了解其任务的组织方式以及各个神经元如何执行这些任务。

感觉系统之间的功能差异源于驱动它们的不同刺激和构成每个系统的各自的通路。由于这些特征,每个神经元执行特定的任务,由它产生的动作电位序列对于所有突触后神经元具有特定的功能意义。这一基本思想体现在Charles Bell和Johannes Müller在19世纪提出的特异性理论中,并且仍然是感觉神经科学的基石之一。

丰富的感官体验始于成千上万的高度特异的感受器。每种受体在身体的特定位置响应特定种类的刺激,有时仅响应具有特定时间或空间模式的刺激.感受器将刺激转化为电信号,从而在所有感觉系统中建立共同的信号传导机制。感受器产生电活动称为感受器电位,其幅度和持续时间与感受器接受的刺激的强度和时间过程有关。将特定刺激转换成电信号的过程称为换能作用。

