\chapter{内隐记忆的细胞机制和个性的生物基础}


\section{内隐记忆的存储涉及突触传递有效性的变化}
\subsection{突触传递的突触前抑制导致习惯化}
\subsection{敏化涉及突触传递的突触前促进}
\subsection{经典威胁条件反射涉及促进突触传递}

\section{内隐记忆的长期存储涉及由 cAMP-PKA-CREB 通路介导的突触变化}
\subsection{循环 AMP 信号在长期致敏中起作用}
\subsection{非编码 RNA 在转录调控中的作用}
\subsection{长期突触促进是突触特定的}
\subsection{维持长期突触促进需要局部蛋白质合成的类似朊病毒的蛋白质调节剂}
\subsection{存储在感觉运动突触中的记忆在检索后变得不稳定但可以重新稳定}

\section{果蝇防御反应的经典威胁条件反射也使用 cAMP-PKA-CREB 途径}

\section{哺乳动物的威胁学习记忆涉及杏仁核}

\section{学习引起的大脑结构变化有助于个性的生物学基础}

\section{亮点}
\subsection{选读}
\subsection{参考文献}