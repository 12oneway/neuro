\chapter{有意识和无意识的心理过程障碍}
% PDF所在目录: /data2/whd/win10/learn/neuro/neuro_神经科学原理_28_中枢神经系统的听觉处理.pdf

\section{有意识和无意识的认知过程具有不同的神经相关性}

\section{在脑损伤后可以以夸张的形式看到感知过程中的有意识和无意识过程之间的差异}

\section{行动的控制在很大程度上是无意识的}

\section{有意识地回忆记忆是一个创造性的过程}

\section{行为观察需辅以主观报告}
\subsection{主观报告的验证具有挑战性}
\subsection{装病和歇斯底里会导致不可靠的主观报告}

\section{亮点}
\subsection{选读}
\subsection{参考文献}