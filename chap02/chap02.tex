\chapter{基因和行为}
% PDF所在目录: /data2/whd/win10/learn/neuro/neuro_神经科学原理_28_中枢神经系统的听觉处理.pdf

\section{了解分子遗传学和遗传率对于研究人类行为至关重要}

\section{对基因组结构和功能的理解正在演变}
\subsection{基因排列在染色体上}

\section{基因型和表现型之间的关系通常很复杂}

\section{基因在进化中得以保存}

\section{可以在动物模型中研究行为的遗传调控}
\subsection{转录振荡器调节苍蝇、小鼠和人类的昼夜节律}
\subsection{蛋白激酶的自然变异调节果蝇和蜜蜂的活性}
\subsection{神经肽受体调节几种物种的社会行为}

\section{人类遗传综合症的研究为社会行为的基础提供了初步见解}
\subsection{人类脑部疾病是基因与环境相互作用的结果}
\subsection{罕见的神经发育综合症为社会行为、知觉和认知的生物学提供了见解}

\section{精神疾病涉及多基因特征}
\subsection{自闭症谱系障碍遗传学的进展突出了罕见和从头突变在神经发育障碍中的作用}
\subsection{精神分裂症基因的鉴定突出了罕见和常见风险变异的相互作用}

\section{神经精神疾病遗传基础的观点}

\section{亮点}

\section{术语表}

\section{选读}




