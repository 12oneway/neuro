\chapter{语言}

\section{语言有许多结构层次:音素、语素、单词和句子}

\section{儿童的语言习得遵循普遍模式}
\subsection{“普遍主义者”婴儿在 1 岁时变得语言专业化}
\subsection{视觉系统参与语言的产生和感知}
\subsection{韵律线索早在子宫内就已习得}
\subsection{过渡概率有助于区分连续语音中的单词}
\subsection{语言学习有关键期}
\subsection{“父母语”说话风格增强语言学习}
\subsection{成功的双语学习取决于学习第二语言的年龄}

\section{一种新的语言神经基础模型已经出现}
\subsection{许多专门的皮层区域有助于语言处理}
\subsection{语言的神经结构在婴儿期迅速发展}
\subsection{左半球主导语言}
\subsection{韵律根据传达的信息同时影响右半球和左半球}

\section{失语症的研究为语言处理提供了见解}
\subsection{布洛卡失语症是由左额叶的大损伤引起的}
\subsection{Wernicke 的失语症是由于左侧后颞叶结构受损所致}
\subsection{传导性失语症是由后语言区的一部分受损引起的}
\subsection{完全性失语症源于多个语言中心的广泛受损}
\subsection{布罗卡区和韦尼克区附近区域受损导致经皮层失语}
\subsection{不太常见的失语症涉及对语言重要的其他大脑区域}

\section{亮点}
\subsection{选读}
\subsection{参考文献}

