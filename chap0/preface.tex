
% 参考: https://zhuanlan.zhihu.com/p/273186198
\label{chap:preface}
\begin{table}[htbp]
	\newcommand{\tabincell}[2]{\begin{tabular}{@{}#1@{}}#2\end{tabular}} %换行指令
	\centering
	\caption{名词列表 \label{tab:0_1}}
	\renewcommand\arraystretch{1.0}	%设置表格内行间距
	\setlength{\tabcolsep}{8mm}{
	\begin{tabular}{llll}
		\toprule 
		 名词(缩略词)   && 定义 \\
		 
		\midrule
		Best Frequency (BF)     && 最佳频率   \\
		 
		\midrule
		Boold Oxygen-Level Dependent (BOLD)     && 血氧水平依赖   \\
		
		\midrule
		Constant-Frequency (CF)     && 恒频   \\
		
		\midrule
		Doppler-shifted constant-frequency (DSCF)     && 多普勒频移恒频   \\
		
		\midrule
		electrocorticographical (ECoG)     && 脑电图   \\
		
		\midrule
		Excited-Excited neuron (EE neuron)     && 双耳兴奋神经元   \\
		
		\midrule
		Excited-Inhibited neuron (EI neuron)     && 兴奋-抑制神经元   \\
		
		\midrule
		frequency-modulated (FM)     && 调频   \\
		
		\midrule
		functional magnetic resonance imaging (fMRI),     && 功能性磁共振成像   \\
		
		\midrule
		Interaural Time 		Delay(ITD)   && 双耳时间延迟  \\
		
		\midrule
		Lateral Superior Olivary(LSO)   && 外侧上橄榄  \\
		
		\midrule
		Cochlear
		Nucleus(CN)   && 耳蜗核  \\
		
		\midrule
		Magnetoencephalography (MEG)   && 脑磁图  \\
		
		\midrule
		Medial Geniculate Body (MGB)   && 内侧膝状体  \\
		
		\midrule
		Medial Nucleus of the Trapezoid Body(MNTB)   && 斜方体内侧核  \\
		
		\midrule
		Medial Superior 		Olive(MSO)   && 内侧上橄榄  \\
		
		\midrule
		N-Methyl-D-Aspartate (NMDA)   && N-甲基-D-天冬氨酸  \\
		
		\midrule
		positron emission tomography (PET)     && 正电子发射断层成像   \\
		
		\midrule
		Posterior Parietal Cortex (PPC)     && 后顶叶皮层   \\
		
		\midrule
		primary auditory cortex (A1)   && 初级听觉皮层  \\
		
		\midrule
		Receptive Field (RF)   && 感受野  \\
		
		\midrule
		Rostral auditory cortex (R)   && 嘴侧听觉皮层  \\
		
		\midrule
		Rostrotemporal auditory cortex (R)   && 前颞听觉皮层 \\
		
		\midrule
		tonotopic map   && 音调拓扑图  \\
		
		\midrule
		transverse temporal gyri (Heschl's gyrus)   && 颞横回  \\
		
		\midrule
		V1   && 初级视觉皮层  \\
		
		\midrule
		Ventral Nucleus of the Trapezoid Body(MNTB)   && 斜方体腹侧核  \\
		
		\midrule
		What/Who pathway/stream  && 内容通路  \\
		
		\midrule
		Where/How pathway/stream && 空间通路  \\
		
		
		\bottomrule  

	\end{tabular}}
\end{table}%

