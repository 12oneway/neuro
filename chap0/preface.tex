
% 参考: https://zhuanlan.zhihu.com/p/273186198
%\label{chap:preface}
%\begin{table}[htbp]
%	\newcommand{\tabincell}[2]{\begin{tabular}{@{}#1@{}}#2\end{tabular}} %换行指令
%	\centering
%	\caption{名词列表 \label{tab:0_1}}
\renewcommand\arraystretch{1.0}	%设置表格内行间距
%\setlength{\tabcolsep}{4.5mm}{
\begin{longtable}{lll}
% https://blog.csdn.net/maths_girl/article/details/107030167
\caption{名词中英对照表 \label{tab:0_1}} \\
	\toprule 
 英文(缩略词)   && 中文 \\
 
 	\midrule
 	5-hydroxyindoleacetic acid (5-HIAA)     && \href{https://baike.baidu.com/item/5-\%E7\%BE%9F%E5%9F%BA%E5%90%B2%E5%93%9A%E4%B9%99%E9%85%B8/16984024}{5-羟基吲哚乙酸}    \\
 	
 	\midrule
 	5-hydroxytryptophan (5-HT)     && \href{https://baike.baidu.com/item/5-\%E7%BE%9F%E5%9F%BA%E8%89%B2%E6%B0%A8%E9%85%B8/5687636}{5-羟基色氨酸}    \\
 	
 	\midrule
 	5$\alpha$-Dihydrotestosterone (DHT)    && 5$\gamma$-二氢睾酮   \\
 	
 	% 阿尔茨海默病
 	\midrule
 	α-amino-3-hydroxy-5-methyl-4-isoxazolepropionic acid (AMPA)   && α-氨基-3-羟基-5-甲基异恶唑-4-丙酸   \\
 	
 	\midrule
 	$\gamma$-aminobutyric acid (GABA)    && $\gamma$-氨基丁酸   \\
 	
 	\midrule
 	abdominals (ABD)     && 腹肌   \\
 
 	\midrule
 	Abducens nerve (abducens)     && 外旋神经   \\
 
 	\midrule
 	abduction     && 外转(角膜向外的眼球运动)   \\
 	
 	% XI
 	\midrule
 	Accessory nerve   &&  副神经   \\
 	
 	\midrule
 	Accessory olfactory bulb  (AOB) &&  副嗅球   \\
 	
 	\midrule
 	Accessory optic nuclei   && 视副核   \\
 
	\midrule
	acetylcholine (ACh)     && 乙酰胆碱   \\
	
	\midrule
	acetylcholinesterase (AChE)     && 乙酰胆碱酯酶   \\
	
	\midrule
	Acetylcholine receptors (AChR)    && 乙酰胆碱受体   \\
	
	\midrule
	action potential  (AP)  && 动作电位   \\
	
	\midrule
	Adenine (A)     && 腺嘌呤   \\
	
	\midrule
	adenosine triphosphate (ADP)     && 二磷酸腺苷   \\
	
	\midrule
	adenosine triphosphatase     && 腺苷三磷酸酶   \\
	
	\midrule
	adenosine triphosphate (ATP)     && 三磷酸腺苷   \\
	
	\midrule
	Adenylyl cyclase     && 腺苷酸环化酶   \\
	
	\midrule
	adduction     && 内转   \\
	
	\midrule
	adrenocorticotropic hormone (ACTH)     && \href{https://baike.baidu.com/item/\%E4%BF%83%E8%82%BE%E4%B8%8A%E8%85%BA%E7%9A%AE%E8%B4%A8%E6%BF%80%E7%B4%A0/2388734}{促肾上腺皮质激素}   \\
	
	\midrule
	agonist–antagonist     &&  兴奋-拮抗  \\
	
	\midrule
	agrin     &&  聚集蛋白  \\
	
	% (胚胎)翼板
	\midrule
	Alar plate     &&  翼板  \\
	
	\midrule
	Alden Spencer     &&  奥尔登·斯宾塞  \\
	
	\midrule
	Alexander Luria     &&  亚历山大·鲁利亚  \\
	
	\midrule
	Allan Rechtschaffen     &&  艾伦·雷克尚芬  \\
	
	\midrule
	alleles     &&  等位基因  \\
	
	\midrule
	alien hand syndrome (AHS)     &&  异己手综合征  \\
	
	\midrule
	Alzheimer disease (AD)     &&  阿尔茨海默病  \\
	
	\midrule
	amacrine cell     && 无长突细胞   \\
	
	\midrule
	amnesic patients (AMN)    && 失忆症患者   \\
	
	\midrule
	amygdala; amygdaloid (Am)    && 杏仁核   \\
	
	\midrule
	amyloid precursor protein (APP)     && 淀粉样前体蛋白   \\
	
	\midrule
	amyotrophic lateral sclerosis (ALS), Lou Gehrig disease    && \href{https://baike.baidu.com/item/\%E8\%82%8C%E8%90%8E%E7%BC%A9%E4%BE%A7%E7%B4%A2%E7%A1%AC%E5%8C%96/9336045}{肌萎缩侧索硬化}   \\
	
	\midrule
	androstadienone (AND)     && 雄二烯酮   \\
	
	\midrule
	Angelman Syndrome     && \href{https://baike.baidu.com/item/\%E5%A4%A9%E4%BD%BF%E7%BB%BC%E5%90%88%E5%BE%81/4662845}{天使综合症}   \\
	
	\midrule
	annulus of Zinn     && 总腱环   \\
	
	\midrule
	anterior cingulate cortex (ACC)     && 前扣带皮层   \\
	
	\midrule
	anterior commissure (AC)     && 前连合   \\
	
	\midrule
	anterior insula (AI)     && 前脑岛   \\
	
	\midrule
	Anterior intraparietal area (AIP)     && 前顶叶   \\
	
	\midrule
	anterior thalamus (antTHAL)     && 前丘脑   \\
	
	\midrule
	antisense oligonucleotides (ASO, ASOs)     && 反义寡核苷酸   \\
	
	\midrule
	Anterior commissure     && 	前连合   \\
	
	\midrule
	Anterior cranial fossa     && 	颅前窝   \\
	
	\midrule
	anterior tibial muscle     && 	胫骨前肌   \\
	
	\midrule
	anterograde amnesia     && 	顺行性遗忘症   \\
	
	\midrule
	antigravity muscles     && 	抗引力肌   \\
	
	\midrule
	Anxiety Sensitivity Index     && 	焦虑度   \\
	
	\midrule
	Aplysia     && 海兔   \\
	
	\midrule
	apolipoprotein E (APOE)     && \href{https://baike.baidu.com/item/\%E8%BD%BD%E8%84%82%E8%9B%8B%E7%99%BDE/4226374}{载脂蛋白E}   \\
	
	\midrule
	Area Under Curve (AUC)     && 曲线下面积   \\
	
	\midrule
	Arcuate sulcus     && 弓形沟   \\
	
	\midrule
	Asperger syndrome     && 阿斯伯格综合症   \\
	
	\midrule
	Associative learning     && 联想学习   \\
	
	\midrule
	Ataxin-1 (ATXN1)     && 共济失调蛋白-1   \\
	
	\midrule
	Attention deficit hyperactivity disorder (ADHD)     && 注意缺陷多动障碍   \\
	
	\midrule
	autism spectrum disorder (ASD)     && \href{https://baike.baidu.com/item/\%E8%87%AA%E9%97%AD%E7%97%87%E8%B0%B1%E7%B3%BB%E9%9A%9C%E7%A2%8D/1704369}{自闭症谱系障碍}   \\
	
	\midrule
	autistic savant     && 自闭症学者   \\
	
	\midrule
	Axonal transport     && 轴突运输   \\
	
	\midrule
	Barrel cortex   && 桶状皮层  \\
	
	\midrule
	Basal plate   && 基板  \\
	
	\midrule
	Basal temporal   && 基底颞叶  \\
	
	\midrule
	Basis pedunculi   && 基脚  \\
	
	\midrule
	Basket cell   && 篮状细胞  \\
	
	\midrule
	Beck Anxiety Inventory   && 贝克焦虑问卷  \\
	
	\midrule
	bed nucleus of the stria terminalis (BNST)  && 终纹床核  \\
	
	\midrule
	Ben Barres   && 本·巴瑞斯  \\
	
	\midrule
	Bert Sakmann   && 伯特·萨克曼  \\
 
	\midrule
	Best Frequency (BF)     && 最佳频率   \\
	
	\midrule
	biceps brachii     && 	肱二头肌   \\
	
	\midrule
	Bithorax complex     && 	双胸复合体   \\
	
	\midrule
	bone morphogenetic protein  (BMP)   && 	骨形态发生蛋白   \\
 
	\midrule
	Boold Oxygen-Level Dependent (BOLD)     && 血氧水平依赖   \\
	
	\midrule
	bradykinin (BK)     && 缓激肽   \\
	
	\midrule
	brain derived neurotrophic factor (BDNF)     && 脑源性神经营养因子   \\
	
	\midrule
	BRAIN Initiative     && 脑计划   \\
	
	\midrule
	brain stem (BS)     && 脑干   \\
	
	\midrule
	Brodmann area (Cg25)   && 布罗德曼 25 区  \\
	
	\midrule
	Bulbocavernosus muscle  && 球海绵体肌  \\
	
	\midrule
	B. F. Skinner  && 斯金纳  \\
	
	\midrule
	Caenorhabditis elegans   && \href{https://baike.baidu.com/item/\%E7%A7%80%E4%B8%BD%E9%9A%90%E6%9D%86%E7%BA%BF%E8%99%AB/154672}{秀丽隐杆线虫}  \\
	
	\midrule
	calcitonin gene–related peptide (CGRP)   && 降钙素基因相关肽  \\
	
	\midrule
	Calcium-calmodulin (CaM)-dependent protein kinase II (CaMKII)   && 钙/钙调蛋白依赖性蛋白激酶 2  \\
	
	\midrule
	cAMP response element binding protein (CREB)  && 环磷酸腺苷应答元件结合蛋白  \\
	
	\midrule
	canonical splice site mutation   && 经典剪切位点突变  \\
	
	\midrule
	Carl Wernicke   && 卡尔·韦尼克  \\
	
	\midrule
	Caspase   && 半胱天冬酶  \\
	
	\midrule
	Cavernous sinus   && 海绵窦  \\
	
	\midrule
	center of mass (CoM)   && 质心  \\
	
	\midrule
	center of pressure (CoP)   && 压力中心  \\
	
	\midrule
	central pattern generators (CPGs)   && 中枢模式发生器  \\
	
	\midrule
	Cephalic flexure   && 头曲  \\
	
	\midrule
	Cerebellar flocculus   && 小脑小叶  \\
	
	\midrule
	Cerebellar peduncles   && 小脑脚  \\
	
	\midrule
	Cerebellopontine angle (CPA)   && 桥小脑角区  \\
	
	\midrule
	Cervical flexure   && 颈曲  \\
	
	\midrule
	Cervical ventral roots   && 颈腹根  \\
	
	\midrule
	Chaperone proteins   && 伴侣蛋白  \\
	
	\midrule
	Charles Darwin   && 查尔斯·达尔文  \\
	
	\midrule
	Charles F. Stevens   && 查尔斯·史蒂芬斯  \\
	
	\midrule
	Charles Sherrington   && 查尔斯·谢林顿  \\
	
	\midrule
	Chiara Cirelli   && 基娅拉·奇雷利  \\
	
	\midrule
	Cholecystokinin (CCK)   && 胆囊收缩素  \\
	
	\midrule
	choline (Ch)   && 胆碱  \\
	
	\midrule
	choline acetyltransferase (ChAT)   && 胆碱乙酰转移酶  \\
	
	\midrule
	choline transporter (CHT)   && 胆碱转运蛋白  \\
	
	\midrule
	Chondroitin sulphate proteoglycans (CSPG) && 硫酸软骨素蛋白多糖  \\
	
	\midrule
	Chorda tympani nerve   && 鼓索神经  \\
	
	\midrule
	chronic traumatic encephalopathy (CTE)   && 慢性创伤性脑病  \\
	
	\midrule
	ciliary neurotrophic factor (CNTF)  && 睫状神经营养因子  \\
	
	\midrule
	cingulate motor areas (CMA)   && 扣带运动区  \\
	
	\midrule
	Climbing fiber   && 攀缘纤维  \\
	
	\midrule
	Clive Waring   && 克里夫·韦林  \\
	
	\midrule
	Claude Bernard   && 克劳德·伯纳德  \\
	
	\midrule
	Cochlear Nucleus(CN)   && 耳蜗核  \\
	
	\midrule
	Commissural fibers   && 连合纤维  \\
	
	\midrule
	complementary DNA (cDNA)   && 互补脱氧核糖核酸  \\
	
	\midrule
	Complete androgen insensitivity syndrome (CAIS)  && 雄激素不敏感综合征  \\
	
	\midrule
	computed tomographic (CT)   && 计算机断层扫描  \\
	
	\midrule
	conditioned stimulus (CS)     &&  条件刺激  \\
	
	\midrule
	Constant-Frequency (CF)     &&  恒频  \\
	
	\midrule
	cone cell      && 视锥细胞  \\
	
	\midrule
	Congenital adrenal hyperplasia (CAH)  && 先天性肾上腺皮质增生症  \\
	
	\midrule
	continuous positive airway pressure (CPAP)     && 持续气道正压  \\
	
	\midrule
	copy number variation (CNV)      && \href{https://baike.baidu.com/item/\%E6%8B%B7%E8%B4%9D%E6%95%B0%E5%8F%98%E5%BC%82}{拷贝数变异}  \\
	
	\midrule
	cornu Ammonis (CA)    &&  阿蒙角  \\
	
	\midrule
	coronal plane     &&  冠状面  \\
	
	\midrule
	corpus callosum     &&  胼胝体  \\
	
	\midrule
	cortical; cortex     &&  皮层  \\
	
	\midrule
	cortical plate (CP)     &&  皮层板  \\
	
	\midrule
	corticomotoneuronal (CM)     &&  皮层运动神经  \\
	
	\midrule
	corticotropin-releasing hormone (CRH)    &&  \href{https://baike.baidu.com/item/\%E4%BF%83%E8%82%BE%E4%B8%8A%E8%85%BA%E7%9A%AE%E8%B4%A8%E6%BF%80%E7%B4%A0%E9%87%8A%E6%94%BE%E6%BF%80%E7%B4%A0/3760624}{促肾上腺皮质激素释放激素}  \\
	
	\midrule
	CREB binding protein  (CBP)   &&  CREB 结合蛋白  \\
	
	\midrule
	Cribiform plate     &&  筛板  \\
	
	\midrule
	cuneiform nucleus (CNF)     &&  楔形核  \\
	
	\midrule
	cyclic adenosine monophosphate (cAMP)     &&  环磷酸腺苷  \\
	
	\midrule
	Cyclic nucleotide–gated channel    &&  环核苷酸门控离子通道  \\
	
	\midrule
	cytosine (C)     &&  胞嘧啶  \\
	
	\midrule
	cytosine-adenine-guanine (CAG)     &&  胞嘧啶-腺嘌呤-鸟嘌呤  \\
	
	\midrule
	David Ferrier     &&  大卫·费里尔  \\
	
	\midrule
	David Hubel     &&  大卫·休伯尔  \\
	
	\midrule
	David Marr     &&  大卫·马尔  \\
	
	\midrule
	de novo mutation     &&  新生突变  \\
	
	\midrule
	Deep cerebellar nucleus     &&  小脑深部核团  \\
	
	\midrule
	DeoxyriboNucleic Acid (DNA)     &&  脱氧核糖核酸  \\
	
	\midrule
	Deep brain stimulation (DBS)     &&  深部脑刺激  \\
	
	\midrule
	delayed-match-to-sample  (DMS)   &&  延迟样本匹配  \\
	
	\midrule
	deleted in colon cancer (DCC)     &&  结直肠癌缺失蛋白  \\
	
	\midrule
	Dentate region     && 齿状区 \\
	
	\midrule
	dentatorubro-pallidoluysian atrophy (DRPLA)     &&  \href{https://baike.baidu.com/item/\%E9%BD%BF%E7%8A%B6%E6%A0%B8%E7%BA%A2%E6%A0%B8%E8%8B%8D%E7%99%BD%E7%90%83%E8%B7%AF%E6%98%93%E4%BD%93%E8%90%8E%E7%BC%A9%E7%97%87/1486358}{齿状核红核苍白球路易体萎缩症}  \\
	
	\midrule
	depression, infraduction     &&  下转(角膜向下的眼球运动)  \\
	
	\midrule
	\makecell{Diagnostic and Statistical Manual of \\Mental Disorders}     &&  《精神疾病诊断和统计手册》  \\
	
	\midrule
	\makecell{Diagnostic and Statistical Manual of \\Mental Disorders, Fifth Edition (DSM-5)}     &&  《精神疾病诊断和统计手册(第五版)》  \\
	
	\midrule
	Donald Hebb    &&  唐纳德·赫布  \\
	
	\midrule
	dopamine transporter (DAT)     &&  多巴胺转运蛋白  \\
	
	\midrule
	Doppler-shifted constant-frequency (DSCF)     &&  多普勒频移恒频  \\
	
	\midrule
	dorsal anterior cingulate cortex (dACC)     &&  背前扣带皮层  \\
	
	\midrule
	dorsal caudate nucleus (dCN)     &&  背侧尾核  \\
	
	\midrule
	dorsal premotor cortex (PMd)     &&  背侧前运动皮层  \\
	
	\midrule
	dorsal root ganglion (DRG)     &&  背根神经节  \\
	
	\midrule
	dorsolateral prefrontal cortex (DLPFC, F9)     &&  背外侧前额叶皮层  \\
	
	\midrule
	dorsomedial prefrontal cortex (DMPFC)     &&  背内侧前额叶皮层  \\
	
	\midrule
	dorsomedial thalamus (dmTHAL)     &&  背内侧丘脑  \\
	
	\midrule
	Dorsum sellae     &&  鞍背  \\
	
	\midrule
	Down syndrome     &&  唐氏综合症  \\
	
	\midrule
	Drosophila melanogaster     &&  黑腹果蝇  \\
	
	\midrule
	Dynorphin     &&  强啡肽  \\
	
	\midrule
	Edgar Adrian     &&  埃德加·阿德里安  \\
	
	\midrule
	Edinger-Westphal nucleus     &&  动眼神经副核  \\
	
	\midrule
	Eduard Hitzig     &&  希兹格  \\
	
	\midrule
	Edward Lee Thorndike     &&  爱德华·李·桑代克  \\
	
	\midrule
	Edward Tolman     &&  爱德华·托尔曼  \\
	
	\midrule
	Efference copy     &&  传出副本  \\
	
	\midrule
	electroconvulsive therapy (ECT)     &&  电痉挛疗法  \\
	
	\midrule
	electrocorticography (ECoG)     &&  脑皮层电图  \\
	
	\midrule
	electroencephalogram (EEG)   &&  脑电图  \\
	
	\midrule
	electromyogram (EMG)     &&  肌电图  \\
	
	\midrule
	electro-oculogram (EOG)     &&  眼电图  \\
	
	\midrule
	elevation, supraduction     &&  上转(角膜向上的眼球运动)  \\
	
	\midrule
	Emboliform nucleus     &&  栓状核  \\
	
	\midrule
	embryonic stem  (ES)   &&  胚胎干细胞  \\
	
	\midrule
	Endel Tulving     &&  安道尔·图威  \\
	
	\midrule
	Endoplasmic reticulum     &&  内质网  \\
	
	\midrule
	Endorphin     &&  内啡肽  \\
	
	\midrule
	entorhinal cortex     &&  内嗅皮层  \\
	
	\midrule
	Erwin Neher    &&  厄温·内尔  \\
	
	\midrule
	estratetraenol (EST)   &&  雌四烯醇  \\
	
	\midrule
	ethyl methanesulfonate (EMS)    &&  甲磺酸乙酯  \\
	
	\midrule
	Excitatory postsynaptic currents (EPSCs)     &&  兴奋性突触后电流  \\
	
	\midrule
	Excitatory PostSynaptic Potential (EPSP)     &&  兴奋性突触后电位  \\
	
	\midrule
	Excited-Excited neuron (EE neuron)     &&  双耳兴奋神经元  \\
	
	\midrule
	Excited-Inhibited neuron (EI neuron)     && 兴奋-抑制神经元   \\
	
	\midrule
	external globus pallidus (GPe)     && 外侧苍白球   \\
	
	\midrule
	External segment     && 外侧部   \\
	
	\midrule
	extorsion     && 外旋   \\
	
	\midrule
	Extrastriate     && 纹外皮层   \\
	
	% 小脑
	\midrule
	fastigial nucleus (FN)     &&  顶核  \\
	
	\midrule
	Fechner     &&  费希纳  \\
	
	\midrule
	Fergus Craik     &&  弗格斯•克雷克  \\
	
	\midrule
	fiber tract     &&  纤维束  \\
	
	\midrule
	flexor carpi ulnaris (FCU)     &&  尺侧腕屈肌  \\
	
	\midrule
	Flocculonodular lobe     &&  绒球小结叶  \\
	
	\midrule
	 focal damage     &&  局灶性损伤  \\
	
	\midrule
	formant frequencies     &&  共振峰频率  \\
	
	\midrule
	Fourth ventricle     &&  第四脑室  \\
	
	\midrule
	fragile X syndrome     &&  \href{https://baike.baidu.com/item/\%E8\%84%86%E6%80%A7X%E7%BB%BC%E5%90%88%E5%BE%81/12612308}{脆性X综合症}  \\
	
	\midrule
	frameshift mutation     &&  \href{https://baike.baidu.com/item/\%E6\%A1%86%E7%A7%BB%E7%AA%81%E5%8F%98/5783764}{框移突变}  \\
	
	\midrule
	Francis Galton     &&  \href{https://baike.baidu.com/item/\%E5%BC%97%E6%9C%97%E8%A5%BF%E6%96%AF%C2%B7%E9%AB%98%E5%B0%94%E9%A1%BF}{弗朗西斯·高尔顿}  \\
	
	\midrule
	Franz Joseph Gall (Gall)     &&  弗朗兹·约瑟夫·加尔  \\
	
	\midrule
	Free nerve endings     &&  游离神经末梢  \\
	
	\midrule
	Frederic Bartlett     &&  弗雷德里克·巴特莱特  \\
	
	\midrule
	frequency-modulated (FM)     &&  调频  \\
	
	\midrule
	Frontal Eye Field (FEF)     &&  额叶眼区  \\
	
	\midrule
	frontoinsular cortex (FI)     &&  前岛叶皮层  \\
	
	\midrule
	fruitless (Fru)     &&  无效基因  \\
	
	\midrule
	functional electrical stimulation (FES)     &&  功能性电刺激  \\
	
	\midrule
	functional magnetic resonance imaging (fMRI)     &&  功能性核磁共振成像  \\
	
	\midrule
	fusiform gyrus (FG)     &&  梭状回  \\
	
	\midrule
	GTP-bindingprotein (G protein)     &&  三磷酸鸟苷结合蛋白(GTP, Guanosine triphosphate)  \\
	
	\midrule
	gastrocnemius (GAS)    &&  腓肠肌  \\
	
	\midrule
	Generalized anxiety disorder (GAD)     &&  广泛性焦虑障碍  \\
	
	\midrule
	Genital ridge     &&  生殖嵴  \\
	
	\midrule
	genome-wide association studies (GWAS)     &&  全基因组关联研究  \\
	
	\midrule
	George Widener     &&  \href{https://baike.baidu.com/item/\%E4%B9%94%E6%B2%BB%C2%B7%E6%80%80%E5%BE%B7%E7%BA%B3/58006951}{乔治·怀德纳}  \\
	
	\midrule
	Gill-withdrawal reflex     &&  缩鳃反射  \\
	
	\midrule
	Giulio Tononi     &&  朱利奥.托诺尼  \\
	
	\midrule
	glial cell (G)     &&  胶质细胞  \\
	
	\midrule
	glial cell line–derived neurotrophic factor (GDNF)     &&  胶质细胞源性神经营养因子  \\
	
	\midrule
	Glial scar     &&  胶质瘢痕  \\
	
	\midrule
	Globose nucleus     && 球状核  \\
	
	\midrule
	globus pallidus (GP)    && 苍白球  \\
	
	\midrule
	Glossopharyngeal nerve     && 舌咽神经  \\
	
	\midrule
	glucocerebrosidase-1 (GBA1)     &&  葡糖脑甘酯酶-1  \\
	
	\midrule
	glucocorticoid receptor (GR)    &&  糖皮质激素受体  \\
	
	\midrule
	glutamate decarboxylase     &&  谷氨酸脱羧酶  \\
	
	\midrule
	Glutamatergic neurons     &&  谷氨酸能神经元  \\
	
	\midrule
	Glycoprotein 130 (GP130)    &&  糖蛋白 130  \\
	
	\midrule
	goat-like     &&  	膻味  \\
	
	\midrule
	Golgi cell     &&  	高尔基细胞  \\
	
	\midrule
	granule cell     &&  	颗粒细胞  \\
	
	\midrule
	Granular layer     &&  	颗粒细胞层  \\
	
	\midrule
	ground reaction force (GRFh)     &&  地面反作用力  \\
	
	\midrule
	growth associated protein 43  (GAP-43)   &&  神经生长相关蛋白-43  \\
	
	\midrule
	Guanine (G)     &&  鸟嘌呤  \\
	
	\midrule
	Gustatory afferent nerve     &&  味觉传入神经  \\
	
	\midrule
	gustatory receptor     &&  味觉感受器  \\
	
	\midrule
	Gustav Fritsch     &&  费理屈  \\
	
	\midrule
	% 当一个物体做定轴转动时,磁矩的和角动量的大小之比
	gyromagnetic ratio     &&  磁旋比  \\
	
	\midrule
	Hair follicle    &&  毛囊  \\
	
	\midrule
	hamstring (HAM)    &&  腘绳肌  \\
	
	\midrule
	Hans Spemann    &&  汉斯·斯佩曼  \\
	
	\midrule
	Henry Head    &&  亨利·海德  \\
	
	\midrule
	Henry Molaison    &&  亨利·莫莱森  \\
	
	\midrule
	Hering-Breuer reflex     &&  黑-伯反射(黑林-伯鲁反射)  \\
	
	\midrule
	high vocal center (HVC)    &&  高级发声中枢  \\
	
	\midrule
	highfunctioning autism     &&  高功能自闭症  \\
	
	\midrule
	Hilde Mangold     &&  希尔德·曼戈尔德  \\
	
	\midrule
	Horace Barlow     &&  霍勒斯·巴洛  \\
	
	\midrule
	Horizontal canal     &&  水平耳道  \\
	
	\midrule
	horizontal plane     &&  水平面  \\
	
	\midrule
	Howard Eichenbaum     &&  霍华德·艾肯鲍姆  \\
	
	\midrule
	Huntington     &&  \href{https://baike.baidu.com/item/\%E4%BA%A8%E5%BB%B7%E9%A1%BF%E7%97%85/10377104}{亨廷顿病}  \\
	
	\midrule
	Huntington-like 2     &&  类亨廷顿病2型  \\
	
	% XII
	\midrule
	Hypoglossal nerve     &&  舌下神经  \\
	
	\midrule
	Hypoglossal nucleus (nXIIts)    &&  舌下神经核  \\
	
	\midrule
	hypothalamic–pituitary–adrenal (HPA)     &&  下丘脑-垂体-肾上腺  \\
	
	\midrule
	hypothalamus (HT)     &&  下丘脑  \\
	
	\midrule
	Immunoglobulin superfamily   && 免疫球蛋白超家族  \\
	
	\midrule
	In vitro   && 体外  \\
	
	\midrule
	induced pluripotent stem (iPS)  && 诱导的多能性干细胞  \\
	
	\midrule
	indel   && 插入缺失突变  \\
	
	\midrule
	inferior frontal gyrus (IFG)   && 额下回  \\
	
	\midrule
	inferior oblique   && 下斜肌  \\
	
	\midrule
	\makecell{inferior posterior regions of \\ prefrontal cortex (IPPFC)}  && 后下部前额叶皮层  \\
	
	\midrule
	inferior rectus   && 下直肌  \\
	
	\midrule
	inferior rectus   && 下直肌  \\
	
	\midrule
	information transfer rate (ITR)   && 信息传递率  \\
	
	\midrule
	inner fiber layer (IFL)   && 内纤维层  \\
	
	\midrule
	inner subventricular zone (ISVZ)   && 室下内侧区  \\
	
	\midrule
	insula (Ins)   && 脑岛  \\
	
	\midrule
	intermediate zone (IZ)  && 中间区  \\
	
	\midrule
	Internal auditory canal   && 内耳道  \\
	
	\midrule
	Internal segment  && 内侧部  \\
	
	% 苍白球
	\midrule
	Internal segment  && 内侧部  \\
	
	\midrule
	interposed nuclei (IP)  && 间位核  \\
	
	\midrule
	interpositus nucleus  && 间位核  \\
	
	\midrule
	Intracortical electrodes   && 皮层内电极  \\
	
	\midrule
	intraparietal sulcus (IPS)   && 顶内沟  \\
	
	\midrule
	\makecell{interstitial nucleus of the \\medial longitudinal fasciculus}   && 内侧纵束间质核  \\
	
	\midrule
	\makecell{interstitial nucleus of the \\anterior hypothalamus}  (INAH) && 下丘脑前间质核  \\
	
	\midrule
	internal globus pallidus (GPi, Gpi)  && 苍白球内侧核  \\
	
	\midrule
	intracortical microstimulation (ICMS)  && 皮层内微刺激  \\
	
	\midrule
	intorsion   && 内旋  \\
	
	\midrule
	isoamyl acetate   && 乙酸异戊酯  \\
	
	\midrule
	immunoglobulin (IgE)   && 免疫球蛋白  \\
	
	\midrule
	infundibular recess (IFR)   && 漏斗隐窝  \\
	
	\midrule
	interaural time difference (ITD)   && 双耳时间差  \\
	
	\midrule
	intracranial electroencephalography (iEEG)  && 颅内脑电图  \\
	
	\midrule
	interstitial nucleus of Cajal (iC)   && 间位核  \\
	
	\midrule
	Ivan Pavlov   && 伊万·巴甫洛夫  \\
	
	\midrule
	James J. DiCarlo   && 詹姆斯·迪卡罗  \\
	
	\midrule
	James Olds   && 詹姆斯·奥尔兹  \\
	
	\midrule
	James Papez   && 詹姆斯‧帕佩兹  \\
	
	\midrule
	Janus kinase-signal transducer and activator of transcription (JAK/STAT)  && 两面神激酶-信号转导和转录激活因子  \\
	
	\midrule
	John Carew Eccles (John C. Eccles)   && 约翰·卡鲁·埃克尔斯  \\
	
	\midrule
	jugular foramen   && 颈静脉孔  \\
	
	\midrule
	Junctional fold   && 接头皱褶  \\
	
	\midrule
	Jonathan Wolpaw   && 乔纳森·沃尔帕乌  \\
	
	\midrule
	just noticeable difference (JND)   && 最小可觉差  \\
	
	\midrule
	J. N. Langley   && 兰列  \\
	
	\midrule
	K complex   && K-复合波  \\
	
	\midrule
	Kalman filter decoding movement velocity (V-KF)   && 解码运动速度卡尔曼滤波器  \\
	
	\midrule
	kiss-and-run   && 亲完就跑  \\
	
	\midrule
	Lambert-Eaton syndrome   && 兰伯特-伊顿综合症(肌无力综合症)  \\
	
	\midrule
	Larry Squire   && 拉里·斯奎尔  \\
	
	\midrule
	Larry Weiskrantz   && 拉里·维斯克兰茨  \\
	
	\midrule
	Larmor equation   && 拉莫尔方程  \\
	
	\midrule
	lateral geniculate nucleus (LGN)   && 外侧膝状体核  \\
	
	\midrule
	lateral hypothalamic area (LHA)  && 下丘脑外侧区  \\
	
	\midrule
	lateral hypothalamus (LH)  && 外侧下丘脑  \\
	
	\midrule
	Lateral intraparietal area (LIP)   && 侧顶叶  \\
	
	\midrule
	lateral habenula   && 外侧缰核  \\
	
	\midrule
	lateral magnocellular nucleus of the anterior neostriatum (LMAN)  && 新纹状体前部大细胞核外侧部  \\
	
	\midrule
	lateral motor columns (LMC)   && 外侧运动柱  \\
	
	\midrule
	lateral parietal   && 顶叶外侧  \\
	
	\midrule
	lateral rectus   && 外直肌  \\
	
	\midrule
	Lateral sinus   && 横窦  \\
	
	\midrule
	Lateral Superior Olivary(LSO)   && 外侧上橄榄  \\
	
	\midrule
	lateral vestibular nucleus (LVN)  && 前庭外侧核  \\
	
	\midrule
	lateral vestibulospinal tracts (LVST)  && 外前庭脊髓束  \\
	
	\midrule
	lateral view   && 侧视图  \\
	
	\midrule
	Laterodorsal tegmental nucleus   && 背外侧被盖核  \\
	
	\midrule
	left flexor motor neurons (lFmn)   && 左屈肌运动神经元  \\
	
	\midrule
	left forelimb (lFL)   && 左前肢  \\
	
	\midrule
	left hindlimb (lHL)   && 左后肢  \\
	
	\midrule
	Levi-Montalcini   && 列维-蒙塔尔奇尼  \\
	
	\midrule
	locus   && \href{https://baike.baidu.com/item/Locus/1628923}{基因座}  \\
	
	\midrule
	Leucine-enkephalin   && 亮氨酸脑啡肽  \\
	
	\midrule
	levator   && 眼提肌  \\
	
	\midrule
	lick and groom (LG)  && 舔舐和梳理  \\
	
	\midrule
	long noncoding RNA (lncRNA)  && \href{https://baike.baidu.com/item/%E9%95%BF%E9%9D%9E%E7%BC%96%E7%A0%81rna/3674902}{长链非编码核糖核酸}  \\
	
	\midrule
	long-term depression (LTD)  && 长时程抑制  \\
	
	\midrule
	Long-term memory  && 长时记忆  \\
	
	\midrule
	long-term potentiation (LTP)  && 长时程增强  \\
	
	\midrule
	\makecell{low-density lipoprotein \\receptor-related protein 4 (LRP4)}   && 低密度脂蛋白受体相关蛋白  \\
	
	\midrule
	low-threshold mechanoreceptors (LTMR)   && 低阈值机械感受器  \\
	
	\midrule
	Lugaro cell   && 卢加洛细胞  \\
	
	\midrule
	Machado-Joseph disease   && 马查多-约瑟夫病  \\
	
	\midrule
	MacLean   && 麦克莱恩  \\
		
	\midrule
	Magnetoencephalography (MEG)   && 脑磁图  \\
	
	\midrule
	main olfactory bulb (MOB) && 主嗅球  \\
	
	\midrule
	main olfactory epithelium (MOE)  && 主嗅上皮  \\
	
	\midrule
	marginal zone (MZ)   && 边缘区  \\
	
	\midrule
	major depression   && 重度抑郁症  \\
	
	\midrule
	major histocompatibility (MHC)   && 主要组织相容性  \\
	
	\midrule
	mammalian target of rapamycin (mTOR)   && 哺乳动物雷帕霉素靶蛋白  \\
	
	\midrule
	Mammillothalamic tract   && 乳头丘脑束  \\
	
	\midrule
	maximal voluntary contraction (MVC)   && 最大主动收缩  \\
	
	\midrule
	maximum pulling force (MPF)   && 最大拉力  \\
	
	\midrule
	Medial Geniculate Body (MGB)   && 内侧膝状体  \\
	
	\midrule
	medial frontal cortex (mF10)   && 内侧前额叶皮层  \\
	
	\midrule
	Medial ganglionic eminence   && 内侧神经节隆起  \\
	
	\midrule
	medial geniculate nucleus (MGN)  && 内侧膝状体核  \\
	
	\midrule
	medial intraparietal area (MIP)   && 内顶叶内区  \\
	
	\midrule
	medial longitudinal fasciculus   && 内侧纵束  \\
	
	\midrule
	medial nucleus of the dorsolateral thalamus (DLM)  && 丘脑背外侧内侧核  \\
	
	\midrule
	Medial Nucleus of the Trapezoid Body (MNTB)   && 斜方体内侧核  \\
	
	\midrule
	Medial prefrontal cortex    && 内侧前额叶皮层  \\
	
	\midrule
	Medial preoptic nucleus    && 视前内侧核  \\
	
	\midrule
	medial rectus    && 内直肌  \\
	
	\midrule
	medial reticular formation (MRF)    && 内侧网状结构  \\
	
	\midrule
	Medial Superior Olive(MSO)   && 内侧上橄榄  \\
	
	\midrule
	medial superior temporal area (MST)   && 上颞内侧区  \\
	
	\midrule
	Median eminence   && 正中隆起  \\
	
	\midrule
	medial vestibulospinal tracts (MVST) && 内前庭脊髓束  \\
	
	\midrule
	Medullary reticular formation && 延髓网状结构  \\
	
	\midrule
	melaninconcentrating hormone && 黑色素浓缩素  \\
	
	\midrule
	Merkel cell && 梅克尔细胞  \\
	
	\midrule
	mesencephalic locomotor region (MLR)   && 中脑运动区  \\
	
	\midrule
	mesencephalic reticular formation   && 中脑网状结构  \\
	
	\midrule
	messenger Ribonucleic Acid (mRNA)   && 信使核糖核酸  \\
	
	\midrule
	Methionine-enkephalin   && 甲硫氨酸脑啡肽  \\
	
	\midrule
	Michael Meaney   && 迈克尔·米尼  \\
	
	\midrule
	microRNA (miRNA)   && \href{https://baike.baidu.com/item/micro\%20RNA/3683223}{微小核糖核酸}  \\
	
	\midrule
	Macrophage infiltration   && 巨噬细胞浸润  \\
	
	\midrule
	middle cingulate cortex (MCC)   && 中扣带皮层  \\
	
	\midrule
	Middle cranial fossa   && 颅中窝  \\
	
	\midrule
	mid-subcallosal cingulate (Mid-SCC)  && 中下胼胝体扣带皮层  \\
	
	\midrule
	mild cognitive impairment (MCI)  && 轻度认知损伤  \\
	
	\midrule
	missense mutation  && \href{https://baike.baidu.com/item/\%E9%94%99%E4%B9%89%E7%AA%81%E5%8F%98/4086994}{错义突变}  \\
	
	\midrule
	mitogenactivated protein kinase (MAPK)   && 有丝分裂原活化蛋白激酶  \\
	
	\midrule
	Mitral cell   && 僧帽细胞  \\
	
	\midrule
	Molecular layer   && 分子层  \\
	
	\midrule
	monoamine oxidase (MAO)   && 单胺氧化酶  \\
	
	\midrule
	Mossy fiber   && 苔藓纤维  \\
	
	\midrule
	Motor proteins   && 马达蛋白  \\
	
	\midrule
	Müllerian duct   && 副中肾管  \\
	
	\midrule
	Multiple sclerosis   && 多发性硬化  \\
	
	\midrule
	muscle tone   && 肌张力  \\
	
	\midrule
	\makecell{muscle-specific trk-related receptor \\with a kringle domain(MuSK)}   && 跨膜受体蛋白酪氨酸激酶  \\
	
	\midrule
	Mushroom body   && 蕈体  \\
	
	\midrule
	Müllerian inhibiting substance (MIS)   && 副中肾管抑制物  \\
	
	\midrule
	Müller's muscle   && 米勒肌  \\
	
	\midrule
	myelin-associated glycoprotein (MAG)     && 髓磷脂相关糖蛋白   \\
	
	\midrule
	mygdala     && 杏仁核   \\
	
	\midrule
	Myosin heavy chain (MHC)    && 肌球蛋白重链   \\
	
	\midrule
	Neal Cohen   &&  尼尔·科恩 \\
	
	\midrule
	Neoendorphin   &&  新内啡肽 \\
	
	\midrule
	nerve growth factor (NGF)   &&  神经生长因子 \\
	
	\midrule
	Netrin   &&  轴突导向因子 \\
	
	\midrule
	neurokinin-1 (NK1) receptor   && 神经激肽受体-1 \\
	
	\midrule
	Neuroligins (NLs)   && 神经连接蛋白 \\
	
	\midrule
	neurotrophins (NT)   && 神经营养因子 \\
	
	\midrule
	New World monkey   && 新大陆猴 \\
	
	\midrule
	Neurogenin   && 神经元素 \\
	
	\midrule
	Nissl substance   && 尼氏体 \\
	
	\midrule
	Nonassociative learning   && 非联想学习 \\
	
	\midrule
	nonsense mutation   && \href{https://baike.baidu.com/item/%E6%97%A0%E4%B9%89%E7%AA%81%E5%8F%98/4087071}{无义突变} \\
	
	\midrule
	nonsteroidal anti-inflammatory drugs (NSAIDS)   && 非甾体抗炎药 \\
	
	\midrule
	Non-coding RNA (ncRNA)   && \href{https://baike.baidu.com/item/%E9%9D%9E%E7%BC%96%E7%A0%81RNA/10066623}{非编码核糖核酸} \\
	
	\midrule
	noradrenergic (NA)   && 去甲肾上腺素能 \\
	
	\midrule
	norepinephrine transporter (NET)   && 去甲肾上腺素转运蛋白 \\
	
	\midrule
	Nucleus accumbens   && 伏隔核  \\
	
	\midrule
	Nucleus ambiguus   && 疑核  \\
	
	\midrule
	nucleus of Darkshevich   && 达克谢维奇核  \\
	
	\midrule
	Nucleus of the solitary tract   && 孤束核  \\
	
	\midrule
	nucleus reticularis magnocellularis (NRMc)   && 大细胞网状核  \\
	
	\midrule
	nucleus reticularis gigantocellularis (NRGc)   && 巨细胞网状核  \\
	
	\midrule
	nucleus reticularis pontis oralis (NRPo)   && 网状脑桥嘴核  \\
	
	\midrule
	N-Methyl-D-Aspartate (NMDA)   && N-甲基-D-天冬氨酸  \\
	
	\midrule
	N-terminal domain   && N-末端结构域  \\
	
	\midrule
	oculocutaneous albinism II (OCA2)     && 2 型眼皮肤白化病   \\
	
	% III
	\midrule
	Oculomotor nerve (oculomotor)     && 动眼神经   \\
	
	\midrule
	oculomotor vermis (OMV)     && 动眼神经小脑蚓体   \\
	
	\midrule
	Odorant     && 气味剂   \\
	
	\midrule
	Old World monkeys     && 旧大陆猴   \\
	
	\midrule
	Olfactory bulb     && 嗅球   \\
	
	\midrule
	olfactory cortex     && 嗅觉皮层   \\
	
	\midrule
	Olfactory epithelium     && 嗅上皮   \\
	
	\midrule
	olfactory marker protein     && 嗅觉标记蛋白   \\
	
	\midrule
	Olfactory sensory neurons     && 嗅觉感受神经元   \\
	
	\midrule
	olfactory tract     && 嗅束   \\
	
	\midrule
	olfactory tubercle     && 嗅结节   \\
	
	\midrule
	Oligodendrocyte     && 少突细胞   \\
	
	\midrule
	oligodendrocyte-myelin glycoprotein (OMgp)    && 髓鞘少突胶质细胞糖蛋白   \\
	
	\midrule
	oocyte     && \href{https://baike.baidu.com/item/%E5%8D%B5%E6%AF%8D%E7%BB%86%E8%83%9E}{卵母细胞}   \\
	
	\midrule
	optic chiasm (OC)     && 视交叉   \\
	
	\midrule
	Optic foramen     && 视神经孔   \\
	
	\midrule
	optic nerve     && 视神经   \\
	
	\midrule
	optimal linear estimator (OLE)    && 最佳线性估计器   \\
	
	\midrule
	orbital frontal cortex (OFC, OF11)   && 眶额皮层 \\
	
	\midrule
	Orphanin FQ     && 孤啡肽   \\
	
	\midrule
	outer fiberlayer (OFL)     && 外纤维层   \\
	
	\midrule
	outer subventricular zone (OSVZ)     && 室下外侧区   \\
	
	\midrule
	outsider artist     && 世外艺术家   \\
	
	\midrule
	owl monkey     && 夜猴   \\
	
	\midrule
	parahippocampal cortex (PHC)  && 海马旁回   \\
	
	\midrule
	parahippocampal gyrus  (Ph)   && 海马旁回   \\
	
	\midrule
	Parallel fiber     && 平行纤维   \\
	
	\midrule
	paramedian pontine reticular formation     && 桥旁正中网状结构   \\
	
	\midrule
	paraspinals (PSP)     && 椎旁肌   \\
	
	\midrule
	Paraventricular nucleus     && 旁室核   \\
	
	\midrule
	paraventricular nucleus of the hypothalamus (PVH)    && 下丘脑室旁核   \\
	
	\midrule
	Paraxial mesoderm     && 轴旁中胚层   \\
	
	\midrule
	Parietal reach region (PRR)     && 顶叶到达区   \\
	
	\midrule
	parieto-insular vestibular cortex (PIVC)     && 顶-岛前庭皮层   \\
	
	\midrule
	pedunculopontine nucleus (PPN)     && 脑桥脚核   \\
	
	\midrule
	percentage of maximum score possible (POMP)     && 可能的最大分数百分比   \\
	
	\midrule
	Perforant pathway     && 穿通通路   \\
	
	\midrule
	periaqueductal gray matter (PAG)    && 中脑导水管周围灰质   \\
	
	\midrule
	Periglomerular cell     && 球周细胞   \\
	
	\midrule
	Perineural sheath     && 神经鞘   \\
	
	\midrule
	Petrous temporal bone     && 颞骨岩   \\
	
	\midrule
	Phineas Gage     && 菲尼斯·盖奇   \\
	
	\midrule
	Phosphatase and tensin homolog (PTEN)    && 蛋白酪氨酸磷酸酶基因   \\
	
	\midrule
	phospholipase C (PLC)     && 磷脂酶C   \\
	
	\midrule
	photoreceptors     && 光感受器   \\
	
	\midrule
	Pierre Broca     && 皮埃尔·布罗卡   \\
	
	\midrule
	Piriform cortex     && 梨状皮层   \\
	
	\midrule
	Pituitary gland     && 垂体腺   \\
	
	\midrule
	PIWI-interacting RNA  (piRNA)   && 与Piwi蛋白相作用的核糖核酸   \\
	
	\midrule
	PTEN-induced putative kinase 1 (PINK1)     && \href{https://baike.baidu.com/item/PINK1/5405430}{PINK1}   \\
	
	\midrule
	point mutation     && 点突变   \\
	
	\midrule
	polygenic risk scores (PRS)     && 多基因风险评分   \\
	
	\midrule
	Pontine flexure     && 桥曲   \\
	
	\midrule
	positron emission tomography (PET)     && 正电子发射断层成像   \\
	
	\midrule
	posterior commissure     && 后连合   \\
	
	\midrule
	Posterior cranial fossa     && 颅后窝   \\
	
	\midrule
	Posterior Parietal Cortex (PPC)     && 后顶叶皮层   \\
	
	\midrule
	pontine nuclei (PN)    && 	脑桥核   \\
	
	\midrule
	pontomedullary reticular formation (PMRF)   && 	桥髓网状结构   \\
	
	\midrule
	population vector algorithm (PVA)   && 	群体向量法   \\
	
	\midrule
	Positive and Negative Affect Schedule (PANAS)     && 	正性负性情绪量表   \\
	
	\midrule
	Postganglionic neurons     && 	节后神经元   \\
	
	\midrule
	posttraumatic stress disorder (PTSD)     && 	创伤后应激障碍   \\
	
	\midrule
	Prader-Willi syndromes     && 	\href{https://baike.baidu.com/item/\%E5%B0%8F%E8%83%96%E5%A8%81%E5%88%A9%E7%97%87/7472495}{小胖威利症}   \\
	
	\midrule
	Prechordal plate     && 	脊索前板   \\
	
	\midrule
	prefrontal cortex (F46)     && 	前额叶皮层   \\
	
	\midrule
	Preganglionic neurons     && 	节前神经元   \\
	
	\midrule
	Preganglionic autonomic motor neurons (PGC)     && 	节前自主运动神经元   \\
	
	\midrule
	Presenilin-1 (PS1)     && 	早老蛋白-1   \\
	
	\midrule
	presupplementary motor area (pre-SMA)    && 	前辅助运动区   \\
	
	\midrule
	Presynaptic terminal     && 	突触前末梢   \\
	
	\midrule
	Pretectum     && 	前顶盖   \\
	
	\midrule
	pre-supplementary motor area (Pre-SMA)     && 	前辅助运动区   \\
	
	% 小脑
	\midrule
	Primary fissure (PF)   && 原裂  \\
	
	\midrule
	primary motor cortex (M1)   && 初级运动皮层  \\
	
	% 启动效应是指快速呈现的刺激(启动刺激)对随后出现的第二个刺激(目标刺激)的处理产生的积极或消极的影响
	\midrule
	Priming   && 启动  \\
	
	\midrule
	Principles of Psychology   && 《心理学原理》  \\
	
	\midrule
	progenitor cell (P)   && 祖细胞  \\
	
	\midrule
	protein kinase A (PKA)   && 蛋白激酶A  \\
	
	\midrule
	protein kinase C (PKC)   && 蛋白激酶C  \\
	
	\midrule
	Rapsyn   && 突触受体相关蛋白  \\
	
	\midrule
	Posttetanic potentiation   && 强直刺激后增强  \\
	
	\midrule
	primary auditory cortex (A1)   && 初级听觉皮层  \\
	
	\midrule
	Purkinje cell (PC)   && 浦肯野细胞  \\
	
	\midrule
	putamen (Put)   && 壳核  \\
	
	\midrule
	Pyramidal decussation   && 锥体交叉  \\
	
	\midrule
	quadriceps (QUAD)   && 四头肌  \\
	
	\midrule
	Ragnar Granit   && 拉格纳·格拉尼特  \\
	
	\midrule
	rapid eye movement (REM)   && 快速眼动  \\
	
	\midrule
	rapidly adapting low-threshold mechanoreceptors (RALTMRs)  && 快适应低阈值机械感受器  \\
	
	\midrule
	recalibrated feedback intention-trained Kalman filter (RF)   && 重新校准反馈意图训练卡尔曼滤波器  \\
	
	\midrule
	receiver operating characteristic (ROC)   && 受试者工作特征  \\
	
	\midrule
	Receptive Field (RF)   && 感受野  \\
	
	\midrule
	Receptor tyrosine kinase   && 受体酪氨酸激酶  \\
	
	\midrule
	rectus muscle   && 直肌  \\
	
	\midrule
	Renshaw cell   && 闰绍细胞  \\
	
	\midrule
	reticulospinal tracts (RST)   && 网状脊髓束  \\
	
	\midrule
	retrosplenial cortex   && 压后皮层  \\
	
	\midrule
	Rett syndrome   && \href{https://baike.baidu.com/item/\%E9%9B%B7%E7%89%B9%E9%9A%9C%E7%A2%8D/22296155}{雷特综合症}  \\
	
	\midrule
	Ribonucleic Acid   && \href{https://baike.baidu.com/item/\%E6%A0%B8%E7%B3%96%E6%A0%B8%E9%85%B8/541373}{核糖核酸}   \\
	
	\midrule
	Ribosomal RNA (rRNA)   && \href{https://baike.baidu.com/item/\%E6%A0%B8%E7%B3%96%E4%BD%93RNA/3752312}{核糖体核糖核酸}  \\
	
	\midrule
	Richard Thompson   && 理查德·汤普森  \\
	
	\midrule
	right flexor motor neurons (rFmn)   && 右屈肌运动神经元  \\
	
	\midrule
	Robert Lockhart   && 罗伯特·洛哈特  \\
	
	\midrule
	Robert Stickgold   && 罗伯特·史蒂克戈德  \\
	
	\midrule
	robust nucleus of the archistriatum (RA)   && 古纹状体强健核  \\
	
	\midrule
	rod bipolar   && 杆状双极细胞  \\
	
	\midrule
	rod cell   && 视杆细胞  \\
	
	\midrule
	René Spitz   && 雷诺·史必兹  \\
	
	\midrule
	Roger Sperry   && 罗杰·斯佩里  \\
	
	\midrule
	Rostral auditory cortex (R)   && 嘴侧听觉皮层  \\
	
	\midrule
	Rostrotemporal auditory cortex (R)   && 前颞听觉皮层 \\
	
	\midrule
	Rubrospinal tract   && 红核脊髓束 \\
	
	\midrule
	sagittal plane   && 矢状面 \\
	
	\midrule
	Sally-Anne test   && 萨莉-安妮测试 \\
	
	\midrule
	Santiago Ramony Cajal   && 圣地亚哥·拉蒙-卡哈尔 \\
	
	\midrule
	sarcomere length   && 肌节长度 \\
	
	\midrule
	Sarcoplasmic reticulum   && 肌质网 \\
	
	\midrule
	savant syndrome   && \href{https://baike.baidu.com/item/\%E5%AD%A6%E8%80%85%E7%BB%BC%E5%90%88%E7%97%87/4453123}{学者综合症} \\
	
	\midrule
	selective serotonin reuptake inhibitors   && 选择性血清再吸收抑制剂 \\
	
	\midrule
	sensor molecule   && 受体分子 \\
	
	\midrule
	sensory threshold   && 受体分子 \\
	
	\midrule
	Serous gland   && 浆液腺	 \\
	
	\midrule
	Schaffer collateral pathway   && 谢弗旁路 \\
	
	\midrule
	Schwann cell   && 施旺细胞 \\
	
	\midrule
	serotonin reuptake transporter   && 血清素再摄取转运蛋白 \\
	
	\midrule
	serotonin transporter   && 血清素转运蛋白 \\
	
	\midrule
	sex-determining region on Y (SRY)   && Y染色体性别决定区 \\
	
	\midrule
	sex-linked inheritance   && \href{https://baike.baidu.com/item/\%E4%BC%B4%E6%80%A7%E9%81%97%E4%BC%A0/4078141}{伴性遗传} \\
	
	\midrule
	Sexual dimorphism   && 两性异形 \\
	
	\midrule
	sexually dimorphic nucleus of the preoptic area (SDN-POA)   && 视前区性二态核团 \\
	
	\midrule
	shell shock   && 战斗疲劳症 \\
	
	\midrule
	Shereshevski   && 舍雷舍夫斯 \\
	
	\midrule
	sickle cell anemia   && \href{https://baike.baidu.com/item/%E9%95%B0%E5%88%80%E5%9E%8B%E7%BB%86%E8%83%9E%E8%B4%AB%E8%A1%80%E7%97%85}{镰状细胞贫血} \\
	
	\midrule
	silent mutation   && \href{https://baike.baidu.com/item/%E6%B2%89%E9%BB%98%E7%AA%81%E5%8F%98/9716444}{沉默突变} \\
	
	\midrule
	silent nociceptor  && 寂静性伤害性感受器 \\
	
	\midrule
	small noncoding RNA   && \href{https://wenku.baidu.com/view/60f60e595427a5e9856a561252d380eb63942371.html?_wkts_=1693876684239}{小非编码核糖核酸} \\
	
	\midrule
	small nuclear RNA (snRNA)   && \href{https://baike.baidu.com/item/%E5%B0%8F%E6%A0%B8RNA/10326792}{小核核糖核酸} \\
	
	\midrule
	Social anxiety disorder   && 社交焦虑症 \\
	
	\midrule
	Social Avoidance and Distress Scale   && 社交回避及苦恼量表 \\
	
	\midrule
	soleus muscle   && 比目鱼肌 \\
	
	\midrule
	Solitary nucleus   && 孤束核 \\
	
	\midrule
	spike timing–dependent plasticity   && 脉冲时序的可塑性 \\
	
	\midrule
	spinal nucleus of the bulbocavernosus (SNB)  && 球海绵体肌脊髓核 \\
	
	\midrule
	Spinal proprioceptor   && 脊髓本体感受器 \\
	
	\midrule
	Spinal trigeminal nucleus   && 三叉神经脊髓核 \\
	
	\midrule
	spinobulbar muscular atrophy (SBMA)   && 脊延髓肌萎缩症 \\
	
	\midrule
	spinocerebellar ataxias (SCAs)   && 脊髓小脑共济失调 \\
	
	\midrule
	Stanley Cohen   && 斯坦利·科恩 \\
	
	\midrule
	Stellate cell   && 星状细胞 \\
	
	\midrule
	Stria terminalis  && 终纹 \\
	
	\midrule
	Striatum (STR)  && 纹状体 \\
	
	\midrule
	STS-temporalparietal junction   && 颞上沟-颞顶联合区 \\
	
	\midrule
	Subfornical organ   && 穹窿下器 \\
	
	\midrule
	sublenticular extended amygdala (SLEA)   && 近管状延伸杏仁核 \\
	
	\midrule
	subplate (SP)   && 底板 \\
	
	\midrule
	substance P   && P 物质(肽物质) \\
	
	\midrule
	\makecell{substantia nigra and ventral tegmental \\area of the midbrain (SN/VTA)}   && 黑质/中脑腹侧被盖区 \\
	
	\midrule
	substantia nigra pars compacta (SNc)  && 黑质紧密区 \\
	
	\midrule
	substantia nigra pars reticulata (SNr)  && 黑质网状部 \\
	
	\midrule
	subthalamic nucleus (STN)   && 丘脑底核 \\
	
	\midrule
	subventricular zone (SVZ)   && 室下区 \\
	
	\midrule
	Sulcus limitans   && 界沟 \\
	
	\midrule
	Superior cervical ganglion   && 颈上神经节 \\
	
	\midrule
	superior colliculus   && 上丘 \\
	
	\midrule
	superior oblique   && 上斜肌 \\
	
	\midrule
	Superior orbital fissure   && 眶上裂 \\
	
	\midrule
	superior rectus   && 上直肌 \\
	
	\midrule
	superior temporal sulcus (STS)   && 颞上沟 \\
	
	\midrule
	superior view   && 俯视图 \\
	
	\midrule
	supplementary motor area (SMA)   && 辅助运动区 \\
	
	\midrule
	Suppressor of cytokine signaling 3 (SOCS3)  && 细胞因子信号通路抑制因子3 \\
	
	\midrule
	supraoptic nucleus (SO)  && 视上核 \\
	
	\midrule
	Synaptobrevin (VAMP)   && 小突触囊泡蛋白 (synaptic vesicle-associated membrane protein, 突触小泡缔合性膜蛋白) \\
	
	\midrule
	Synaptotagmin   && 突触结合蛋白 \\
	
	\midrule
	Taste bud   && 	味蕾  \\
	
	\midrule
	Taste pore   && 	味孔  \\
	
	\midrule
	tetraethylammonium (TEA)   && 四乙胺  \\
	
	\midrule
	temporal pole   && 颞极  \\
	
	\midrule
	Terminal cisterna   && 终池  \\
	
	\midrule
	tetrodotoxin (TTX)   && 河豚毒素  \\
	
	\midrule
	theory of mind   && \href{https://baike.baidu.com/item/\%E5%BF%83%E6%99%BA%E7%90%86%E8%AE%BA/8719175}{心智理论}   \\
	
	\midrule
	Thomas Hunt Morgan  && 托马斯·亨特·摩尔根  \\
	
	\midrule
	Thymine (T)  && 胸腺嘧啶  \\
	
	% 胫(小腿)
	\midrule
	tibialis anterior (TIB) && 胫前肌  \\
	
	\midrule
	Timothy syndrome  && 蒂莫西综合症  \\
	
	\midrule
	tonotopic map   && 音调拓扑图  \\
	
	\midrule
	Torsten Wiesel   && 托斯坦·威泽尔  \\
	
	\midrule
	Tourette syndrome   && 图雷特综合症  \\
	
	\midrule
	toxins   && 毒素  \\
	
	\midrule
	trace amine-associated receptors (TAAR)   && 微量胺相关受体  \\
	
	\midrule
	Transcranial magnetic stimulation (TMS)   && 经颅磁刺激  \\
	
	\midrule
	transfer RNA (tRNA)   && \href{https://baike.baidu.com/item/\%E8%BD%AC%E8%BF%90RNA/5270033}{转运核糖核酸}  \\
	
	\midrule
	transient receptor potential (TRP)   && 瞬时受体电位  \\
	
	\midrule
	transient receptor potential vanilloid (TRPV)   && 瞬时受体电位香草醛受体  \\
	
	\midrule
	transit amplifying cell   && 过渡放大细胞  \\
	
	\midrule
	transmembrane AMPA receptor regulatory proteins (TARP)  && 跨膜AMPA受体调控蛋白  \\
	
	\midrule
	Transneuronal degeneration  && 跨神经元变性  \\
	
	\midrule
	transverse temporal gyri (Heschl's gyrus)   && 颞横回  \\
	
	\midrule
	transverse tubules   && 横小管  \\
	
	\midrule
	trapezoid body   && 斜方体  \\
	
	\midrule
	Trigeminal nerve   && 三叉神经  \\
	
	\midrule
	Trochlear nerve (trochlear)   && 滑车神经  \\
	
	\midrule
	Tufted cell   && 簇状细胞  \\
	
	\midrule
	tyrosine kinases (Trk)   && 酪氨酸激酶  \\
	
	\midrule
	unconditioned stimulus (US)  && 非条件刺激  \\
	
	\midrule
	Uridine (U)     &&  \href{https://baike.baidu.com/item/%E5%B0%BF%E8%8B%B7/4644045}{尿苷}  \\
	
	\midrule
	US Food and Drug Administration (FDA)     &&  美国食品和药物管理局  \\
	
	\midrule
	utilization behavior   && 使用性行为  \\
	
	\midrule
	V1   && 初级视觉皮层  \\
	
	\midrule
	vagus nerves   && 迷走神经  \\
	
	\midrule
	ventral caudate (vCD)   && 腹侧尾状核  \\
	
	\midrule
	ventral intraparietal area (VIP)   && 顶内沟腹侧区  \\
	
	\midrule
	Ventral Nucleus of the Trapezoid Body(MNTB)   && 斜方体腹侧核  \\
	
	\midrule
	Ventral posterior medial nucleus of thalamus   && 丘脑腹后核  \\
	
	\midrule
	ventral premotor cortex (PMv)   && 腹侧前运动皮层  \\
	
	\midrule
	ventral subparaventricular zone (vSPZ)  && 腹侧脑室下区  \\
	
	% 中脑
	\midrule
	ventral tegmental area (VTA)   && 腹侧被盖区  \\
	
	\midrule
	ventricular zone (VZ)   && 脑室区  \\
	
	\midrule
	ventrolateral funiculus (VLF)   && 腹外侧索  \\
	
	\midrule
	ventrolateral prefrontal cortex (VLPFC, F47)   && 腹外侧前额叶皮层  \\
	
	\midrule
	ventrolateral preoptic nuclei (VLPFC, F47)   && 腹外侧视前核  \\
	
	% 腹侧视丘
	\midrule
	Ventrolateral thalamus   && 丘脑腹外侧核  \\
	
	\midrule
	ventromedial hypothalamus (VMH)  && 腹内侧下丘脑  \\
	
	\midrule
	vergence movement   && 聚散运动  \\
	
	\midrule
	Vernon Mountcastle   && 弗农·芒卡斯尔  \\
	
	\midrule
	version movement   && 同向运动  \\
	
	\midrule
	vesicular ACh transporter (VAChT)   && 囊泡乙酰胆碱转运蛋白  \\
	
	\midrule
	vesicular glutamate transporter (V-GluT, VGlut)   && 囊泡谷氨酸转运蛋白  \\
	
	\midrule
	vesicular monoamine transporter (VMAT2)   && 囊泡单胺转运蛋白  \\
	
	\midrule
	vestibular nuclei   && 前庭核  \\
	
	\midrule
	vestibulospinal tracts (VST)   && 前庭脊髓束  \\
	
	\midrule
	vestibulo-ocular reflexes (VOR)   && 前庭眼反射  \\
	
	\midrule
	Viktor Hamburger   && 维克多·汉堡  \\
	
	\midrule
	visual posterior sylvian area (VPS)   && 视后外侧裂  \\
	
	\midrule
	Vomeronasal organ (VNO)  && 犁鼻器  \\
	
	\midrule
	von Economo   && 冯·伊克诺莫  \\
	
	\midrule
	Wallerian degeneration  && 华勒氏变性  \\
	
	\midrule
	Wallerian degeneration slow (Wlds) && 华勒氏慢变性  \\
	
	\midrule
	Walter B. Cannon  && 沃尔特•坎农  \\
	
	\midrule
	Walter Hess  && 沃尔特·赫斯  \\
	
	\midrule
	Weber  && 韦伯  \\
	
	\midrule
	What/Who pathway/stream  && 内容通路  \\
	
	\midrule
	Where/How pathway/stream && 空间通路  \\
	
	\midrule
	William de Kooning && 威廉·德·库宁  \\
	
	\midrule
	William James && 威廉·詹姆斯  \\
	
	\midrule
	Williams syndrome && 威廉综合症  \\
	
	% 沃尔夫管
	\midrule
	Wolffian duct && 中肾管  \\
	
	\midrule
	Wolfram Schultz && 沃尔夫勒姆·舒尔茨  \\
	
	\midrule
	X-linked recessive && \href{https://baike.baidu.com/item/X%E8%BF%9E%E9%94%81%E9%9A%90%E6%80%A7/53170799}{X连锁隐性}  \\
	
	\midrule
	Yngve Zotterman && 左特曼  \\
	
	\midrule
	zona limitans intrathalamica (ZLI) && 限制性间脑区  \\
	
	\midrule
	β-actin && 肌动蛋白  \\
	
	\midrule
	γ-secretase && γ分泌酶  \\
	
	
	\bottomrule  

\end{longtable}
%}
%\end{table}%

