\chapter{耳蜗的听觉处理}
% PDF所在目录: /data2/whd/win10/learn/neuro/neuro_神经科学原理_28_中枢神经系统的听觉处理.pdf

\section{耳朵具有三个功能部分}

\section{听力始于耳朵对声音能量的捕捉}

\section{耳蜗的流体动力学和机械装置向受体细胞提供机械刺激}
\subsection{基底膜是声频的机械分析仪}
\subsection{科蒂氏器官是耳蜗中机电转导的部位}

\section{毛细胞将机械能转化为神经信号}
\subsection{发束的偏转引发机电转导}
\subsection{机械力直接打开转导通道}
\subsection{直接机电转换速度很快}
\subsection{耳聋基因提供了机械传导机制的组成部分}

\section{动态反馈机制决定毛细胞的敏感性}
\subsection{毛细胞被调整到特定的刺激频率}
\subsection{毛细胞适应持续刺激}
\subsection{声能在耳蜗中被机械放大}
\subsection{霍普夫分岔为声音检测提供了一般原理}

\section{毛细胞使用专门的带状突触}

\section{听觉信息最初通过耳蜗神经流动}
\subsection{螺旋神经节中的双极神经元支配耳蜗毛细胞}
\subsection{耳蜗神经纤维编码刺激频率和水平}

\section{感音神经性听力损失很常见,但可以治疗}

\section{要点}
\subsection{选读}
\subsection{参考文献}