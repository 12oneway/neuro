\chapter{耳蜗的听觉处理}
人类的经验因能够区分各种不同的声音而变得丰富——从耳语的亲切感到谈话的热情,从交响乐的复杂性到体育场的轰鸣声。 当耳蜗的感觉细胞(内耳的感受器官)将声能转换为电信号并将其转发给大脑时,听力就开始了。 我们识别声音细微差异的能力源于耳蜗区分频率分量、振幅和相对时间的能力。

听力取决于毛细胞的显着特性,毛细胞是内耳的细胞麦克风。 毛细胞将声音引起的机械振动转换为电信号,然后将其传递给大脑进行解释。 毛细胞可以测量原子尺寸的运动,并转换从静态输入到频率为数十千赫兹的刺激。 值得注意的是,毛细胞还可以充当增强听觉灵敏度的机械放大器。 每对耳蜗都含有大约 16,000 个这样的细胞。 毛细胞及其神经支配的退化是造成工业化国家约 10\% 人口听力损失的主要原因。



\section{耳朵具有三个功能部分}
声音由弹性介质(空气)以大约 340 m/s 的速度传播的交替压缩和稀疏组成。 这种压力变化波携带的机械能源于我们的发声器官或其他声源对空气产生的功。 机械能被捕获并传输到受体器官,在那里它被转换成适合神经分析的电信号。 这三个任务分别与外耳、中耳和内耳的耳蜗相关(图 26-1)。


\section{听力始于耳朵对声音能量的捕捉}

\section{耳蜗的流体动力学和机械装置向受体细胞提供机械刺激}
\subsection{基底膜是声频的机械分析仪}
\subsection{科蒂氏器官是耳蜗中机电转导的部位}

\section{毛细胞将机械能转化为神经信号}
\subsection{发束的偏转引发机电转导}
\subsection{机械力直接打开转导通道}
\subsection{直接机电转换速度很快}
\subsection{耳聋基因提供了机械传导机制的组成部分}

\section{动态反馈机制决定毛细胞的敏感性}
\subsection{毛细胞被调整到特定的刺激频率}
\subsection{毛细胞适应持续刺激}
\subsection{声能在耳蜗中被机械放大}
\subsection{霍普夫分岔为声音检测提供了一般原理}

\section{毛细胞使用专门的带状突触}

\section{听觉信息最初通过耳蜗神经流动}
\subsection{螺旋神经节中的双极神经元支配耳蜗毛细胞}
\subsection{耳蜗神经纤维编码刺激频率和水平}

\section{感音神经性听力损失很常见,但可以治疗}

\section{要点}
\subsection{选读}
\subsection{参考文献}