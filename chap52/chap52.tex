\chapter{学习和记忆}

加夫列尔·加西亚·马尔克斯 (Gabriel Garcia Márquez) 在他的巨著小说《百年孤独》中描述了一种奇怪的瘟疫,它侵入了一个小村庄并夺走了人们的记忆。 村民们首先失去了个人的记忆,然后是共同物品的名称和功能。 为了对抗瘟疫,一个人在家里的每一件物品上都贴上了文字标签。 但他很快意识到这种策略是徒劳的,因为瘟疫最终甚至摧毁了他对单词和字母的了解。

这个虚构的事件提醒我们学习和记忆在日常生活中的重要性。 学习是指因获取有关世界的知识而导致的行为变化,而记忆是指对知识进行编码、存储和随后检索的过程。 Marquez 的故事挑战我们想象没有学习和记忆能力的生活。 我们会忘记我们曾经认识的人和地方,不再能够使用和理解语言或执行我们曾经学过的运动技能; 我们不会回忆起我们生命中最快乐或最悲伤的时刻,甚至会失去我们的个人认同感。 学习和记忆对于人和动物的全面运作和独立生存至关重要。

1861 年,皮埃尔·保罗·布罗卡 (Pierre Paul Broca) 发现左额叶后部(布罗卡区)受损会导致特定的语言缺陷。 此后不久,人们就清楚其他心理功能,如知觉和随意运动,也由大脑的离散部分调节(第 1 章)。 这自然引出了一个问题:是否存在与记忆有关的离散神经系统? 如果是这样,是否存在“记忆中心”,或者记忆处理是否广泛分布于整个大脑?

与认知功能局限于大脑的普遍观点相反,许多学习学习的学生怀疑记忆是局部的。 事实上,直到 20 世纪中叶,许多心理学家都怀疑记忆是一种独立的功能,独立于感知、语言或运动。 持续怀疑的原因之一是记忆存储涉及大脑的许多不同部分。 然而,我们现在意识到,这些地区并非都同样重要。 有几种根本不同的记忆类型,大脑的某些区域对于编码某些类型的记忆比对其他区域更为重要。

在过去的几十年里,研究人员在对学习和记忆的分析和理解方面取得了重大进展。 在本章中,我们重点研究正常的人类记忆行为、因受伤或手术导致脑损伤后的扰动,以及使用功能磁共振成像 (fMRI) 和细胞外电生理记录测量学习和记忆回忆期间的大脑活动。 这些研究产生了三个主要见解。

首先,学习和记忆有多种形式。 每种形式的学习和记忆都具有独特的认知和计算特性,并由不同的大脑系统支持。 其次,记忆涉及编码、存储、检索和巩固。 最后,记忆的不完美和错误可以提供有关学习和记忆的性质和功能以及记忆在指导行为和规划未来方面所起的基本作用的线索。

记忆可以按两个维度分类:(1) 存储的时间过程和 (2) 存储信息的性质。 在本章中,我们考虑存储的时间过程。 在接下来的两章中,我们主要基于对动物模型的研究,重点介绍不同形式的学习和记忆的细胞、分子和基于电路的机制。

\section{短期和长期记忆涉及不同的神经系统}
\subsection{短期记忆维持与即时目标相关的信息的瞬态表示}
当我们思考记忆的本质时,我们通常会想到 William James 称之为“记忆本身”或“次要记忆”的长期记忆。 也就是说,我们认为记忆是“对先前的精神状态的了解,它已经从意识中消失了”。 这种知识依赖于持久记忆痕迹的形成,即使其内容已经在很长一段时间内失去意识,表征仍然存在。

然而,并非所有形式的记忆都构成“以前的心理状态”。 事实上,存储信息的能力取决于一种称为工作记忆的短期记忆形式,它可以保持当前(尽管是短暂的)目标相关知识的表征。 在人类中,工作记忆至少由两个子系统组成——一个用于语言信息,另一个用于视觉空间信息。 这两个子系统的功能由称为执行控制过程的第三个系统协调。 执行控制过程被认为将注意力资源分配给语言和视觉空间子系统,并监控、操纵和更新存储的表征。

当我们试图将基于语音的(语音)信息保持在有意识的意识中时,我们会使用语言子系统,就像我们在输入密码之前在心里排练密码一样。 语言子系统由两个交互组件组成:一个代表语音知识的存储和一个在我们需要时保持这些表示活动的排练机制。 语音存储依赖于后顶叶皮质,复述部分依赖于布罗卡区的发音过程。

工作记忆的视觉空间子系统保留视觉对象的心理图像和对象在空间中的位置。 空间和物体信息的排练被认为涉及额叶和前运动皮质在顶叶、颞下和枕叶皮质中对这些信息的调制。

非人类灵长类动物的单细胞记录表明,在几秒钟内,一些前额叶神经元保持空间表征,另一些保持物体表征,还有一些前额叶神经元代表空间和物体知识的整合。 虽然与物体工作记忆相关的神经元倾向于位于腹外侧前额叶皮层,而与空间知识相关的神经元倾向于位于背外侧前额叶皮层,但所有三类神经元都存在于前额叶的两个子区域中(图 52-1)。

因此,工作记忆涉及激活存储在根据信息内容而变化的专门皮层区域中的信息表征,以及激活前额叶皮层中的一般控制机制。 工作记忆中的前额叶控制信号进一步依赖于与纹状体的相互作用和来自中脑的上升多巴胺能输入。

\subsection{存储在短期记忆中的信息被选择性地转移到长期记忆中}
在 20 世纪 50 年代中期,关于长期记忆的神经基础的令人吃惊的新证据出现在对接受双边切除海马体和内侧颞叶邻近区域作为癫痫治疗的患者的研究中。 第一个也是研究最多的案例是一位名叫 H.M. 由心理学家布伦达米尔纳和外科医生威廉斯科维尔研究。 (在 H.M. 于 2008 年 12 月 2 日去世后,他的全名 Henry Molaison 被公诸于世。)

H.M. 由于 7 岁时在一次自行车事故中遭受的脑损伤而导致无法治愈的颞叶癫痫,多年来一直深受其害。 成年后,他的癫痫发作使他无法工作或过正常的生活,在 27 岁时,他接受了手术。 Scoville 移除了被认为与癫痫发作有关的大脑区域,包括海马结构、杏仁核和双侧颞叶皮层的多模式联合区域的一部分(图 52-2)。 手术后,H.M. 的癫痫发作得到了更好的控制,但他留下了毁灭性的记忆缺陷(或健忘症)。 H.M. 的缺陷之所以如此引人注目,是因为它的特殊性。

他仍然有正常的工作记忆,持续数秒或数分钟,这表明内侧颞叶对于瞬时记忆来说不是必需的。 他对手术前发生的事件也有长期记忆。 例如,他记得自己的名字、从事的工作和童年经历。 此外,他保留了对语言的掌握,包括他的词汇量,这表明语义记忆——关于人、地方和事物的事实知识——得以保留。 他的智商没有变化,在正常范围内。

什么 H.M. 现在缺乏,而且非常缺乏,是将新信息转移到长期记忆中的能力,这种缺陷被称为顺行性遗忘症。 他无法长时间保留有关他刚刚遇到的人、地点或物体的信息。 当被要求记住一个新的电话号码时,H.M. 由于他完整的工作记忆,他可以立即重复几秒钟到几分钟。 但是当分心时,即使是短暂的,他也会忘记号码。 H.M. 无法认出他在手术后遇到的人,即使他一次又一次地遇到他们。 几年来,他每个月都见到米尔娜,但每次她走进房间,他的反应就好像他以前从未见过她一样。 H.M. 不是唯一的。 所有内侧颞叶边缘联合区双侧广泛损伤的患者都表现出相似的长期记忆缺陷。

H.M. 这是一个历史性的案例,因为他的缺陷提供了记忆与内侧颞叶(包括海马体)之间的第一个明确联系。 拉里·斯奎尔 (Larry Squire) 和其他人对海马体脑损伤患者的后续研究证实了海马体在记忆中的核心作用。 观察到 H.M. 和其他患有内侧颞叶损伤的人在新记忆的形成方面存在严重缺陷,而旧记忆的提取基本保持完整,这表明记忆必须随着时间的推移从海马体和内侧颞叶转移到其他大脑结构。 这些研究提出了四个核心问题,这些问题至今仍在推动记忆研究:首先,内侧颞叶记忆系统的功能是什么? 第二,不同次区域在这个体系中的作用是什么? 第三,这些子区域如何与其他脑回路协同工作以支持不同形式的记忆? 第四,依赖海马体的记忆最终存储在哪里?


\section{内侧颞叶对情景式长期记忆至关重要}
关于 H.M. 的重要发现 是长期记忆的形成仅针对某些类型的信息受损。 H.M. 和其他内侧颞叶受损的患者能够像健康受试者一样形成和保留某些类型的持久记忆。

例如,H.M. 他在镜子里看着星星和他的手,学会了画星星的轮廓(图 52-3)。 就像健康的受试者学习重新映射手眼协调一样,H.M. 最初犯了很多错误,但经过几天的训练,他的表现没有错误,可以与健康受试者相媲美。 然而,他并没有自觉地记得自己曾经完成过这项任务。

健忘症患者的长期记忆形成不仅限于运动技能。 这些患者保留了简单的反身学习,包括习惯化、敏感化和某些形式的条件反射(将在本章后面讨论)。 此外,他们能够提高他们在某些感知和概念任务上的表现。 例如,他们擅长一种称为启动的记忆形式,在这种记忆中,通过先前的接触可以改善对单词或对象的感知或对单词或对象含义的访问。 因此,当仅显示先前学习过的单词的前几个字母时,失忆症患者能够产生与正常受试者相同数量的学习过的单词,即使失忆症患者对最近遇到的单词没有有意识的记忆(图 52) –4).

健忘症患者这种选择性受损的表现模式引发了关于如何对这些不同形式的记忆进行分类的问题:区分内侧颞叶损伤后存活的记忆和未损伤记忆的关键特征是什么? Squire 及其同事的早期理论表明,一个关键因素可能是有意识的意识——对内侧颞叶的损伤似乎会损害可以有意识地访问、可以用语言报告或表达的记忆形式,同时留下完整的记忆形式 不能。 因此,依赖于内侧颞叶的记忆通常被称为外显(或陈述性)记忆。 外显记忆可进一步分为情景记忆(个人经历的记忆或自传体记忆)和语义记忆(对事实的记忆)。 情景记忆是指我们能够及时记住瞬间的丰富细节,包括有关发生的事情、时间和地点的信息。 例如,情景记忆用于回忆我们昨天看到了春天的第一朵花,或者几个月前我们听到了贝多芬的《月光奏鸣曲》。 语义记忆用于回忆单词或概念的含义,以及其他事实。

认知心理学家通过使用记忆表达方式不同的任务,在健康受试者的不同记忆形式之间发现了类似的区别。 一种类型是一种无意识的记忆形式,在执行任务时很明显。 这种形式的记忆通常称为内隐记忆(也称为非陈述性或程序性记忆)。 内隐记忆通常以自动方式表现出来,主体几乎没有意识处理。 不同的形式会引起启动、技能学习、习惯记忆和条件反射(图 52-5)。 显式内存被认为是高度灵活的; 在不同的情况下,可以关联多条信息。 然而,内隐记忆与学习发生的原始条件紧密相关。

术语“外显记忆”和“内隐记忆”用于描述两种广泛的记忆形式,这两种记忆的标志性行为特征和神经基础不同。 这些形式的记忆可以并行获得。 例如,一个人可能会对昨天进入一家面包店时闻起来有多香形成明确的记忆,而与此同时,一个人可能会在看到面包店的图片时产生唾液分泌增加的自动条件反应。 此外,我们现在认为,这些形式的记忆虽然截然不同,但通常会相互作用以支持行为,尽管它们相互作用的确切性质和范围是正在进行的调查的主题。

关于意识意识在记忆中的作用以及它是否确实是内侧颞叶支持的记忆的必要特征,也存在持续的争论。 这些争论是由越来越多的工作驱动的,这些工作表明外显记忆所必需的相同内侧时间回路对于某些形式的内隐记忆也是必需的(如下所述)。 事实上,虽然情景记忆通常通过要求受试者报告他们记忆的内容来评估,但意识可及性是否是记忆本身不可或缺的特征仍然未知。 尽管如此,内隐记忆和外显记忆之间的区别在区分记忆形式方面发挥了重要的历史作用,并且仍然为考虑记忆的神经基础提供了一个富有成效的框架。 因此,我们在这里使用术语“外显记忆”和“内隐记忆”来区分这两种记忆形式以及它们所基于的主观体验和行为的类别。 在接下来的部分中,我们将重点关注情景记忆,它一直是健忘症患者和健康个体的大量认知神经科学研究的目标。

\subsection{情景记忆处理涉及编码、存储、检索和整合}
情景记忆已得到广泛研究,它提供了一个了解大脑如何构建、存储和检索我们生活中情节细节的窗口。 我们现在知道大脑没有单一的情景记忆长期存储。 相反,任何知识项的存储都广泛分布在处理记忆内容的不同方面的许多大脑区域中,并且可以独立访问(通过视觉、口头或其他感官线索)。 其次,情景记忆至少由四种相关但截然不同的处理类型介导:编码、存储、巩固和检索。

编码是在新记忆形成过程中最初获取和处理新信息的过程。 这种处理的程度对于确定所学材料的记忆程度至关重要。 为了让记忆持久并被牢记,传入的信息必须经过心理学家 Fergus Craik 和 Robert Lockhart 所说的“深度”编码。 这是通过关注信息并将其与已经建立的记忆相关联来实现的。 当一个人有记忆的动机时,记忆编码也会更强,无论是因为信息具有特定的情感或行为相关性(例如,在愉快的第一次约会时记得特别美味的一餐)还是信息本身是中性的但与某事相关联 有意义(例如,记住那家餐馆的位置)。

存储是指神经机制和站点,新获取的信息通过这些机制和站点随着时间的推移保留为持久的记忆。 长期存储的显着特征之一是它似乎具有几乎无限的容量。 相比之下,工作记忆存储非常有限; 心理学家认为,人类的工作记忆在任何时候只能保存几条信息。

合并是将临时存储且仍然不稳定的信息转换为更稳定形式的过程。 正如我们将在接下来的两章中学到的,巩固涉及基因表达和蛋白质合成,它们会引起突触的结构变化。

最后,检索是回忆存储的信息的过程。 它涉及让人们想起存储在不同站点的不同类型的信息。 记忆的检索很像感知。 它是一个建设性的过程,因此容易受到扭曲,就像感知受到幻觉的影响一样(专栏 52-1)。 当记忆被检索时,它会再次活跃起来,为旧记忆再次编码提供机会。 由于检索是建设性的,因此对检索到的记忆进行重新编码可能与原始记忆不同。 例如,重新编码可以包括来自旧记忆的信息以及检索它的新上下文。 这种重新编码允许在记忆中连接不同时刻的记忆,但它也为记忆中的错误打开了大门,如本章后面所述。

当检索线索提醒个人连接编码体验元素的事件的情节性质时,信息检索最有效。 例如,在一个经典的行为实验中,Craig Barclay 及其同事要求一些受试者对诸如“The man lifted the piano”之类的句子进行编码。 在后来的提取测试中,“重的东西”比“声音好听的东西”更能有效地回忆起钢琴。 然而,其他受试者编码了“The man tuned the piano”这句话。 对他们来说,“声音好听的东西”是比“重的东西”更有效的钢琴检索提示,因为它更好地反映了最初的体验。 检索,尤其是外显记忆的检索,也部分依赖于工作记忆。

\subsection{情景记忆涉及内侧颞叶和联合皮层之间的相互作用}
尽管过去几十年对健忘症患者的研究加深了我们对各种类型记忆的理解,但内侧颞叶损伤会影响记忆的所有四种操作——编码、存储、巩固和检索——因此通常很难辨别如何 内侧颞叶对每个都有贡献。 fMRI 允许我们在建立新记忆或检索现有记忆的过程中扫描大脑活动,从而识别在不同过程中活跃的特定区域(第 6 章)。

使用 fMRI 研究编码的常用方法是后续记忆范式。 在一个典型的后续记忆任务中,人类受试者在使用 fMRI 扫描时一次观察一系列刺激(例如,单词或图片),通常是在从事掩饰任务(例如,确定图片是彩色的还是 黑与白)。 然后在扫描仪之外测试受试者对刺激的记忆,使研究人员能够将所有编码事件分类为后来记住的事件和后来忘记的事件。 fMRI 扫描显示,与忘记的项目相比,记住的项目与编码过程中海马体的更多活动相关。 这种差异在大脑其他部分的同步活动中也很明显,包括前额叶、压后和顶叶皮质。 通常,在记忆编码过程中,这些区域的活动与海马体的活动时时刻刻都在发生变化,这表明这些区域在功能上是相互联系的(图 52-6)。

这些 fMRI 研究结果,连同健忘症患者的研究结果,为海马体在编码情景记忆中的重要作用提供了强有力的支持。 fMRI 的发现还扩展了健忘症患者的发现,表明情景记忆的成功形成取决于额顶叶网络和内侧颞叶之间的相互作用。 然而,由于内侧颞叶是一个大型结构,关键目标是了解其不同分区的作用。 此类信息由使用更强大的大脑扫描技术的更高分辨率的 fMRI 研究提供。 这些研究表明,海马体内外的不同亚区对记忆编码的不同方面有贡献。 因此,虽然海马体周围的一些皮质区域对于物体识别(鼻周皮质)特别重要,但其他区域对于编码空间上下文(海马旁皮质)非常重要。 这些皮层区域为海马体提供了强大的(但间接的)输入,海马体被认为将空间和物体信息结合在一起,形成统一的记忆。

内侧颞叶和广泛分离的皮层区域之间的相互作用也是记忆巩固和检索的核心。 最初认为海马体对于检索并不重要,因为内侧颞叶被手术切除的患者 H.M. 仍然可以回忆起童年记忆。 事实上,早期的观察表明 H.M. 直到他手术前几年,他都能回忆起他生活中的许多经历。 H.M. 的这些观察。 和其他内侧颞叶受损的健忘症患者表明,旧的记忆必须通过与内侧颞叶的相互作用最终存储在其他各种皮质区域中。 然而,尽管像 H.M. 这样的海马损伤患者 有一定的回忆旧记忆的能力,有证据表明这些患者的记忆回忆程度可能受损。 目前的想法表明,存在一个涉及多个大脑区域的用于合并和检索的分布式电路,海马体在编码和检索过程中的关联绑定中起着至关重要的作用。 皮层区域作为构成记忆的独立信息元素的长期储存库,以及记忆本身内容的受控检索和重新激活。

与编码研究一样,情景知识检索研究涉及联合皮层、额顶叶网络和内侧颞叶的特定区域。 与情景记忆相关的上下文或事件细节的检索也涉及海马体的活动,内侧颞叶检索过程促进了编码过程中存在的新皮质表征的激活。

由于与大脑活动相关的血流变化的时间过程相对较慢,fMRI 扫描的时间分辨率相当有限。 为了获得更高的大脑活动时间分辨率,研究人员可以使用细胞外电极记录人脑的电活动。 这种记录很少见,只有在已经因医学原因(例如严重癫痫)接受脑部手术的人类患者中才有可能,当时使用电极植入来定位癫痫发作的部位。 在一项研究中,使用放置在内侧颞叶和皮质其他区域的硬膜下电极测量颅内脑电图 (iEEG) 信号。 受试者首先学习成对单词之间的关联,然后必须检索这些关联的记忆。 记忆的恢复与海马体中的神经活动有关,并与颞叶联合皮层中的神经活动相关,颞叶联合皮层是一个涉及语言和多感官整合的区域。 这种耦合的神经活动与皮层模式的重新激活有关,这些模式最初是在参与者第一次记住单词对时观察到的。 这一发现提供了记忆初始编码期间在海马体中观察到的神经活动与检索期间时间关联皮层中后来耦合活动之间的联系。 许多人体功能成像研究报告了在检索过程中重新激活编码模式的相关观察,记录了这种影响的普遍性。 与情景记忆的编码一样,提取涉及内侧颞叶和分布式皮层区域之间的复杂相互作用,包括额顶叶网络和其他高级关联区域。

\subsection{情景记忆有助于想象力和目标导向的行为}
记忆使我们能够利用过去的经验来预测未来的事件,从而促进适应性行为。 就像记忆的检索一样,对未来事件的想象涉及从记忆中构建细节。 记忆和想象之间可能存在联系的第一份报告来自患者 K.C. 的个案研究,正如 Endel Tulving 在 1985 年报道的那样。 由于海马体和内侧颞叶受损,他表现出典型的毁灭性失忆症。 与患者 H.M. 相似,他完全缺乏情景记忆,但语言和非情景功能未受损。 图尔文的研究进一步表明,这种脑损伤与丧失想象未来事件的能力有关。 当被问到他第二天会做什么时,K.C. 无法提供详细信息。

海马体在想象未来事件中的重要性也体现在 fMRI 研究中。 此类研究检查了健康人的大脑活动,将受试者被要求记住过去的事件(例如,想想你去年的生日)时的活动与他们想象未来事件(例如,想象明年夏天的海滩度假)时的活动进行比较 ). 受试者被要求报告想到的事件的任何生动细节。 MRI 扫描显示在记忆检索和对未来事件的想象过程中活跃的大脑区域网络中存在惊人的重叠。 该网络包括海马体、前额叶皮层、后扣带皮层、压后皮层以及外侧顶叶和颞叶区域(图 52-7)。

支持情景记忆和海马体功能对于规划未来行为所必需的观点的进一步证据来自使用虚拟现实模拟对空间导航任务的人类表现的研究。 高分辨率 fMRI 和多体素模式分析(第 6 章)表明海马体的活动与导航目标的模拟有关。 此外,计划期间的海马体活动与前额叶、内侧颞叶和内侧顶叶皮层的目标相关活动共变(图 52-8)。

情景记忆的编码和存储也受到事件适应性价值的影响。 Alison Adcock 及其同事表明,对潜在奖励的预期可以通过引发富含多巴胺神经元的内侧颞叶和中脑区域之间的协调活动来增强记忆力。 奖励也可以追溯增强记忆。 当人类参与者在迷宫中寻找奖励时,他们对奖励前发生的中性事件有更好的记忆。 根据结果追溯塑造情景记忆的能力很重要,因为特定事件的相关性可能只有在事后才知道。 连同情景记忆在构建过去事件的检索以及想象和模拟未来事件中的作用,关于奖赏的研究结果支持这样的观点,即情景记忆的主要功能是指导适应性行为。

\subsection{海马体通过建立关系关联来支持情景记忆}
除了海马体在情景记忆、未来思维和目标导向行为中的广泛作用外,对啮齿动物的研究首先指出了海马体在空间导航中的作用(第 54 章),这些发现后来得到非人类研究的支持 灵长类动物和人类。 在啮齿动物中,海马体中的单个神经元编码特定的空间信息,海马体的损伤会干扰动物的空间位置记忆。 健康人大脑的功能成像显示,当回忆空间信息时,右侧海马体的活动会增加,而当回忆单词、物体或人物时,左侧海马体的活动会增加。 这些生理学发现与临床观察结果一致,即右侧海马体的损伤不同地引起空间定向问题,而左侧海马体的损伤不同地导致言语记忆缺陷。

海马体支持空间处理、语义记忆和情景记忆这一事实引发了关于海马体如何促成这些不同行为的问题。 Howard Eichenbaum 和 Neal Cohen 提出的一个令人信服的理论表明,海马体提供了一种形成和存储复杂多模式关联的通用机制。 根据这种观点,海马体在记忆中结合了经验的独立元素,将事件编码为空间和时间上下文中项目的关系图,从而构成了一个“记忆空间”,可以区分不同的情节或事件序列,即使当 相同(或相似)的事件发生在不同的情节中(图 52-9)。 正如本章后面所讨论的,海马体编码关系的观点提供了对记忆建立机制的见解,并解释了为什么在某些情况下,海马体可能有助于无法有意识地访问但确实编码关系的记忆过程。


\section{内隐记忆支持人类和动物的一系列行为}
正如外显记忆以多种方式引导行为一样,非外显形式的记忆(那些没有意识的记忆)也可以通过多种方式影响行为。 内隐记忆是指在没有意识的情况下指导行为的知识形式。 例如,启动是暴露于一个提示对处理后续提示的自动影响。

启动可分为概念启动或知觉启动。 概念启动提供了对与任务相关的语义知识的增强访问,因为以前已经使用过该知识。 它与有助于语义知识初始检索的左前额叶区域活动减少相关。 相比之下,知觉启动发生在特定的感觉形态中,并依赖于皮层模块,这些模块根据关于单词和物体的形式和结构的感觉信息进行操作。

皮层单峰感觉区域的损伤会损害特定模态的知觉启动。 例如,一名右枕叶有广泛手术损伤的患者未能表现出对单词的视觉启动,但具有正常的外显记忆(图 52-10)。 这种情况与在 H.M. 等健忘症患者中发现的情况相反,表明启动的神经机制与外显记忆的神经机制不同。 由于内侧颞叶损伤导致的健忘症患者的知觉启动可能完好无损,这一事实进一步表明它与外显记忆不同。

\subsection{不同形式的隐式记忆涉及不同的神经回路}
其他形式的内隐记忆有助于习惯和运动、知觉和认知技能的学习,以及条件反应的形成和表达。 一般来说,这些形式的内隐记忆的特点是增量学习,随着重复逐渐进行,在某些情况下,是由强化驱动的。

习惯、运动技能和条件反应的学习可以独立于内侧颞叶系统进行。 例如,H.M. 能够获得新的视觉运动技能,如镜像追踪任务(见图 52-3)。 因此,早期的理论认为这些形式的记忆通常不依赖于内侧颞叶,而是依赖于基底神经节和小脑(见第 37 和 38 章)。 然而,随后的研究表明,这不是一般规则,内侧颞叶是存储关系关联的内隐学习形式所必需的,即使这种关联是通过重复学习的,并且似乎是在没有意识的情况下发生的。

现在认为有几种增量内隐学习涉及内侧颞叶。 例如,Turk-Browne 及其同事研究了视觉线索之间规律性的内隐学习,称为统计学习。 在典型的统计学习任务中,人类受试者会收到一系列声音或图像,这些声音或图像遵循结构化序列或重复的“语法”。 序列的学习通常通过与非重复序列相比对重复序列更快的反应时间来衡量。 乍一看,统计学习似乎不应该涉及内侧颞叶:学习是非语言的,不需要有意识的思考,因此是隐含的,并且假设它反映了跨多个事件的概率关系的累积计算, 而不是某一集的具体记忆。 然而,fMRI 研究表明,海马体在统计学习过程中是活跃的,并且已经发现内侧颞叶的损伤会损害这项隐性任务的表现。

统计学习是学习如何通过重复发生的一个例子。 新的感知、运动或认知能力也通过重复学习。 通过练习,性能变得更准确、更快,并且这些改进可以推广到学习新信息。 技能学习从认知阶段转移到自主阶段,在认知阶段知识被明确表示,学习者必须非常注意表现,自主阶段可以在没有太多有意识的关注的情况下执行技能。 例如,驾驶汽车最初需要有意识地了解技能的每个组成部分,但在练习之后,人们不再关注各个组成部分。

感觉运动技能的学习取决于许多大脑区域,这些区域随着所学习的特定关联而变化。 正如我们在第 38 章中了解到的,这些包括基底神经节、小脑和新皮质。 帕金森病和亨廷顿病患者的基底神经节功能障碍会影响运动技能的学习。 患有小脑病变的患者也难以获得一些运动技能。 健康个体在感觉运动学习过程中的功能成像显示基底神经节和小脑的活动及其与皮质区域的连接发生变化。 Danielle Bassett 及其同事使用应用于全脑 fMRI 数据的网络分析算法来表征在运动技能学习期间发生的网络功能连接的动态变化。 最后,熟练的行为可能取决于运动新皮层的结构变化,如音乐家手指的皮层表征的扩展所示(第 53 章)。

习惯是从提示或行动与奖励结果的反复关联中产生的。 人类的习惯学习是通过涉及刺激-奖励关联的增量学习的任务来研究的。 在一项典型的任务中,受试者会进行一系列试验,要求他们在视觉提示中进行选择,并逐个试验地接受有关他们选择的反馈。 提示和反馈之间的关系在任务过程中可能会发生变化,因此参与者必须根据反馈不断更新他们的反应。 因为学习是在多次试验中进行的,所以对任何一个特定试验的明确记忆对于成功的表现可能不如逐渐积累刺激-结果关联的反馈驱动学习那么有用。

fMRI 研究表明,刺激-奖励关联的增量学习取决于纹状体,即接收来自新皮质输入的基底神经节区域及其调节性多巴胺能输入。 纹状体多巴胺缺失的患者,如帕金森病患者,在逐项强化试验的基础上学习效果较差。 这些发现与其他研究一致,表明多巴胺在调节强化学习的皮质-纹状体回路中具有重要作用(见第 38 章)。

乍一看,刺激奖赏学习似乎恰恰是一种不依赖于内侧颞叶的学习:它是隐性的而不是显性的,它是逐渐发生的而不是通过对单个事件的显性记忆。 事实上,早期的理论假设学习概率刺激-奖励关联并不依赖于内侧颞叶。 然而,随后的工作表明,在某些情况下,海马体确实有助于刺激-奖励学习,例如当任务需要学习更复杂的刺激-刺激关联时(图 52-11)。 海马体对内隐学习的贡献是通过与其他皮层和皮层下回路的相互作用发生的。 fMRI 研究显示海马体和纹状体之间的功能连接支持跨各种任务的学习。 海马体和纹状体之间的相互作用有时是竞争性的,有时是合作性的,这取决于任务的要求。

\subsection{内隐记忆可以是关联的或非关联的}
一些形式的内隐记忆也在非人类动物身上进行了研究,这些动物研究区分了两种类型的内隐记忆:非联想和联想。 通过非联想学习,动物可以了解单一刺激的特性。 通过联想学习,动物可以了解两种刺激之间或刺激与行为之间的关系。 我们将在下一章考虑动物内隐记忆的细胞机制。

当受试者一次或多次暴露于单一类型的刺激时,就会产生非联想学习。 两种形式的非联想学习在日常生活中很常见:习惯化和敏感化。 习惯化是当反复出现良性刺激时发生的反应减少。 例如,大多数美国人在独立日第一次听到鞭炮声时会感到吃惊,但随着时间的推移,他们会习惯这种噪音而不会做出反应。 敏化(或伪条件反射)是在呈现强烈或有害刺激后对各种刺激的增强反应。 例如,动物在受到痛苦的挤压后会对轻微的触觉刺激做出更强烈的反应。 此外,敏化刺激可以超越习惯化的影响,这一过程称为去习惯化。 例如,在习惯化减少了对噪音的惊吓反应后,可以通过用力捏一下来恢复对噪音的反应强度。

对于致敏和去习惯,刺激的时间并不重要,因为不必了解刺激之间的关联。 相反,对于两种形式的联想学习,关联刺激的时机至关重要。 经典条件反射涉及学习两种刺激之间的关系,而操作条件反射涉及学习有机体行为与该行为的后果之间的关系。

俄罗斯生理学家 Ivan Pavlov 在 1900 年代初首次描述了经典条件反射。 经典条件反射的本质是两种刺激的配对:条件刺激和非条件刺激。 选择条件刺激 (CS),例如光、音调或触摸,是因为它不会产生明显的反应或通常与最终将学会的反应无关的微弱反应。 选择无条件刺激 (US),例如食物或电击,因为它通常会产生强烈且一致的反应(无条件反应),例如流涎或缩回肢体。 无条件的反应是天生的; 它们是在没有学习的情况下产生的。 在 US 之后重复呈现 CS 会逐渐引发一种新的或不同的反应,称为条件反应。

一种解释条件作用的方法是,CS 和 US 的重复配对导致 CS 成为 US 的预期信号。 有了足够的经验,动物会对 CS 做出反应,就好像它在期待 US 一样。 例如,如果一盏灯之后反复出现肉,最终看到灯本身就会使动物垂涎三尺。 因此,经典条件反射是动物学习预测事件的一种方式。

如果在没有 US 的情况下重复呈现 CS,则已建立的条件反应发生的可能性会降低。 这个过程被称为灭绝。 如果与食物配对的光后来在没有食物的情况下反复出现,它会逐渐停止引起流涎。 灭绝是一种重要的适应机制; 动物继续对不再有意义的线索做出反应是不适应的。 现有证据表明,灭绝与遗忘不同; 相反,学到了一些新东西——CS 现在发出美国不会出现的信号。

多年来,心理学家认为只要 CS 在关键时间间隔内先于 US,就会产生经典条件反射。 根据这种观点,每当 CS 后面跟着 US(强化刺激)时,刺激和反应的内部表征之间或一种刺激和另一种刺激的表征之间的联系就会得到加强。 连接的强度被认为取决于 CS 和 US 的配对数量。 现在大量证据表明,经典条件反射不能简单地用两个事件或刺激相继发生这一事实来充分解释(图 52-12)。 事实上,仅仅依赖顺序是不适应的。 相反,所有能够进行联想调节的动物,从蜗牛到人类,都记得关联事件之间的显着关系。 因此,经典条件作用,或许还有所有形式的联想学习,使动物能够区分可靠地一起发生的事件和那些只是随机关联的事件。

大脑多个区域的损伤会影响经典条件反射。 一个经过充分研究的例子是保护性眨眼反射的调节,这是一种运动学习形式。 一股空气吹到眼睛自然会引起眨眼。 可以通过将粉扑与粉扑之前的音调配对来建立有条件的眨眼。 对兔子的研究表明,条件反应(对音调的眨眼反应)会被两个部位中任一处的损伤所消除。 对小脑蚓部的损伤会消除条件反应,但不会影响非条件反应(眨眼以响应一股空气)。 有趣的是,小脑同一区域的神经元表现出依赖于学习的活动增加,这与条件行为的发展密切相关。 小脑深部核 interpositus nucleus 的损伤也会消除条件性眨眼。 因此,小脑蚓部和深部核团在调节眨眼以及涉及骨骼肌运动的其他简单形式的经典调节中都起着重要作用。

另一个经过充分研究的例子是恐惧条件反射,它取决于杏仁核。 在恐惧条件反射中,中性提示(例如语气)与令人厌恶的结果(例如震惊)配对。 这种配对会导致条件性恐惧反应,其中中性语气会单独引发行为反应,例如冻结。 恐惧条件反射取决于杏仁核亚核输入和连接的可塑性,尤其是基底外侧杏仁核,我们将在下一章讨论。

\subsection{操作性条件反射涉及将特定行为与强化事件相关联}
由 Edgar Thorndike 发现并由 B. F. Skinner 等人系统研究的联想学习的第二个主要范例是操作条件反射(也称为试错学习)。 在操作性条件反射的典型实验室示例中,一只饥饿的老鼠或鸽子被放置在一个试验室中,在该试验室中动物会因特定行为而获得奖励。 例如,腔室可具有从一个壁突出的杠杆。 由于之前的学习,或通过玩耍和随机活动,动物会偶尔按下杠杆。 如果动物在按下杠杆后立即接受正强化物(例如,食物),它将开始比自发率更频繁地按下杠杆。 动物可以被描述为已经学会在它的许多行为(例如,梳理、饲养和行走)中,一种行为之后是食物。 有了这些信息,动物很可能会在饥饿时按下控制杆。

如果我们将经典条件反射视为两种刺激(CS 和 US)之间预测关系的形成,则操作性条件反射可被视为行动与结果之间预测关系的形成。 与测试反射对刺激的反应性的经典条件反射不同,操作性条件反射测试自发发生或没有可识别刺激的行为。 因此,据说操作性行为是发出的而不是引发的。 一般来说,获得奖励的行为往往会被重复,而伴随着厌恶行为的行为虽然不一定是痛苦的,但其后果往往不会重复。 许多实验心理学家认为,这个被称为效果法则的简单概念支配着许多自愿行为。

操作性条件反射和经典条件反射涉及不同类型的关联——分别是动作与奖励之间或两种刺激之间的关联。 然而,操作性条件反射和经典条件反射的规律非常相似。 例如,时机对两者都至关重要。 在操作性条件反射中,强化物通常必须紧跟操作性动作。 如果强化剂延迟太久,只会发生弱调节。 同样,如果 CS 和 US 之间的间隔太长或者如果 US 在 CS 之前,则经典调节通常很差。

\subsection{联想学习受到生物体生物学的限制}
动物通常学会将与其生存相关的刺激联系起来。 例如,动物很容易学会避免某些食物,这些食物随后会产生负强化(例如,毒药引起的恶心),这种现象称为味觉厌恶。

与大多数其他形式的条件反射不同,即使在长时间延迟后发生非条件反应(毒物引起的恶心)时,味觉厌恶也会发生,直到 CS(特定味道)后数小时。 这在生物学上是有道理的,因为受感染的食物和天然产生的毒素的不良影响通常是在经过一段时间的摄入后才产生的。 对于包括人类在内的大多数物种,只有当某些口味与疾病相关时才会发生味觉厌恶调节。 如果味觉之后是不产生恶心的痛苦刺激,则味觉厌恶发展不佳。 此外,动物不会对伴随恶心的视觉或听觉刺激产生厌恶。


\section{记忆中的错误和缺陷揭示了正常的记忆过程}
记忆让我们重温个人的过去; 提供对事实、关联和概念的庞大网络的访问; 并支持学习和适应行为。 但记忆并不完美。 我们经常会迅速或逐渐忘记事件,有时会扭曲过去,偶尔会记住我们宁愿忘记的事件。 在 1930 年代,英国心理学家弗雷德里克·巴特利特 (Frederic Bartlett) 报告了人们阅读并试图记住复杂故事的实验。 他表明,人们经常会记错故事的许多特征,经常会根据他们对本应发生的事情的预期来歪曲信息。 遗忘和扭曲可以提供对记忆运作的重要见解。

记忆的缺陷被分为七个基本类别,被称为“记忆的七大罪”:短暂性、注意力不集中、阻塞、错误归因、易受暗示、偏见和坚持。 在这里,我们重点关注其中的六个。

心不在焉是由于缺乏对直接经验的关注。 编码过程中的注意力不集中可能是常见记忆失败的根源,例如忘记最近放置对象的位置。 当我们忘记执行某项特定任务(例如在从办公室回家的路上买杂货)时,也会出现心不在焉,即使我们最初对相关信息进行了编码。

阻塞是指暂时无法访问存储在内存中的信息。 人们通常对一个受欢迎的词或图像有部分意识,但仍然无法准确或完整地回忆起整个词。 有时,感觉就像一个被阻塞的词在“舌尖”——我们知道这个词的首字母、其中的音节数或一个发音相似的词。 确定哪些信息是正确的,哪些是不正确的需要大量的有意识的努力。

心不在焉和阻塞是遗漏的罪过:在我们需要记住信息的时刻,它是无法访问的。 然而,记忆的特征还在于犯下错误,即存在某种形式的记忆但错误的情况。

错误归因是指将记忆与不正确的时间、地点或人联系起来。 错误识别是一种错误归因,当个人报告说他们“记得”从未发生过的项目或事件时,就会发生这种情况。 这种虚假记忆已在受控实验中得到记录,在这些实验中,人们声称看到或听到了以前没有出现过但在含义或外观上与实际出现的相似的词语或物体。 使用正电子发射断层扫描成像和 fMRI 的研究表明,许多大脑区域在正确识别和错误识别过程中表现出相似的活动水平,这可能是错误记忆有时感觉像真实记忆的原因之一。

暗示性是指将新信息纳入记忆的倾向,通常是由于对可能经历过的事情提出引导性问题或建议的结果。 使用催眠暗示的研究表明,可以将各种虚假记忆植入高度易受暗示的个体中,例如记住在晚上听到很大的噪音。 对年轻人的研究还表明,反复暗示童年经历可以产生对从未发生过的事件的记忆。 这些发现在理论上很重要,因为它们强调记忆不仅仅是对过去经历的“回放”(方框 52-1)。 尽管有这些重要的理论和实践意义,但人们对暗示性的神经基础几乎一无所知。

偏见是指对记忆的扭曲和无意识影响,反映了一个人的一般知识和信念。 人们常常错误地记住过去,以使其与他们目前所相信、所知道或所感受到的相一致。 这个想法与研究支持的“预测编码”的想法是一致的,这些研究表明即使是感知和感觉的低级神经机制也是由期望塑造的。 期望影响记忆的特定大脑机制尚不清楚。

持久性是指强迫性记忆,不断记住我们可能想要忘记的信息或事件。 神经影像学研究阐明了一些有助于持久情绪记忆的神经生物学因素。 一些关键结果与杏仁核有关,杏仁核是靠近海马体的杏仁状结构,长期以来已知参与情绪处理(第 42 章)。 研究表明,故事情感成分的回忆水平与故事呈现过程中杏仁核的活动水平相关。 相关研究表明杏仁核参与编码和检索情绪激动的经历,这些经历会反复侵入意识。

尽管坚持可能会导致残疾,但它也具有适应性价值。 对令人不安的经历的持续记忆增加了我们在某些时候可能对生存至关重要的时候回忆起有关唤醒或创伤事件的信息的可能性。

事实上,许多记忆缺陷可能具有适应性价值。 错误记忆和易受暗示可能都与记忆最基本的适应性功能之一有关:将在时间上分离的经验整合到学习联想网络中。 记忆要在指导未来行为方面发挥重要作用,就必须具有灵活性,这样即使情况发生变化,我们也可以利用过去的经验对未来事件进行推断。 同样,虽然各种形式的遗忘(短暂、心不在焉和阻塞)可能令人讨厌,但自动保留每一次经历的每一个细节的记忆系统可能会导致无用的琐事堆积如山。 这正是俄罗斯神经心理学家亚历山大·卢里亚 (Alexander Luria) 研究并在《助记者的心灵》一书中描述的助记者谢列舍夫斯基 (Shereshevski) 这个引人入胜的案例所发生的事情。 Shereshevski 对他过去的经历充满了非常详细的记忆,但无法概括或抽象地思考。 一个健康的记忆系统不会对每一次经历的所有细节进行编码、存储和检索。 因此,转瞬即逝、心不在焉和阻塞让我们避免了谢列舍夫斯基的不幸命运。


\section{亮点}

1. 不同形式的学习和记忆可以在行为和神经上加以区分。 工作记忆在短期内保持与目标相关的信息。 外显(或陈述性)记忆涉及两类知识:情景记忆(代表个人经历)和语义记忆(代表一般知识和事实)。 内隐记忆包括知觉和概念启动的形式,以及运动和知觉技能的学习、知觉规律和强化的习惯。 

2. 新外显记忆的编码、存储、检索和巩固取决于新皮质和内侧颞叶特定区域与特定海马亚区之间的相互作用。 外显记忆的长期存储的启动需要颞叶系统,正如 H.M. 巩固过程稳定了存储的表征,使外显记忆减少对内侧颞叶的依赖。 外显记忆的提取涉及内侧颞叶,以及促进注意力和认知控制的额顶叶网络。 

3. 多个进程交互以支持内存引导行为。 情景记忆的检索指导对未来事件的想象,这对于做出关于未来选择和行动的决定很重要。 通过增强编码、存储和整合过程,具有重要激励意义的事件在记忆中被优先排序。 动机也可能通过不同的优先排序机制影响检索。 

4. 内隐记忆在感知、思考和行动过程中自动出现。 它往往是不灵活的,甚至在没有意识的情况下也表现在任务的执行中。 内隐记忆涉及各种各样的大脑区域和回路,包括支持特定感知、概念或运动系统的皮层区域,这些系统被招募来处理刺激或执行任务,以及纹状体和杏仁核。 涉及关系关联编码的内隐学习还涉及海马体。 

5. 记忆中的缺陷和错误提供了有关学习和记忆机制的线索。 过去可以被遗忘或扭曲,这表明记忆并不是对每一次经历的所有细节的忠实记录。 恢复的记忆是不同大脑区域之间复杂相互作用的结果,并且可以随着时间的推移受到多种影响而重塑。 各种形式的遗忘和扭曲告诉我们很多关于记忆的灵活性,它允许大脑适应物理和社会环境。
\subsection{荐读}
\subsection{参考文献}