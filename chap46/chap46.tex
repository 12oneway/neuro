\chapter{神经细胞的分化与存活}


\section{神经祖细胞的增殖涉及对称和不对称细胞分裂}

\section{放射状胶质细胞充当神经祖细胞和结构支架}

\section{神经元和神经胶质细胞的生成受 Delta-Notch 信号和基本螺旋-环-螺旋转录因子的调节}

\section{大脑皮层的层数是通过新生神经元的顺序添加而建立的}

\section{神经元从它们的起源位置长距离迁移到它们的最终位置}
\subsection{兴奋性皮层神经元沿神经胶质指南径向迁移}
\subsection{皮层中间神经元在皮层下出现并切向迁移到皮层}
\subsection{周围神经系统中的神经嵴细胞迁移不依赖于支架}

\section{结构和分子创新是人类大脑皮层扩展的基础}

\section{内在程序和外在因素决定神经元的神经递质表型}
\subsection{神经递质的选择是神经元分化转录程序的核心组成部分}
\subsection{来自突触输入和目标的信号可以影响神经元的递质表型}

\section{神经元的存活受来自神经元靶标的神经营养信号的调节}
\subsection{神经生长因子的发现证实了神经营养因子假说}
\subsection{神经营养因子是研究最透彻的神经营养因子}
\subsection{神经营养因子抑制潜伏细胞死亡程序}

\section{要点}
\subsection{选读}
\subsection{参考文献}